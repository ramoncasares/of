% OL3.TEX (RMCG20000320)

\Part Interlude

\Section Conceiving the Object

Before we begin our return trip, we will pause here to reconcile
everything that we have seen on our way in. While we were preparing the
trip before we started out, we stopped to look at some prolegomena in
which we over-simplified some points. And even though the two main
affirmations of the prolegomena, that objective ^{reality} is subjective
and that reality is involuntary, have been confirmed at the entry, we
now know that, contrary to what we stated at that time, there are
objects that are posterior to ^{perception}.

What happened is that what we called an ^{object} in the prolegomena,
turned out, upon more detailed observation, to be a ^{thing}. A thing is
a practical object, and so is foreign to the ^{will}. But there are
other objects, concepts_{concept}, that are theoretic and voluntary.
Thus, it is not correct to state that all objects are involuntary and
previous to the word, since there are voluntary ones, created with
words, or more exactly, with ideas_{idea}.

We must, therefore, rectify what we said in ^>Reality Is Involuntary>.
It turns out that only the reality of things is involuntary. This
reality is constructed, as we saw in ^>Things and Concepts>, by
perception, learning, and emotion, but without the intervention of
thought; and this reality coincides with the subject's reality, but not
with its world, which also includes theoretic concepts generated at
will.

Since there are voluntary and involuntary objects in the subject's
^{world}, a first explanation can establish that the involuntary objects
are autonomous and independent of the subject, while the voluntary ones
have no existence outside of the subject's thoughts. Thus,
^{objectivism} founded the essential difference between the real world
of things and the theoretic world of concepts. But, as we have seen, the
difference is not essential; rather it is merely circumstantial or
genetic, since it has its origin in the peculiar evolution of the
nervous system that turned man into a subject.


\Section Contingencies

Let us take advantage of this pause to state another reservation. We
have described the entry path as if each step were an inevitable
consequence of the previous one, but this is not so. It mustn't be
thought, for example, that the learner is necessarily followed by the
knower. The only requirement for a new step is that it improve the
previous steps in some ^{niche}, that is, given certain conditions that
are fulfilled at some time and place. If this is so, it is possible, but
not certain, that ^{evolution} will take advantage of the improvement in
this niche. On the other hand, out of all of the possible evolutionary
sequences for ^{cognition}, we have tried to describe the one that
produced \latin{homo sapiens}.

We will use the rest of this pause to reflect upon the concept of
explanation, taking time in particular to study the difficulties that
the ^{entry path explanation} presents in order to prepare the exit
path.


\Section Down with Materialism!

If an ^{explanation} is an explanation, it cannot appeal to any act of
^{faith}. If any step of the explanation, no matter how small, needs
faith to get through it, then it is not explained. If a detail of an
explanation, even a tiny one, is inexplicable, then the explanation is
not complete. And a partial explanation is no explanation; it is just a
more precise restatement of the ^{problem}. All of these obvious matters
are, I believe, what have given ^{materialism} a preponderant position
in ^{science}.

Some scientists are not materialists, but this is only because
present-day specialization allows them to think that, even though
everything can be explained in their own field of investigation, there
are phenomena that are impossible to understand in other fields, either
because ^{God} is ineffable or because people are free. But in the most
basic sciences, even this is not possible. This is what leads
^[Hawking]^(Hawking1988), for example, to conclude by denying God any
possibility of choice, and ^[Minsky]^(Minsky1985) to end up by denying
^{freedom} of ^{will} because, he alleges, everything is ^{cause} and
^{chance}, as ^[Monod] said. In the following pages, I will try to show
that, in spite of its good intentions, materialism cannot be correct.

^[Descartes]^(Descartes1641), who went back to the first principles, to
what was clear and distinct, asked the right question. ``I~think,
therefore I~am''.  That is, what comes first is the word, and reality
comes after. Besides asking the right question, ^[Descartes] answered it
correctly. Material things, that really exist, can be described as
machines_{machine}, like mechanical clocks_{clock}, but my thinking and
speaking freely cannot be assimilated to a machine, as
^[Chomsky]^(Chomsky1966) pointed out.

Standing on ^[Descartes]' shoulders, ^[Newton] was able to describe the
universe as an enormous precision clock. But even though ^[Newton] did
not want to feign hypotheses, his clock was not mechanical, at least
according to Cartesian specifications. Action at a distance_{action at a
distance} threw ^{matter} into great ontological difficulties, and it
has been arduous to define what matter is ever since. Despite these
inconveniences, Newtonian physics, capable of spectacular
predictions_{prediction}, took over so completely that all doubts were
forgotten and materialism came to rule once and for all in science.

But what is materialism?  Perhaps its most famous slogan is the one that
we have already quoted and that takes up a saying of old
^[Democritus]^(Monod1970): ``Everything is chance and ^{necessity}''. It
must be said, however, that no materialist admitted this before the
advent of ^{quantum mechanics}, and that even
^[Einstein]^(Einstein1936), one of its precursors, always denied chance.
What this means, in short, is that materialists hold that physical
science_{physics} is what offers the final explanation. Thus, it can be
understood that what is material changes as physics evolves: first the
idea of action by contact, later the fields of energy, and even later,
chance, to give just three examples. So that another slogan such as
`everything is physical' or `the world is physical' may be more adequate
for materialism.

Materialism puts the explanations and, in turn, the sciences that
produce them, in order of importance. Physics, according to the
materialist postulate, provides the definitive explanations, and so is
the most fundamental, basic, or important (the `hardest') kind of
knowledge. After physics comes ^{chemistry}, with ^{biology} afterwards.
Last of all, the materialist order places ^{psychology} and the
^{humanities}, but only when they can be reduced to biology, chemistry,
and physics. Since everything can be finally reduced to physics for
materialism, ^{free will}, ^{consciousness}, and ^{self} turn out to be
mere illusions or figures of speech, and, at any rate, they have no
influence upon reality.

This is all absurd and untenable. Because explanations, just like the
materialist explanation itself, are no more than symbolic
expressions_{syntactic expression}. And symbolic expressions only have
meaning for symbolic subjects_{subject}, that is, for the selves.
Because of this, only the symbolic subjects, that is, the selves, are
interested in producing them. Let us give a graphic example. Imagine
that humanity disappears from the Earth and that, as a result, there are
no more selves left. With no one to interpret words, and since the
relationship between the sequence of letters and the meaning of the word
is the fruit of established conventions between subjects, do you think
that it would matter whether this printed paper that you are reading
said what it says or meant any other thing?  So materialism provides an
explanation that, if it were coherent with itself, would have no meaning
whatsoever. Materialism is ^{absurd}.

Perhaps you may suspect that the only alternative to materialism, which
is to ^{ontology} what monotheism is to religion, is ^{dualism}. It is
not the only alternative. Not if we abandon objectivism and adopt
^{subjectivism}. The trick, and it isn't exactly a trick either,
consists of noticing that, since all objects are representations or
images that do not exist outside of our heads, it is not relevant
whether they are formed of only one type of ^{substance} or of two
types. In other words: ^{epistemology} is previous to ontology.

The ^{world} is symbolic, that is, it is made up of two layers:
^{syntax}, where rational thought with concepts is found, and
^{semantics}, that we compare with physical things, with real things.
^[Descartes] got this far, and at this point he had to postulate a
\latin{^{res cogitans}} as opposed to the \latin{^{res extensa}} in
order to resolve, because he didn't really solve the problem of defining
the world. We are luckier. ^[Turing] equipped us with tools that allow
us to solve this problem. In 1936, ^[Turing]^(Turing1936) demonstrated
that symbolism, or more exactly, a ^{syntax engine}, can be physically
constructed. Each and every ^{computer} is palpable proof of the truth
of his theoretical demonstration.

Both propositions, `it is possible to physically construct a syntax
engine' and `syntax can have real physical effects', that is, it can
have meaning, are logically equivalent. Both propositions are
equivalent, and even so, as the undoubtable self is in the syntactic
layer, which is the layer of symbolic thought and speech, the second
proposition is preferable.

This conception of the world refutes the materialist postulate, because
it shows that it is not true that everything is physical. For example,
the self is not physical, it is syntactic, and it is part of the world;
it can even have real physical effects, as we have just seen.

These kinds of reasoning show, by the way, how ^{philosophy} always
follows  ^{engineering}. ^[Descartes] made use of a mechanical clock in
order to imagine the world, but we use a computer, and that is why we
can understand it a different way. We could call our way symbolic or
linguistic.


\Section Mechanisms

The entry path, because it started with the objects already established,
left out the first step: the mechanism.
$$\hbox{Mechanism} \supset \hbox{Adaptor} \supset
  \hbox{Learner} \supset \hbox{Knower} \supset \hbox{Subject}
\abovedisplayskip=9pt
\belowdisplayskip=9pt
$$

A ^|mechanism| is anything that interacts with its ^{environment}, that
is, its only characteristic is that it has a ^{behavior}. All living
beings, even plants_{plant}, are mechanisms.
$$\hbox{\strut Phenomenon}
  \underbrace{\strut \longrightarrow}_{\hidewidth
   \hbox{\strut Mechanism}\hidewidth}
  \hbox{\strut Action}
\abovedisplayskip=9pt plus 3pt
\belowdisplayskip=9pt plus 3pt
$$
A mechanical clock is also a mechanism. A ^{computer}, as we saw in
^>Reality>, is capable of different behavior for every ^{program} that
it can execute and is, therefore, capable of imitating different
mechanisms, for example, a clock;  the computer itself, however, is
another mechanism.

We have already presented the other stages of epistemological evolution.
An ^{adaptor} is a mechanism with two parts, body and nervous system;
the nervous system is what selects, based on the present objective
reality, the behavior that the body executes. A ^{learner} is an adaptor
capable of tuning reality to its environment. A ^{knower} is a learner
that can use reality in different ways, selecting the way thanks to an
interior perception called emotion. Last of all, a ^{subject} is a
knower that has symbolic language available to use, thus permitting it
to broaden reality beyond perception and emotion, thanks to learning,
with ideal theoretic concepts.

To conclude, a subject is a mechanism with a nervous system in which it
models reality, a reality that it can use in different ways, and with
symbolic language available for use. In other words, a subject is a
mechanism with a series of characteristics that distinguish it from
other mechanisms that are not subjects.


\Section Is the Subject Free?

If we explain things from outside to inside, the ^{subject} is a
mechanism, and mechanisms are not free. Freedom_{freedom} in the subject
comes out of nothing, like magic, and this is not acceptable in an
explanation. The sequence goes through the following stages.

Is a mechanism free? No, it is not in any way free. A ^{mechanism} is
the prototype of determinism. For a mechanism, everything is ^{chance}
and ^{necessity}. Is an adaptor free? No. An ^{adaptor} is nothing more
than a mechanism in which two parts have been differentiated: the body,
that executes the behaviors, and the nervous system that selects the
behavior to execute, and both are mechanisms. And a learner? No, a
^{learner} is not free, either, because it is just an adaptor with a
nervous system that mechanically tunes into its environment. And is the
knower free? No, because a ^{knower} is nothing more than a learner that
is capable of feeling the internal needs of its own body which, I
repeat, is a mechanism. And the subject, is it free?

Is the subject free? If we look at the ^{entry path}, then the subject
is just another mechanism, too. We could say that freedom appears with
the subject, but then we would have to suppose that freedom is, somehow,
latent in the simplest mechanism that is, paradoxically, the prototype
of determinism. On the other hand, if we look at this problem in the
other direction, not from outside inwards, but from inside outwards, we
get the opposite answer. The subject, by its very nature, is free. The
subject sees itself as its \meaning{self},_{self} that is, it sees
itself as free to do its own ^{will}.

And so it happens that this matter, looked at from outside gives a
completely different impression than if we look at it from inside. This
situation is not comfortable, and so we will, in the following pages,
try to reconcile both points of view.


\Section The Stranger

The entry path, from the mechanism to the subject, follows the same
direction as ^{evolution}. This direction, from simple to complex, may
explain things better, but this path is not the one that was really
taken; it goes in exactly the opposite direction of the path that was
taken. Because the ^{question} comes first. It is only because we can
first ask that we can afterwards respond. There is an ^{explanation}
because there is a question, although, in order to ask questions and set
forth problems, things will have to get much more complicated. Things
get so complicated, in fact, that only a subject can ask questions.

It follows, then, that although ^{symbolic language} is the last thing
that appears in the entry path explanation, it is actually the
indispensable condition, not just to begin to explain things, but even
to start to ask questions. Because questions as well as the explanations
themselves are expressions of symbolic language.

I cannot help myself from considering two curious possibilities. One
possibility is that, if subjects did not exist, consequently neither
would symbolic languages. In this case, there would be no explanation
about why symbolism or anything else for that matter did not exist,
because there would be neither explanations nor questions. There would
be no explanation, nor anyone to demand one, so there would be no
tension either.

Another possibility is that there are in fact subjects, and that their
explanations of the ^{world} manage to explain everything that happens,
except the existence of symbolic language. If this were the case, the
^{subject} would see itself as excluded from the world, as if it did not
belong to it. The subject, seeing itself outside of everything that
surrounds it, would feel like a ^{stranger}. To summarize, if the
subject were not able to explain its own symbolic nature, then it would
feel perplexed, like a stranger in the world. So symbolic language needs
to be explained urgently.


\Section The Material Explanation

In principle, you can take an ^{explanation} as far as you want. All you
have to do is ask why after every explanation, like children do when
they discover how. But this is very unsatisfactory, because you
necessarily end up with a vicious circle of explanations, or else you
reach the point where you have to admit that there is no adequate
explanation. In order to avoid this, people have to agree on what needs
to be explained and what doesn't. This is so basic that the agreement is
usually tacit.

For example, the objectivist solution, the most widespread and natural
solution, establishes that things_{thing} do not need to be explained,
they simply are, and this is enough, so that only concepts_{concept}
have to be explained. And explaining concepts consists, for objectivism,
of making them fit in with things. That is, ^{objectivism} subordinates
the ^{theoretical loop} to the ^{practical loop}. This is reasonable for
the following reasons. Things are the objects of the subject that the
simple knower was already using. The simple knower from which the
subject descends was viable, as the mere existence of its descendent
proves. So it is prudent to construct new concepts upon the solid base
of things. And that is precisely what objectivism's ^{material
explanation} proposes.

As a result of this analysis, we can see that, for objectivism, the
explanation of a concept consists of thing-ifying_{thing-ify} it, that
is, referring it to things. So, for example, in order to explain
electrical phenomena, we use a thing called an ^{electron}, which needs
no ulterior explanations, although its contradictory dual nature as a
wave and as a particle suggests the contrary. When quantum physics
discovered that all things have a dual nature, it revealed that
objectivism has its limits, even if it is sufficient in practice; and
this sufficiency is underwritten by the existence of simple knowers.
Objectivism's limitations come from subordinating explanations to the
^{perception}, ^{learning}, and ^{emotion} natural to the human
^{subject}.


\Section The Automatic Explanation

When ^{physics} came up against the so-called quantum paradoxes, it
decided to go beyond the material explanation devising what we will call
the automatic explanation. We give it this name not because it
automatically obtains explanations, but because it proposes as an
^{explanation} any system of equations that permits us to mechanically
predict_{prediction} the future of the phenomenon to be explained, and
these systems can be modeled mathematically as finite automata. Thus, in
the ideal case, the ^{automatic explanation} provides an automaton that
is indistinguishable in appearance from the ^{phenomenon} explained. An
^{automaton} is a ^{mechanism}, but its physical properties are
discarded and only its capacity to deal with ^{data} is kept. So, by
defininition, an automaton is an abstract mechanism.

For the automatic explanation in present-day physics, any automatism
that allows us to predict what will happen in each case will do, even if
it does not correspond with any ^{thing}. Concretely, the quantum
explanation is a system of equations that, when specified, manages to
predict with unprecedented exactness the results of experiments (see
^[Feynman]^(Feynman1985)). The ultimate explanation is no longer the
^{electron}, but physics equations. For the automatic explanation, the
electron is a consequence of the system of equations, not the other way
around, as in the material explanation, for which the system of
equations is the result of describing the electron's behavior.

The automatic explanation improves the ^{material explanation} because
it does not give preference to things over concepts, the only reason for
this preference being the contingent evolutionary history of \latin{homo
sapiens}. On the contrary, by giving the foresightful automatism first
place, the automatic explanation abandons the meaning that was naturally
in the things from the material explanation. We must remember that real
things always have a natural ^{meaning}, but that the concepts may have
no meaning, as we saw in ^>Existence and Reference>. And so it happens
that the automatic explanation that quantum physics provides is capable
of predicting with precision and exactness, but it means nothing. For an
objectivist, the result is the same as not explaining anything; it means
preferring description to explanation. The automatic explanation does
not explain, it describes.


\Section The Entry Path Explanation

The discussion between ^[Einstein] and ^[Bohr]^(Murdoch1987) must be
understood in the context of this transition from the material
explanation's ontological position, defended by ^[Einstein], to the
pragmatic position that ^[Bohr]'s automatic explanation advocated.

From another point of view, the ^{material explanation} is completed by
the belief in a ^{God}, creator of all and legislator of the universe,
while the ^{automatic explanation} needs only the backing of a universal
legislator. If laws explain everything that occurs in the material
explanation, in the automatic explanation they govern even more, because
in this explanation the laws specify everything that happens and
everything that exists. There is no room for ^{freedom} in either,
because everything that happens is ruled by the ^{universal laws} of
nature.

\breakif1

Material explanations as well as automatic explanations are entry path
explanations_{entry path explanation}, because both are constructed with
resources that are external to the ^{self}, such as laws and things.
And, since freedom doesn't fit into either, it turns out that neither
manages to explain the ^{subject}. And therefore, neither the material
explanation nor the automatic explanation explain ^{symbolic language},
which is peculiar to subjects. In these sad circumstances, symbolism
lacks an explanation and the subject is a ^{stranger}.


\Section A Dirty Trick

In order to overcome this obstacle, the ^{theory of subjectivity}
proposes that we start over from the very beginning. This is equivalent
to going back to ^[Descartes]' ``I think'' and giving up all the
progress that had been made. The situation requires courage and
resolution and this is why I did not announce earlier the grave state in
which we find ourselves. By this time, you are already far from the
safety of home, and there is nothing for it but to find the return path,
with the risk of falling helplessly into ^{perplexity}. I am sorry, but
sometimes you have to play dirty.


\endinput
