% OF.TEX (RMCG20000104)

\mag1200
\input crops

\input metatex

\catcode`\@=11

\english %% OF
\hyphenation{piri-pili kant-ian}

% FONTS

%\let\oldoldstyle=\oldstyle
\font\tenrmj=cmrj10
\def\rm{\fam0 \tenrmj}
\rm

\font\boldstyle=cmmib10

% Fonts for titles

\font\fontzero=cmbx12 scaled\magstep4
\font\fontone=cmbx12 scaled\magstep2
\font\fonttwo=cmbx12
\font\fonttwosym=cmbsy10 scaled\magstep1
\def\Stitle{{\fonttwosym \char"78}}

\catcode`\�=13 \def�{\"o}% for G�del

\font\sc=cmcsc10
\font\msbm=msbm10

\font\frakx=eufm10
\font\frakvii=eufm7
\newfam\frakfam \textfont\frakfam=\frakx \scriptfont\frakfam=\frakvii
\def\frak{\fam\frakfam\frakx}

\font\logo=logo10


% LAYOUT

\nopagenumbers

\def\twodigits#1{\ifnum #1<10 0\fi \number#1}
\def\todayiso{\number\year \twodigits\month \twodigits\day}


\footline={\tenrmj\ifodd\pageno \hfil\folio
 \else \folio \hfil\fi\strut}

\hsize=10.5cm
\vsize=39\baselineskip \advance\vsize\topskip % 40 lines
\parskip=0pt plus 0.0001pt minus 0.0001pt

\def\titulo{Ram\'on Casares\cr On Freedom}
\setcrops

\predisplaypenalty=10
\abovedisplayskip=12pt plus 6pt minus 6pt
\belowdisplayskip=12pt plus 6pt minus 6pt
\abovedisplayshortskip=0pt plus 6pt minus 6pt
\belowdisplayshortskip=12pt plus 6pt minus 6pt


% Every point is a period
\count255=`A \loop\sfcode\count255=1000
 \ifnum\count255<`Z\advance\count255 1 \repeat
\def~{\nobreak\ \ignorespaces}

% Emphasis with automatic italic correction (\/).
% Use: {\em italic, but {\em roman}, text}.
\def\em{\ifdim \fontdimen1\font>0pt \rm
 \else \it \expandafter\aftergroup \fi \itcor}
\def\itcor{\ifhmode \expandafter\itpuncl@ok \fi}
\def\itpuncl@ok{\begingroup\futurelet\ITCt@mpa\itcort@st}
\def\itcort@st{\def\ITCt@mpb{\ITCt@mpa}%
 \ifcat\noexpand\ITCt@mpa,\setbox0=\hbox{\ITCt@mpb}%
  \ifdim\ht0<0.3ex \let\itc@rdo=\endgroup \fi\fi \itc@rdo}
\def\itc@rdo{\skip0=\lastskip \ifdim\skip0=0pt \/\else
 \unskip \/\hskip\skip0 \fi \endgroup}

% Verbatim

\def\uncatcodeall{\deactivate\dospecials\dohigh}
\def\iverb#1{\begingroup\uncatcodeall\obeyspaces\obeylines\d@iverb#1}
\def\d@iverb#1#2{\def\next##1#2{\immediate\write#1{##1}\endgroup}\next}

% Other

\def\siglo#1 {\par \noindent\hang {\sc #1} $\cdot$ \ignorespaces}

\def\point{\par\hang\indent\llap{$\bullet$\enspace}}
\def\goodbreak{\vskip0pt plus 4\baselineskip\penalty-250
 \vskip0pt plus-4\baselineskip}

\def\breakif#1{\vskip0pt plus #1\baselineskip \penalty-250
 \vskip0pt plus-#1\baselineskip}


% INDEX

% PDFTeX

\def\stringactives{\stringate\doaccents\dosymbols}

\def\nocolor{\let\Red=\relax \let\Green=\relax \let\Blue=\relax
 \let\Black=\relax}

\ifx\pdfliteral\undefined
 \def\pdfcode#1\pdfendcode{\relax\closeout-1\relax}
 %\closeout-1 is just a whatsit, i.e. an extension as pdf commands are
 \let\colors=\nocolor
\else
 \let\pdfcode=\relax \let\pdfendcode=\relax
 \def\pdfRed{\pdfliteral{0 1 1 0 k}}
 \def\pdfGreen{\pdfliteral{1 0 1 0 k}}
 \def\pdfBlue{\pdfliteral{1 1 0 0 k}}
 \def\pdfBlack{\pdfliteral{0 0 0 1 k}}
 \def\colors{\let\Red=\pdfRed \let\Green=\pdfGreen \let\Blue=\pdfBlue
  \let\Black=\pdfBlack}
\fi

%%% OPTION
\nocolor % default value
%\colors

\def\pdfgoto#1#2{\pdfcode
 \pdfstartlink attr{/Border [0 0 0]} goto num #2\pdfendcode
 \Red#1\Black \pdfcode\pdfendlink\pdfendcode}

% Numbered footnotes for PDFTeX

\newcount\footnoteno

% adapted from [416]
\def\footnote{\edef\@sf{\spacefactor\the\spacefactor}%
  \global\advance\footnoteno1 \global\advance\d@stno1
  \pdfgoto{$^{\number\footnoteno}$}{\number\d@stno}\@sf
  \insert\footins\bgroup \let\par=\endgraf
  %\interlinepenalty100
  \interlinepenalty10000 \floatingpenalty20000
  \def\"##1{\setbox0=\hbox{\accent"7F ##1}\ht0=\ht\strutbox\box0} % \"Uber
  \splittopskip\ht\strutbox \splitmaxdepth\dp\strutbox
  \lineskiplimit=-10pt %\parskip=0pt
  \leftskip\z@skip \rightskip\z@skip \spaceskip\z@skip \xspaceskip\z@skip
  \pdfcode\pdfdest num \number\d@stno fitbh\pdfendcode
  \indent\llap{$^{\the\footnoteno}\;$}\strut\futurelet\next\fo@t}

\dimen\footins=0.5\vsize
\skip\footins=\baselineskip
\def\footnoterule{\kern-3pt\hrule height 0.4pt width 3cm depth 0pt\kern2.6pt}

% OUTPUT ROUTINE

\def\pagecontents{\dimen@=\dp\strutbox \advance\dimen@-\dp\@cclv
 \advance\dimen@\ht\strutbox \advance\dimen@\dp\strutbox
 \ifnum\pageno=136 \global\tracingpages=1
  \global\showboxbreadth=\maxdimen \global\showboxdepth=1 \fi
 \ifnum\pageno=138 \global\tracingpages=0
  \global\showboxbreadth=2 \global\showboxdepth=2 \fi
 \unvbox\@cclv
 \ifvoid\footins\else\kern\dimen@\footnoterule\unvbox\footins\fi}

%\saveplain\pagebody
%\def\pagebody{\hbox{\plain\pagebody
% \ifvoid\margin\else\kern3pt\vbox to0pt{\vss\unvbox\margin}\fi}}

% PDFTOC

\newtoks\pdft@k \pdft@k{}
\def\pdft@toc#1{\global\pdft@k=\expandafter{\the\pdft@k#1}}
\newcount\@utno

\newcount\@uts
\newcount\@utss
\newcount\@utt

\def\c@rcdr#1#2,{\edef\car{#1}\edef\cdr{#2}\ifx\cdr\empty\edef\cdr{0}\fi}

\def\pdf@ut{\expandafter\c@rcdr\the\pdft@k,\@uts=\car
 \global\pdft@k=\expandafter{\cdr}\toks0=\expandafter{\cdr}%
 \@utno=0 \@utss=\car \advance\@utss1
 \loop \expandafter\c@rcdr\the\toks0,\@utt=\car\toks0=\expandafter{\cdr}%
  \ifnum\@uts<\@utt \ifnum\@utss=\@utt\advance\@utno1\fi \repeat}

% Sectioning

\outer\def\Part#1 \par{\vfill\break \null\setdest
 \nointerlineskip\vbox to7\baselineskip{\vfil
  \centerline{\fontone #1\label{#1}\toc0{#1}}\vfil}}

\newcount\s@one

\outer\def\Section#1 \par{\vskip0pt plus 4\baselineskip\penalty-250
 \vskip0pt plus-4\baselineskip
 \vskip\baselineskip \advance\s@one1 \setdest
 \noindent{\fonttwo\Stitle\the\s@one\space#1\label{#1}}\toc1{#1}\par
 \nobreak\noindent}

% Indexing

\newcount\d@stno \newcount\p@rno \d@stno=1 \p@rno=1

\def\setdest{\global\advance\d@stno1 \global\p@rno=\d@stno
 \pdfcode \pdfdest num \number\d@stno fitbh\pdfendcode}
\def\destpar{\ifhmode \let\next=\setdest \else \let\next=\relax \fi
 \endgraf \next}

\def\ref#1{??} \def\refsc#1{??} \def\refpg#1{??}
\let\toc=\gobbletwo \let\lbl=\gobbletwo \let\ndx=\gobbletwo

\def\stringall{\stringate\dohigh\stringaccents}

\def\fourdigits#1{\ifnum#1>9999 \message{max is 9999}\else
                  \ifnum#1<0 \message{min is 0}\else
 \ifnum#1<1000 0\fi \ifnum#1<100 0\fi \ifnum#1<10 0\fi\fi\fi \number#1}

\def\save#1{{\def\\{\string\\}\def~{\string~}\stringall
  \edef\next{\write\auxf{#1\string{\fourdigits\p@rno\string}%
    \string{\number\s@one\string}%
    \string{\noexpand\folio\string}}}\next}}

\newwrite\auxf \newwrite\tocf \newwrite\ndxf
\def\files{\begingroup \stringall
 \def~{\string~}\def\\{\string\\}% \catcode`\@=11
 \def\ftoc##1##2##3##4##5{\pdft@toc{##1}%
  \immediate\write\tocf{\string\tocline
   \string{##1\string}\string{##2\string}%
   \string{##3\string}\string{##4\string}\string{##5\string}}}
 \def\fndx##1##2##3##4##5{\immediate\write\ndxf{\string\ndxline
  \string{##1\string}\string{##2\string}%
  \string{##3\string}\string{##4\string}\string{##5\string}}}
 \def\flbl##1##2##3##4##5{\setbox0=\hbox{\csname ^^03##1\endcsname}%
  \ifdim\wd0=0pt
   \expandafter\gdef\csname ^^02##1\endcsname{##2}%
   \expandafter\gdef\csname ^^03##1\endcsname{##3}%
   \expandafter\gdef\csname ^^04##1\endcsname{##4}%
   \expandafter\gdef\csname ^^05##1\endcsname{##5}%
  \else\errmessage{label ##1 redefined}\fi}
 \immediate\openout\tocf=auxiliar.toc
 \immediate\openout\ndxf=auxiliar.ind
 \input auxiliar.aux
 \immediate\closeout\tocf \immediate\closeout\ndxf
 \endgroup
 \openout\auxf=auxiliar.aux
 \def\ref##1{{\stringall
  \expandafter\global\expandafter\let\expandafter\next
   \csname ^^02##1\endcsname
  \expandafter\global\expandafter\let\expandafter\n@xt
   \csname ^^03##1\endcsname}\setbox0=\hbox{\next}%
  \ifdim\wd0=0pt \errmessage{label ##1 undefined}\else % ERROR \string please
   \pdfgoto{\next}{\n@xt}\fi}
 \def\refsc##1{{\stringall
  \setbox0=\hbox{\tenrm\csname ^^04##1\endcsname}%
  \ifdim\wd0=0pt \errmessage{label ##1 undefined}\else
  \pdfgoto{\box0}{\csname ^^03##1\endcsname}\fi}}
 \def\refpg##1{{\stringall
  \setbox0=\hbox{\tenrmj\csname ^^05##1\endcsname}%
  \ifdim\wd0=0pt \errmessage{label ##1 undefined}\else
  \pdfgoto{\box0}{\csname ^^03##1\endcsname}\fi}}
 \def\toc##1##2{\save{\string\ftoc\string{##1\string}%
   \string{##2\string}}}%   level, text
 \def\lbl##1##2{\save{\string\flbl\string{##1\string}%
   \string{##2\string}}}%  label, count
 \def\ndx##1##2{\save{\string\fndx\string{##1\string}%
   \string{##2\string}}}% text, type
 \pdft@toc0 }

\def\label#1{\lbl{#1}{\number\s@one}}
\def\index#1{\ndx{#1}{1}}

\files

% end index

% DIAGRAMS

\def\onitself#1{\leavevmode\vbox{
 \baselineskip=0pt \lineskip=0.25ex \everycr={}\tabskip=0pt
 \halign{\hfil##\hfil\cr\msbm\char"78\cr#1\crcr}}}
\def\column#1{\leavevmode\vtop{
 \baselineskip=0pt \lineskip=0.25ex \everycr={}\tabskip=0pt
 \halign{\hfil##\hfil\cr#1\crcr}}}
\def\base#1{\leavevmode\vbox{
 \baselineskip=0pt \lineskip=0.25ex \everycr={}\tabskip=0pt
 \halign{\hfil##\hfil\cr#1\crcr}}}

\def\subjectdia{\leavevmode
 \vbox{\everycr={}\tabskip=0pt \lineskip=0pt
  \halign{\hfil##\hfil\cr
   \hidewidth\strut Idea\hidewidth\cr
   \hfil\hfil\vbox{\hbox{\big\downarrow}\nointerlineskip\null}\hfil
   \hbox{\msbm\char"78}\hfil
   \vbox{\hbox{\big\uparrow}\nointerlineskip\null}\hfil\hfil\cr
   \noalign{\kern-3pt}
   \column{
    \strut Object\cr
    $\uparrow$\cr
    \hidewidth\strut Sentiment\hidewidth\cr
    $\uparrow$\cr
    \strut Libido\cr}\cr}}}

% Bibliography

\def\cite#1{\ndx{#1}0\footnote{\ref{#1}.}}
\def\citation#1{\footnote{{\def\cite##1{\ndx{##1}0\ref{##1}}#1}}}

\newtoks\TITLE \newtoks\SUBTITLE \newtoks\AUTHOR \newtoks\YEAR
\newtoks\PUBLISHER \newtoks\ADDRESS \newtoks\YEAREDITION\newtoks\ISBN
\newtoks\EDITOR \newtoks\BOOKTITLE \newtoks\EDITION \newtoks\NOTE
\newtoks\JOURNAL \newtoks\VOLUME \newtoks\MONTH \newtoks\PAGES
\def\cleantoks{%
 \TITLE={}\SUBTITLE={}\AUTHOR={}\YEAR={}%
 \PUBLISHER={}\ADDRESS={}\YEAREDITION={}\ISBN={}%
 \EDITOR={}\BOOKTITLE={}\EDITION={}\NOTE={}%
 \JOURNAL={}\VOLUME={}\MONTH={}\PAGES{}}

% METATeX

\def\MTendmark{:::}

\MTcode
pickup pencircle scaled 0.3pt; thin_pen:=savepen;
pickup pencircle scaled 0.6pt; med_pen:=savepen;
pickup pencircle scaled 1.2pt; thick_pen:=savepen;

arrow_head_length := 6pt; arrow_head_width := 2.4pt;
point_diameter := 2.4pt; aperture := 3pt; jot := 2pt;
u := 1pt; v := 1pt;

def rectangle(suffix s)(expr width,height) =
 x.s.l = x.s - width/2;
 x.s.r = x.s + width/2;
 y.s.t = y.s + height/2;
 y.s.b = y.s - height/2;
 draw (x.s.l,y.s.b) -- (x.s.r,y.s.b) --
  (x.s.r,y.s.t) -- (x.s.l,y.s.t) -- cycle;
enddef;

def square(suffix s)(expr side) =
 x.s.l = x.s - side/2;
 x.s.r = x.s + side/2;
 y.s.t = y.s + side/2;
 y.s.b = y.s - side/2;
 draw (x.s.l,y.s.b) -- (x.s.r,y.s.b) --
  (x.s.r,y.s.t) -- (x.s.l,y.s.t) -- cycle;
enddef;

def circle(suffix s)(expr diameter) =
 x.s.l = x.s - diameter/2;
 x.s.r = x.s + diameter/2;
 y.s.t = y.s + diameter/2;
 y.s.b = y.s - diameter/2;
 draw (fullcircle scaled diameter shifted z.s);
enddef;

% point_diameter;

def point(suffix s) =
 fill fullcircle scaled point_diameter shifted z.s;
enddef;

% arrow_head_length, arrow_head_width;

def arrowhead(suffix orig,dest) =
 z.dest.head  = (arrow_head_length/length(z.dest-z.orig))[z.dest,z.orig];
 z.dest.right = z.dest.head + arrow_head_width / 2 *
                dir(angle(z.dest-z.orig)+90);
 z.dest.left  = z.dest.head + arrow_head_width / 2 *
                dir(angle(z.dest-z.orig)-90);
 fill z.dest -- z.dest.right -- z.dest.left -- cycle;
enddef;

def arrow(suffix orig,dest) =
 arrowhead(orig,dest); draw z.orig .. z.dest.head;
enddef;

% aperture;

def soft(suffix orig,med,dest) =
 z.med.o = (aperture/length(z.med-z.orig))[z.med,z.orig];
 z.med.d = (aperture/length(z.med-z.dest))[z.med,z.dest];
 draw z.orig --- z.med.o ... z.med.d --- z.dest;
enddef;

def softt(suffix orig,medone,medtwo,dest) =
 z.medone.o = (aperture/length(z.medone-z.orig))[z.medone,z.orig];
 z.medone.d = (aperture/length(z.medone-z.medtwo))[z.medone,z.medtwo];
 z.medtwo.o = (aperture/length(z.medtwo-z.medone))[z.medtwo,z.medone];
 z.medtwo.d = (aperture/length(z.medtwo-z.dest))[z.medtwo,z.dest];
 draw z.orig --- z.medone.o ... z.medone.d
  --- z.medtwo.o ... z.medtwo.d --- z.dest;
enddef;

def fork(suffix orig,med,dest) =
 arrowhead(med,dest); point(orig); soft(orig,med,dest.head);
enddef;

def arroww(suffix orig,med,dest) =
 arrowhead(med,dest); soft(orig,med,dest.head);
enddef;

def arrowww(suffix orig,medone,medtwo,dest) =
 arrowhead(medtwo,dest); softt(orig,medone,medtwo,dest.head);
enddef;

def back(suffix s,o,d)(expr upper,size,margin) =
 pickup thick_pen;
 y.s.del.t = upper; y.s.del.t - y.s.del.b = size;
 y.s.del.l = y.s.del.r = 1/2[y.s.del.b,y.s.del.t];
 x.s.del.r + x.s.del.l = x.o + x.d;
 x.s.del.r - x.s.del.l = size;
 x.s.del.t = x.s.del.b = x.s.del.r;
 draw z.s.del.l -- z.s.del.b -- z.s.del.t -- cycle;
 pickup med_pen;
 z.s.tr = (x.o,y.s.del.r);
 z.s.r =  (x.o+margin,1/2[y.o,y.s.del.r]);
 z.s.tl = (x.d,y.s.del.l);
 z.s.l =  (x.d-margin,1/2[y.d,y.s.del.l]);
 draw z.o {right} .. z.s.r {up} .. z.s.tr {left};
 arrow(s.tr,s.del.r);
 draw z.s.del.l --- z.s.tl {left} .. z.s.l {down} .. z.d {right};
enddef;

def feedback(suffix s,orig,dest)(expr upper,size,margin) =
 z.s.br = z.orig + (margin,0); z.s.bl = z.dest - (margin,0);
 back(s,s.br,s.bl,upper,size,margin);
 draw z.orig .. z.s.br; arrow(s.bl,dest);
enddef;

def arrowlessfeedback(suffix s,orig,dest)(expr upper,size,margin) =
 z.s.br = z.orig + (margin,0); z.s.bl = z.dest - (margin,0);
 back(s,s.br,s.bl,upper,size,margin);
 draw z.orig .. z.s.br; draw z.s.bl .. z.dest;
enddef;

def forkback(suffix s,orig,dest)(expr upper,size,margin) =
 z.s.bl = z.dest - (margin,0); point(orig);
 back(s,orig,s.bl,upper,size,margin); arrow(s.bl,dest);
enddef;

def shortforkback(suffix s,orig,dest)(expr upper,size,margin) =
 z.s.bl = z.dest - (1/2margin,0); point(orig);
 back(s,orig,s.bl,upper,size,margin); arrow(s.bl,dest);
enddef;

:::
% end METATeX

% Maths

\def\Metric{\ifmmode {\cal M}^{\circ} \else ${\cal M}^{\circ}$\fi}
\def\no#1{{\bf #1}}

\def\inmmode$#1${\ifmmode #1\else $#1$\fi}

\def\aut#1{\inmmode$\mathop{\cal #1}\nolimits$}
\def\syn#1{\inmmode$\mathop{\frak #1}\nolimits$}

%%%

\def\universe{\vindex{universe}~$\aut U$}
\def\universes{\vindex{universe}~$\aut U$\kern-0.5pt{\sl's}}
\def\mechanism{\vindex{mechanism}~$\aut A_0$}
\def\mechanisms{\vindex{mechanism}~$\aut A_0$\kern-1.5pt{\rm's}}
\def\adaptor{\vindex{adaptor}~$\!\aut A_1$}
\def\adaptors{\vindex{adaptor}~$\!\aut A_1$\kern-2pt{\rm's}}
 \def\governor{\vindex{governor}~$\aut G$}
 \def\governors{\vindex{governor}~$\aut G$\kern-0.8pt{\sl's}}
 \def\body{\vindex{body}~$\!\aut B$}
 \def\bodys{\vindex{body}~$\!\aut B$\kern-0.8pt{\sl's}}
\def\learner{\vindex{learner}~$\!\aut A_2$}
\def\learners{\vindex{learner}~$\!\aut A_2$\kern-1.5pt{\rm's}}
 \def\modeler{\vindex{modeler}~$\!\aut M$}
 \def\modelers{\modeler\kern-0.8pt{\sl's}}
 \def\simulator{\vindex{simulator}~$\!\aut S$}
 \def\simulators{\simulator\kern-0.8pt{\sl's}}
 \def\reality{\vindex{reality}~$\aut R$}
\def\knower{\vindex{knower}~$\!\aut A_3$}
\def\knowers{\vindex{knower}~$\!\aut A_3$\kern-1.2pt{\rm's}}
 \def\intelligence{\vindex{intelligence}~$\syn A$}
 \def\mind{\vindex{mind}~$\syn M$} % $\Re$
 \def\minds{\mind{\rm's}}
\def\subject{\vindex{subject}~$\aut A_4$}
\def\subjects{\vindex{subject}~$\aut A_4$\kern-1.5pt{\rm's}}
 \def\inquirer{\vindex{inquirer}~$\syn I$}
 \def\inquirers{\inquirer\kern0pt{\rm's}}
 \def\reason{\vindex{reason}~$\syn R$}
 \def\self{\vindex{self}~$\syn X$}

\def\Corporal#1$#2#3${\vindex{#1}~${\aut #2}#3$}
\def\corporal#1$#2#3${#1~${\aut #2}#3$}
\def\Mental#1$#2#3${\vindex{#1}~${\syn #2}#3$}
\def\mental#1$#2#3${#1~${\syn #2}#3$}

\def\tURING{\vperson} % just a trick to alfabetize OK
\def\TM{\tURING[Turing] machine%
 \hindex{\string\string\string\tURING[Turing] machine}~$\mathop{\frak T}$}
\def\TMes{\tURING[Turing] machines%
 \hindex{\string\string\string\tURING[Turing] machine}~$\mathop{\frak T}$}
\def\UTM{universal \tURING[Turing] machine%
 \hindex{universal \string\string\string\tURING[Turing] machine}~$\mathop{\frak U}$}
\def\UTMes{universal \tURING[Turing] machines%
 \hindex{universal \string\string\string\tURING[Turing] machine}~$\mathop{\frak U}$}

\def\processor$#1${processor~${\cal P}_{\frak #1}$}
\def\Processor$#1${\vindex{processor}~${\cal P}_{\frak #1}$}
\def\UP{\vindex{universal processor}~${\cal P}_{\frak U}$} % cal

\def\true{\hbox{\sc true}}
\def\false{\hbox{\sc false}}

\def\llave#1{\inmmode$\left\lbrace\vcenter{
 \halign{&\hbox{\rm\strut##}\hfil\cr#1\crcr}}\right.$}
\def\lopen#1#2{\inmmode$\left#1\vcenter{
 \halign{&\hbox{\rm\strut##}\hfil\cr#2\crcr}}\right.$}

\def\etapa{\inmmode$\mapstochar\Rightarrow$} % for the reader's guide
\def\etapa{\inmmode$\mathrel{\vrule height 3.5pt depth-1.5pt}\kern-1pt
  =\kern-1pt\mathrel{\triangleright}$}

% KEYS

\def\meaning#1{{\it#1\aftergroup\itcor}}
\def\latin#1{{\it#1}\itcor}
\def\booktitle#1{{\sl#1}\itcor}

% Index: 0, for cite, 1 for definition, 2 for person, 3 for index.

% VISIBLE       HIDDEN
% ^{index}      _{index}
% ^[Person]     _[Person]
% ^(cite1999)   _(cite1999)
% ^|definition| _|definition|
% ^<booktitle>  _<label>
% ^>ref>        _>ref>   ( V = \S2.1, p�g.~123     H = \S2.1 )

\def\specialhat{\ifmmode\def\next{^}\else\let\next=\beginvref\fi\next}
\def\specialund{\ifmmode\def\next{_}\else\let\next=\beginhref\fi\next}
\catcode`\^=13 \let^\specialhat \catcode`\_=13 \let_\specialund

\def\beginvref{\futurelet\next\beginvrefx}
\def\beginhref{\futurelet\next\beginhrefx}

\def\beginvrefx{\begingroup
 \ifx\next[\aftergroup\vperson \else
 \ifx\next(\aftergroup\vcite \else
 \ifx\next|\aftergroup\vdefinition \else
 \ifx\next<\aftergroup\vlabel \else
 \ifx\next>\aftergroup\vref \else
 \aftergroup\vindex \fi\fi\fi\fi\fi \endgroup}
\def\beginhrefx{\begingroup
 \ifx\next[\aftergroup\hperson \else
 \ifx\next(\aftergroup\hcite \else
 \ifx\next|\aftergroup\hdefinition \else
 \ifx\next<\aftergroup\hlabel \else
 \ifx\next>\aftergroup\href \else
 \aftergroup\hindex \fi\fi\fi\fi\fi \endgroup}

\def\allowhyphens{\nobreak\hskip\z@skip}

\def\vperson[#1]{\leavevmode\ndx{#1}{2}\allowhyphens{\sc#1}}
\def\hperson[#1]{\allowhyphens{\sc#1}}
\def\vcite(#1){\ndx{#1}0\footnote{\ref{#1}.}}
\def\hcite(#1){\ndx{#1}0\ref{#1}}
\def\vdefinition|#1|{\leavevmode\Presignal\ndx{#1}{1}%
 \Green\allowhyphens#1\allowhyphens\Black\Signal}
\def\hdefinition|#1|{\allowhyphens\ndx{#1}{1}}%\tomargin{\it#1}}
\def\vlabel<#1>{{\sl#1}\itcor} % used for book titles only
%\def\hlabel<#1>{\label{#1}}
\def\vref>#1>{\S\refsc{#1}, page~\refpg{#1}}
\def\href>#1>{\S\refsc{#1}}
\def\vindex#1{\leavevmode\presignal\ndx{#1}{3}%
 \Blue\allowhyphens#1\allowhyphens\Black\signal}
\def\hindex#1{\allowhyphens\ndx{#1}{3}\hsignal}

%\def\signal{\hbox to0pt{\Blue\hss\vbox to0pt{
% \hbox{$\longleftarrow$}\vss}\Black}}
\def\Signal{\hbox to0pt{\hss\vrule width10pt height-2pt depth2.4pt
 \vrule width0.4pt height4pt depth2pt}}
\def\Presignal{\hbox to0pt{\vrule width0.4pt height4pt depth2pt
 \vrule width10pt height-2pt depth2.4pt\hss}}
\def\signal{\hbox to0pt{\hss\vrule width10pt height-2pt depth2.4pt}}
\def\hsignal{\hbox to0pt{\hss\vrule width10pt height-2pt depth2.4pt}}
\def\presignal{\hbox to0pt{\vrule width10pt height-2pt depth2.4pt\hss}}


%% OPTION comment out to get signals
\let\Signal=\null \let\Presignal=\null
\let\signal=\null \let\presignal=\null
\let\hsignal=\null

\def\EPA#1{{\sc epa}\ndx{Casares1999}0~\S{\tenrm#1}} %%%%%%%%%% Provisional

% TEXT

\catcode`\@=12

\input OF0.tex


\setdest \let\par=\destpar

% Contents

\pdfcode
\pdfdest name {contents} fitbh
\pdfoutline goto name {contents} count 0 {Contents}
\pdfendcode

\vbox to22pt{\centerline{\fontone Contents}\vss}
\vskip 2pc plus1pc minus6pt

\begingroup
\catcode`\@=11

\def\tocline#1{\ifcase #1\let\next=\toclinezero \or
 \let\next=\toclineone \else \let\next=\toclineindex \fi \next}

\def\toclinezero#1#2#3#4{\bigbreak
 {\pdfcode \pdf@ut \stringactives \pdfoutline goto num #2
  count -\number\@utno {#1}\pdfendcode}%
 \line{\bf\pdfgoto{#1}{#2}\hfil}\ignorespaces}
\def\toclineone#1#2#3#4{\par
 {\pdfcode \pdf@ut \stringactives \pdfoutline goto num #2
  count -\number\@utno {\noexpand�#3 #1}\pdfendcode}%
 \contentsline{#1}{#2}{$\S#3$\quad}{#4}\ignorespaces}

\def\Referencias{Referencias}
\def\toclineindex#1#2#3#4{\par\def\1{#1}\ifx\1\Referencias\bigbreak\fi
 {\pdfcode \pdf@ut \stringactives \pdfoutline goto num #2
  count -\number\@utno {#1}\pdfendcode}%
 \contentsline{{\bf #1}}{#2}{}{#4}\ignorespaces}

\newdimen\tridig \setbox0=\hbox{$\S123$\quad}\tridig=\wd0

\def\contentsline#1#2#3#4{\setbox0\hbox to\tridig{\hfil#3}\setbox2\hbox{#1}%
 \dimen0=\hsize \advance\dimen0 by -\wd0
 \multiply\dimen0 by 8 \divide\dimen0 by 10
 \ifdim\dimen0>\wd2 \line{\box0 #1\tocleaders \pdfgoto{#4}{#2}}\else
  \line{\box0 \vtop{\hsize=\dimen0 \raggedright \normalbaselines
   \let\\=\ \noindent #1\strut}\tocleaders \pdfgoto{#4}{#2}}\fi}
\def\tocleaders{\leaders\hbox to\baselineskip{\hss\bf.\hss}\hfil}

\raggedbottom \openup3pt
\input auxiliar.toc

\endgroup

\vfill
\break

\null\vfill\break % Blank page

\input OF1.tex
\input OF2.tex
\input OF3.tex
\input OF4.tex
\input OF5.tex
\input OF6.tex

\let\par=\endgraf

\catcode`\@=11

\vfill\break %%%%%%%%%%%%%%%%%%%%%%%%%%%%%%%%%%%

\setdest
\vbox to22pt{\centerline{\fontone References}\toc3{References}\vss}
\vskip 1pc plus1pc minus6pt

\noindent The editions of the works consulted are given.
The numbers that appear at the end of each entry, after the
triangle~{\teni.}, indicate the pages on which the work is cited. Thus,
{\teni.}\refpg{The Chomsky Hierarchy} means that the reference is on
page~\refpg{The Chomsky Hierarchy}.

\newwrite\bibnotes \immediate\openout\bibnotes=auxiliar.bdb
\iverb\bibnotes"% AUXILIAR.BDB"
\iverb\bibnotes"\record{Lettvin1959}{INCOLLECTION}{"
\iverb\bibnotes" \AUTHOR{Lettvin, J.\kern-1pt Y\kern-1pt.; "
\iverb\bibnotes" Maturana, H.R.; McCulloch, \kern-1ptW\kern-2pt.S.; "
\iverb\bibnotes" Pitts, W\kern-2pt.H.}}"
\iverb\bibnotes"\record{Rumelhart1986}{BOOK}{"
\iverb\bibnotes" \AUTHOR{Rumelhart, D.E.; McClelland, J.L.; \& the PDP Research Group}}"
\iverb\bibnotes"\endinput"
\immediate\closeout\bibnotes

\newcount\lastpgno \newcount\thispgno
\newif\ifrange

\def\gobblethree#1#2#3{\ignorespaces}

\def\ndxline#1#2{\ifnum#2=0 \let\oldk@y=\newk@y \def\newk@y{#1}%
 \let\next=\bibline \else \let\next\gobblethree \fi \next}

%\def\bibbreak{\ifdim\lastskip<\smallskipamount \removelastskip
% \penalty-200 \vskip3pt plus 2pt \relax\fi} %plus 1.5pt minus 1pt \fi}
%\let\bibbreak=\filbreak
\def\bibbreak{\par\vskip-2pt plus 1fil \penalty-200 \vskip2pt plus -1fil\relax }

\def\bibline#1#2#3{\ifx\newk@y\oldk@y \thispgno=#3
 \ifnum\lastpgno=\thispgno \else \advance\lastpgno1
  \ifnum\lastpgno=\thispgno \rangetrue \else
   \ifrange \advance\lastpgno-1 --\number\lastpgno\fi
   \lastpgno=#3\rangefalse
   , \pdfgoto{#3}{#1}%
  \fi\fi
 \else % \newk@y<>\oldk@y
 \ifrange --\number\lastpgno\rangefalse \fi
 \bibbreak
 \setdest \cleantoks \lastpgno=0
 \def\record##1##2##3{\def\secondk@y{##1}\ifx\newk@y\secondk@y##3\fi}
 \input OF.bdb
 \input auxiliar.bdb
 \hang\noindent\rm{\sc\the\AUTHOR}:%
 {\def\em{\string\em\space}\def~{\string~}\def\it{\string\it\space}%
  \def\sc{\string\sc\space}%
  \lbl{\newk@y}{\string\rm{\string\sc\space
  \expandafter\expandafter\the\AUTHOR}\space
  (\expandafter\expandafter\the\YEAR):\space
  {\string\sl\space\expandafter\expandafter\the\TITLE}}}%
 {\sl\null{ }\the\TITLE
 \setbox0=\hbox{\the\SUBTITLE}\ifdim\wd0=0pt\else\the\SUBTITLE\fi}%
 \setbox0=\hbox{\the\YEAREDITION}\ifdim\wd0=0pt
  \YEAREDITION={\the\YEAR}\else\ (\the\YEAR)\fi
 \setbox0=\hbox{\the\EDITOR}\ifdim\wd0=0pt\else.
   Edited by \the\EDITOR\fi
 \setbox0=\hbox{\the\BOOKTITLE}\ifdim\wd0=0pt\else
   \ in ``\the\BOOKTITLE''\fi
 \setbox0=\hbox{\the\EDITION}\ifdim\wd0=0pt\else. \the\EDITION\fi
 \setbox0=\hbox{\the\PUBLISHER}\ifdim\wd0=0pt\else, \the\PUBLISHER\fi
 \setbox0=\hbox{\the\ADDRESS}\ifdim\wd0=0pt\else, \the\ADDRESS\fi
 \setbox0=\hbox{\the\JOURNAL}\ifdim\wd0=0pt\else, \the\JOURNAL\fi
 \setbox0=\hbox{\the\VOLUME}\ifdim\wd0=0pt\else, vol.~\the\VOLUME\fi
 \setbox0=\hbox{\the\MONTH}\ifdim\wd0=0pt
  \setbox0=\hbox{\the\YEAREDITION}\ifdim\wd0=0pt\else, \the\YEAREDITION\fi
  \else, \the\MONTH\ de \the\YEAREDITION\fi
 \setbox0=\hbox{\the\PAGES}\ifdim\wd0=0pt\else, pp.~\the\PAGES\fi
 \setbox0=\hbox{\the\ISBN}\ifdim\wd0=0pt\else, {\sc isbn} {\the\ISBN}\fi.
 \setbox0=\hbox{\the\NOTE}\ifdim\wd0=0pt\else \the\NOTE. \fi
 {\teni.}\pdfgoto{#3}{#1}\fi
 \ignorespaces}

\bigskip

\input auxiliar.abc


\vfill\break %%%%%%%%%%%%%%%%%%%%%%%%%%%%%%%%%%%

% Output for INDEX adapted from [417]

\newdimen\fullhsize \fullhsize=\hsize
\newdimen\fullvsize \fullvsize=\vsize
\def\fullline{\hbox to\fullhsize}

\newdimen\gutter \gutter=1pc
\newbox\partialpage
\def\begindoublecolumns{\begingroup
 \output={\global\setbox\partialpage=\vbox{\unvbox255\bigskip}}\eject
 \output={\doublecolumnout}%
 \hsize=\fullhsize \advance\hsize by-\gutter \divide\hsize by2
 \vsize=\fullvsize \multiply\vsize by2 \advance\vsize by2pc}
\def\enddoublecolumns{\output={\balancecolumns}\eject
 \endgroup \pagegoal=\vsize}

\def\doublecolumnout{\splittopskip=\topskip \splitmaxdepth=\maxdepth
 \dimen0=\fullvsize \advance\dimen0 by-\ht\partialpage
 \setbox0=\vsplit255 to\dimen0 \setbox2=\vsplit255 to\dimen0
 \shipout\vbox{\vbox to 0pt{\vskip-22.5pt
  \fullline{\vbox to8.5pt{}\the\headline}\vss}\nointerlineskip
  \vbox to\fullvsize{\boxmaxdepth=\maxdepth \pagesofar}
  \baselineskip=24pt \fullline{\the\footline}}\advancepageno
 \unvbox255 \penalty\outputpenalty}
\def\balancecolumns{\setbox0=\vbox{\unvbox255}\dimen0=\ht0
 \advance\dimen0 by\topskip \advance\dimen0 by-\baselineskip
 \divide\dimen0 by2 \splittopskip=\topskip
 {\vbadness=10000 \loop \global\setbox3=\copy0
  \global\setbox1=\vsplit3 to\dimen0
  \ifdim\ht3>\dimen0 \global\advance\dimen0 by1pt \repeat}
 \setbox0=\vbox to\dimen0{\unvbox1}%
 \setbox2=\vbox to\dimen0{\dimen2=\dp3 \unvbox3 \kern-\dimen2 \vfil}%
 \pagesofar}
\def\pagesofar{\unvbox\partialpage
 \wd0=\hsize \wd2=\hsize \fullline{\box0\hfil\box2}}

%%%

\setdest
\vbox to22pt{\centerline{\fontone Index}\toc3{Index}\vss}
\vskip 1pc plus1pc minus6pt
%\vskip 5.87955pt % just to make exact room for columns

\noindent The concepts in boldface are the most important concepts
of the theory of subjectivity; the numbers in boldface indicate the
pages on which they are defined. The other concepts are less important,
and the other numbers indicate on which pages the different concepts and
authors appear, independently of their importance.

\lastpgno=0 \thispgno=0 \rangefalse

\def\ndxline#1{\let\oldkey=\newkey \def\newkey{#1}\defline}
\def\defline#1#2#3#4{\ifnum#1=0 \else % not biblio
 \ifx\newkey\oldkey \thispgno=#4
  \ifnum\lastpgno=\thispgno \else \advance\lastpgno1
   \ifnum\lastpgno=\thispgno \rangetrue \else
    \ifrange \advance\lastpgno-1 --\number\lastpgno\fi
    \lastpgno=#4\rangefalse
    , \pdfgoto{\ifnum#1=1 {\boldstyle#4}\else #4\fi}{#2}%
   \fi\fi
 \else % \newkey<>\oldkey
  \ifrange --\number\lastpgno\fi
  \bibbreak\hang\noindent \lastpgno=#4\rangefalse
  \ifcase#1 {\tt\newkey}\or{\bf\newkey}\or{\sc\newkey}\or{\rm\newkey}\else
   {\it\Red \newkey\Black}\fi
 ,\space\space
 \pdfgoto{\ifnum#1=1 {\boldstyle#4}\else #4\fi}{#2}\fi\fi\ignorespaces}

\begindoublecolumns
\parindent=12pt \rightskip=0pt plus 4pc \hyphenpenalty=250
\input auxiliar.abc
 \ifrange --\number\lastpgno \fi
\enddoublecolumns

\vfill\break %%%%%%%%%%%%%%%%%%%%%%%%%%%%%%%%%%%

\let\vfill=\relax
\bye
