% of.tex (RMCG20000104)

%\mag1200 % OPTIONAL MAGNIFICATION

\input explain
\input fonts
\input index
\input metatex

%\input crops % OPTIONAL CROPS (Beware the next line!)
\newdimen\hpage \newdimen\vpage % already defined in crops.tex

\newif\ifportada \portadatrue

\def\[{\string\[} % to allow ^álgebra de \[Boole]^

% Fonts

% Fonts for titles

\font\fonttwosym=cmbsy10 scaled\magstep1
\def\Stitle{{\fonttwosym \char"78}}

\lmfonts\lmtitles\ansifont
 \font\tenrm=rm-lmr10 \textfont0=\tenrm
\font\frakx=eufm10
\font\frakvii=eufm7
\newfam\frakfam \textfont\frakfam=\frakx \scriptfont\frakfam=\frakvii
\def\frak{\fam\frakfam\frakx}

\font\msbm=msbm10
\font\logo=logo10

%\pdfcode
% \pdfmapline{+ccicons CCIcons <ccicons.pfb}
%\pdfendcode
\font\ccicons=ccicons
\def\ccbysa{{\ccicons \char0\kern1pt\char1\kern1pt\char2}}

\font\ptitlefont=texnansi-lmssbx10 at 90pt
\font\psubtitlefont=texnansi-lmss10 at 40pt
\font\pauthorfont=texnansi-lmss10 at 50pt
\font\pcitefont=texnansi-lmss10 at 21pt
\font\ptextfont=texnansi-lmss10 at 14pt

\def\pdfBlack{\pdfliteral{0 0 0 1 k 0 0 0 1 K}}
\def\pdfWhite{\pdfliteral{0 0 0 0 k 0 0 0 0 K}}
\def\pdfRed{\pdfliteral{0 1 1 0 k 0 1 1 0 K}}
\def\pdfBlue{\pdfliteral{1 1 0 0 k 1 1 0 0 K}}
\def\pdfGreen{\pdfliteral{1 0 1 0 k 1 0 1 0 K}}

\catcode`\@=11

% Layout

\let\docinfo\relax \let\infodoc\relax

\headline={\hfil}
\footline={\tenrm\ifodd\pageno \docinfo\hfil\folio
            \else \folio\hfil\infodoc \fi\strut}

\hpage=6in \hsize=10.5cm
 \hoffset\hpage \advance\hoffset-\hsize \divide\hoffset2
 \advance\hoffset-1in
\vpage=9in \vsize=39\baselineskip \advance\vsize\topskip % 39 + 1 lines
 \voffset\vpage \advance\voffset-\vsize \advance\voffset-2\baselineskip
 \divide\voffset2 \advance\voffset-1in

\pdfcode \pdfpageheight=\vpage \pdfpagewidth=\hpage \pdfendcode

%\def\titulo{Ramón Casares\cr Sobre la libertad}
%\setcrops % OPTIONAL (Beware the top of this file!)

\parindent=20pt

% Grid

%\tracingpages=1 % to check that the lines are on the grid

\abovedisplayskip=12pt
\belowdisplayskip=12pt
\abovedisplayshortskip=0pt
\belowdisplayshortskip=12pt

\newbox\displaybox
\newdimen\displayht
\newdimen\displaydp

% if \baselineskip is 12pt,
% the strut height is 8.5pt     17/24 = 0.7083..3
% and its depth is 3.5pt.        7/24 = 0.2916..6

% It doesn't work with alignment displays (The {\TeX}book, pg 190)

\def\griddisplay#1$${%
 \setbox\displaybox=\hbox{$\displaystyle{#1}$}%
 \displayht=\ht\displaybox \displaydp=\dp\displaybox
 \advance\displayht by \lineskiplimit
 \advance\displayht by 0.75\baselineskip
 \divide\displayht by \baselineskip
 \multiply\displayht by \baselineskip
 \advance\displaydp by \lineskiplimit
 \advance\displaydp by 0.25\baselineskip
 \divide\displaydp by \baselineskip
 \multiply\displaydp by \baselineskip
 \ht\displaybox=0pt \dp\displaybox=0pt
 \baselineskip=\displayht
 \box\displaybox
 \ifdim\displaydp=0pt\else \vadjust{\kern\displaydp}\fi$$}
\everydisplay{\griddisplay}

\def\nodepth#1{\setbox0=\hbox{#1}\dp0=0pt\box0}

% Covers background color
\def\background{\pdfcode
  \pdfliteral{q}% save current graphic state in the stack
  \pdfGreen
  \pdfliteral{-67 -574 m}% 72 * 9 = 648, 72 * 8 = 576
  \pdfliteral{365 -574 l}% 72 * 6 = 432, 72 * 5 = 360
  \pdfliteral{365 74 l}%
  \pdfliteral{-67 74 l}%
  \pdfliteral{-67 -574 l}%
  \pdfliteral{b Q}% close, stroke, fill, and restore graphic state
 \pdfendcode}

% Every point is a period
\count255=`A \loop\sfcode\count255=1000
 \ifnum\count255<`Z\advance\count255 1 \repeat
\def~{\nobreak\ \ignorespaces}

% Emphasis with automatic italic correction (\/).
% Use: {\em italic, but {\em roman}, text}.
\def\em{\ifdim \fontdimen1\font>0pt \rm
 \else \it \expandafter\aftergroup \fi \itcor}
\def\itcor{\ifhmode \expandafter\itpuncl@ok \fi}
\def\itpuncl@ok{\begingroup\futurelet\ITCt@mpa\itcort@st}
\def\itcort@st{\def\ITCt@mpb{\ITCt@mpa}%
 \ifcat\noexpand\ITCt@mpa,\setbox0=\hbox{\ITCt@mpb}%
  \ifdim\ht0<0.3ex \let\itc@rdo=\endgroup \fi\fi \itc@rdo}
\def\itc@rdo{\skip0=\lastskip \ifdim\skip0=0pt \/\else
 \unskip \/\hskip\skip0 \fi \endgroup}


% Other

\def\beginpoints{\par\begingroup\everypar{}\parindent=20pt}
\def\point{\par \hang\indent\llap{$\circ$\enspace}}
\def\siglo#1 {\par \noindent\hang {\sc #1} $\cdot$ \ignorespaces}
\def\endpoints{\par\endgroup}

\def\goodbreak{\vskip0pt plus 4\baselineskip\penalty-250
 \vskip0pt plus-4\baselineskip}
\def\breakif#1{\vskip0pt plus #1\baselineskip \penalty-250
 \vskip0pt plus-#1\baselineskip}


% INDEX

% PDFTeX

%%% OPTION
\nocolor % default value
%\colors

% Numbered footnotes
\newcount\footnoteno
\saveplain\footnote
\def\footnote{\global\advance\footnoteno1
 \plain\footnote{$^{\number\footnoteno}$}}
\def\footnoteoptions{\everypar{}\parindent=20pt}

% Sectioning

\outer\def\Part#1 \par{\vfill\break \null\pdflabel
 \nointerlineskip\vbox to7\baselineskip{\vfil
  \centerline{\fontone #1\label{#1}\toc1{#1}}\vfil}}

\newcount\secno \secno=0
\newcount\parno \parno=0

\def\section{\number\secno}

\def\Section#1 \par{\everypar{}\parno=0  \advance\secno1
 \vskip0pt plus 4\baselineskip\penalty-43
 \vskip0pt plus-4\baselineskip \vskip\baselineskip
 \titleline{\noindent\Stitle\fonttwo\section\space}%
   {\fonttwo#1\pdflabel\toc2{#1}\lbl{#1}{\section}}%
 \everypar{\numberedpars}\nobreak}

\def\titleline#1#2{\setbox0\hbox{#1}\dimen0=\hsize \advance\dimen0 -\wd0
 \line{\box0\hss\vtop{\hsize=\dimen0 \raggedright\let\\=\ \noindent #2}}}

\newdimen\oldparindent \oldparindent=\parindent \parindent=0pt

\def\numberedpars{\global\advance\parno1 \pdflabel
 \noindent\hbox to\oldparindent{{\sevensy\char"7B %" little par sign
  \teni\number\parno}\hfil}\ignorespaces}
%  \teni\number\parno}\hfil$\cdot$\hfil}\ignorespaces}


% Indexing

\def\destpar{\ifhmode \let\next=\pdflabel \else \let\next=\relax \fi
 \endgraf \next}

\def\label#1{\pdflabel\lbl{#1}{\section}}
\def\index#1{\pdflabel\ndx{#1}{1}}

\def\pdfout#1#2#3{\ifnum#1=2 §#3 #2\else #2\fi}

\files

% end index

% DIAGRAMS

\def\onitself#1{\leavevmode\vbox{
 \baselineskip=0pt \lineskip=0.25ex \everycr={}\tabskip=0pt
 \halign{\hfil##\hfil\cr\msbm\char"78\cr#1\crcr}}}
\def\column#1{\leavevmode\vtop{
 \baselineskip=0pt \lineskip=0.25ex \everycr={}\tabskip=0pt
 \halign{\hfil##\hfil\cr#1\crcr}}}
\def\base#1{\leavevmode\vbox{
 \baselineskip=0pt \lineskip=0.25ex \everycr={}\tabskip=0pt
 \halign{\hfil##\hfil\cr#1\crcr}}}

\def\subjectdia{\leavevmode
 \vbox{\everycr={}\tabskip=0pt \lineskip=0pt
  \halign{\hfil##\hfil\cr
   \hidewidth\strut Idea\hidewidth\cr
   \hfil\hfil\vbox{\hbox{\big\downarrow}\nointerlineskip\null}\hfil
   \hbox{\msbm\char"78}\hfil
   \vbox{\hbox{\big\uparrow}\nointerlineskip\null}\hfil\hfil\cr
   \noalign{\kern-3pt}
   \column{
    \strut Objeto\cr
    $\uparrow$\cr
    \hidewidth\strut Sentimiento\hidewidth\cr
    $\uparrow$\cr
    \strut Libido\cr}\cr}}}

% Bibliography

\def\cite#1{\ndx{#1}0\footnote{\ref{#1}.}}
\def\citation#1{\footnote{{\def\cite##1{\ndx{##1}0\ref{##1}}#1}}}

\newtoks\TITLE \newtoks\SUBTITLE \newtoks\AUTHOR \newtoks\YEAR
\newtoks\PUBLISHER \newtoks\ADDRESS \newtoks\YEAREDITION\newtoks\ISBN
\newtoks\EDITOR \newtoks\BOOKTITLE \newtoks\EDITION \newtoks\NOTE
\newtoks\JOURNAL \newtoks\VOLUME \newtoks\MONTH \newtoks\PAGES
\def\cleantoks{%
 \TITLE={}\SUBTITLE={}\AUTHOR={}\YEAR={}%
 \PUBLISHER={}\ADDRESS={}\YEAREDITION={}\ISBN={}%
 \EDITOR={}\BOOKTITLE={}\EDITION={}\NOTE={}%
 \JOURNAL={}\VOLUME={}\MONTH={}\PAGES{}}

% METATeX

\def\MTendmark{:::}

\MTcode
pickup pencircle scaled 0.3pt; thin_pen:=savepen;
pickup pencircle scaled 0.6pt; med_pen:=savepen;
pickup pencircle scaled 1.2pt; thick_pen:=savepen;

arrow_head_length := 6pt; arrow_head_width := 2.4pt;
point_diameter := 2.4pt; aperture := 3pt; jot := 2pt;
u := 1pt; v := 1pt;

def rectangle(suffix s)(expr width,height) =
 x.s.l = x.s - width/2;
 x.s.r = x.s + width/2;
 y.s.t = y.s + height/2;
 y.s.b = y.s - height/2;
 draw (x.s.l,y.s.b) -- (x.s.r,y.s.b) --
  (x.s.r,y.s.t) -- (x.s.l,y.s.t) -- cycle;
enddef;

def square(suffix s)(expr side) =
 x.s.l = x.s - side/2;
 x.s.r = x.s + side/2;
 y.s.t = y.s + side/2;
 y.s.b = y.s - side/2;
 draw (x.s.l,y.s.b) -- (x.s.r,y.s.b) --
  (x.s.r,y.s.t) -- (x.s.l,y.s.t) -- cycle;
enddef;

def circle(suffix s)(expr diameter) =
 x.s.l = x.s - diameter/2;
 x.s.r = x.s + diameter/2;
 y.s.t = y.s + diameter/2;
 y.s.b = y.s - diameter/2;
 draw (fullcircle scaled diameter shifted z.s);
enddef;

% point_diameter;

def point(suffix s) =
 fill fullcircle scaled point_diameter shifted z.s;
enddef;

% arrow_head_length, arrow_head_width;

def arrowhead(suffix orig,dest) =
 z.dest.head  = (arrow_head_length/length(z.dest-z.orig))[z.dest,z.orig];
 z.dest.right = z.dest.head + arrow_head_width / 2 *
                dir(angle(z.dest-z.orig)+90);
 z.dest.left  = z.dest.head + arrow_head_width / 2 *
                dir(angle(z.dest-z.orig)-90);
 fill z.dest -- z.dest.right -- z.dest.left -- cycle;
enddef;

def arrow(suffix orig,dest) =
 arrowhead(orig,dest); draw z.orig .. z.dest.head;
enddef;

% aperture;

def soft(suffix orig,med,dest) =
 z.med.o = (aperture/length(z.med-z.orig))[z.med,z.orig];
 z.med.d = (aperture/length(z.med-z.dest))[z.med,z.dest];
 draw z.orig --- z.med.o ... z.med.d --- z.dest;
enddef;

def softt(suffix orig,medone,medtwo,dest) =
 z.medone.o = (aperture/length(z.medone-z.orig))[z.medone,z.orig];
 z.medone.d = (aperture/length(z.medone-z.medtwo))[z.medone,z.medtwo];
 z.medtwo.o = (aperture/length(z.medtwo-z.medone))[z.medtwo,z.medone];
 z.medtwo.d = (aperture/length(z.medtwo-z.dest))[z.medtwo,z.dest];
 draw z.orig --- z.medone.o ... z.medone.d
  --- z.medtwo.o ... z.medtwo.d --- z.dest;
enddef;

def fork(suffix orig,med,dest) =
 arrowhead(med,dest); point(orig); soft(orig,med,dest.head);
enddef;

def arroww(suffix orig,med,dest) =
 arrowhead(med,dest); soft(orig,med,dest.head);
enddef;

def arrowww(suffix orig,medone,medtwo,dest) =
 arrowhead(medtwo,dest); softt(orig,medone,medtwo,dest.head);
enddef;

def back(suffix s,o,d)(expr upper,size,margin) =
 pickup thick_pen;
 y.s.del.t = upper; y.s.del.t - y.s.del.b = size;
 y.s.del.l = y.s.del.r = 1/2[y.s.del.b,y.s.del.t];
 x.s.del.r + x.s.del.l = x.o + x.d;
 x.s.del.r - x.s.del.l = size;
 x.s.del.t = x.s.del.b = x.s.del.r;
 draw z.s.del.l -- z.s.del.b -- z.s.del.t -- cycle;
 pickup med_pen;
 z.s.tr = (x.o,y.s.del.r);
 z.s.r =  (x.o+margin,1/2[y.o,y.s.del.r]);
 z.s.tl = (x.d,y.s.del.l);
 z.s.l =  (x.d-margin,1/2[y.d,y.s.del.l]);
 draw z.o {right} .. z.s.r {up} .. z.s.tr {left};
 arrow(s.tr,s.del.r);
 draw z.s.del.l --- z.s.tl {left} .. z.s.l {down} .. z.d {right};
enddef;

def feedback(suffix s,orig,dest)(expr upper,size,margin) =
 z.s.br = z.orig + (margin,0); z.s.bl = z.dest - (margin,0);
 back(s,s.br,s.bl,upper,size,margin);
 draw z.orig .. z.s.br; arrow(s.bl,dest);
enddef;

def arrowlessfeedback(suffix s,orig,dest)(expr upper,size,margin) =
 z.s.br = z.orig + (margin,0); z.s.bl = z.dest - (margin,0);
 back(s,s.br,s.bl,upper,size,margin);
 draw z.orig .. z.s.br; draw z.s.bl .. z.dest;
enddef;

def forkback(suffix s,orig,dest)(expr upper,size,margin) =
 z.s.bl = z.dest - (margin,0); point(orig);
 back(s,orig,s.bl,upper,size,margin); arrow(s.bl,dest);
enddef;

def shortforkback(suffix s,orig,dest)(expr upper,size,margin) =
 z.s.bl = z.dest - (1/2margin,0); point(orig);
 back(s,orig,s.bl,upper,size,margin); arrow(s.bl,dest);
enddef;

:::
% end METATeX

% Maths

\def\Metric{\ifmmode {\cal M}^{\circ} \else ${\cal M}^{\circ}$\fi}
\def\no#1{{\bf #1}}

\def\inmmode$#1${\ifmmode #1\else $#1$\fi}

\def\aut#1{\inmmode$\mathop{\cal #1}\nolimits$}
\def\syn#1{\inmmode$\mathop{\frak #1}\nolimits$}

\def\universe{\vindex{universe}~$\aut U$}
\def\universes{\vindex{universe}~$\aut U$\kern-0.5pt{\sl's}}
\def\mechanism{\vindex{mechanism}~$\aut A_0$}
\def\mechanisms{\vindex{mechanism}~$\aut A_0$\kern-1.5pt{\rm's}}
\def\adaptor{\vindex{adaptor}~$\!\aut A_1$}
\def\adaptors{\vindex{adaptor}~$\!\aut A_1$\kern-2pt{\rm's}}
 \def\governor{\vindex{governor}~$\aut G$}
 \def\governors{\vindex{governor}~$\aut G$\kern-0.8pt{\sl's}}
 \def\body{\vindex{body}~$\!\aut B$}
 \def\bodys{\vindex{body}~$\!\aut B$\kern-0.8pt{\sl's}}
\def\learner{\vindex{learner}~$\!\aut A_2$}
\def\learners{\vindex{learner}~$\!\aut A_2$\kern-1.5pt{\rm's}}
 \def\modeler{\vindex{modeler}~$\!\aut M$}
 \def\modelers{\modeler\kern-0.8pt{\sl's}}
 \def\simulator{\vindex{simulator}~$\!\aut S$}
 \def\simulators{\simulator\kern-0.8pt{\sl's}}
 \def\reality{\vindex{reality}~$\aut R$}
\def\knower{\vindex{knower}~$\!\aut A_3$}
\def\knowers{\vindex{knower}~$\!\aut A_3$\kern-1.2pt{\rm's}}
 \def\intelligence{\vindex{intelligence}~$\syn A$}
 \def\mind{\vindex{mind}~$\syn M$} % $\Re$
 \def\minds{\mind{\rm's}}
\def\subject{\vindex{subject}~$\aut A_4$}
\def\subjects{\vindex{subject}~$\aut A_4$\kern-1.5pt{\rm's}}
 \def\inquirer{\vindex{inquirer}~$\syn I$}
 \def\inquirers{\inquirer\kern0pt{\rm's}}
 \def\reason{\vindex{reason}~$\syn R$}
 \def\self{\vindex{self}~$\syn X$}

\def\Corporal#1$#2#3${\vindex{#1}~${\aut #2}#3$}
\def\corporal#1$#2#3${#1~${\aut #2}#3$}
\def\Mental#1$#2#3${\vindex{#1}~${\syn #2}#3$}
\def\mental#1$#2#3${#1~${\syn #2}#3$}

\def\tURING{\vperson} % just a trick to alfabetize OK
\def\TM{\tURING[Turing] machine%
 \hindex{\string\string\string\tURING[Turing] machine}~$\mathop{\frak T}$}
\def\TMes{\tURING[Turing] machines%
 \hindex{\string\string\string\tURING[Turing] machine}~$\mathop{\frak T}$}
\def\UTM{universal \tURING[Turing] machine%
 \hindex{universal \string\string\string\tURING[Turing] machine}~$\mathop{\frak U}$}
\def\UTMes{universal \tURING[Turing] machines%
 \hindex{universal \string\string\string\tURING[Turing] machine}~$\mathop{\frak U}$}

\def\processor$#1${processor~${\cal P}_{\frak #1}$}
\def\Processor$#1${\vindex{processor}~${\cal P}_{\frak #1}$}
\def\UP{\vindex{universal processor}~${\cal P}_{\frak U}$} % cal

\def\true{\hbox{\sc true}}
\def\false{\hbox{\sc false}}

\def\llave#1{\inmmode$\left\lbrace\vcenter{
 \halign{&\hbox{\rm\strut##}\hfil\cr#1\crcr}}\right.$}
\def\lopen#1#2{\inmmode$\left#1\vcenter{
 \halign{&\hbox{\rm\strut##}\hfil\cr#2\crcr}}\right.$}

\def\etapa{\inmmode$\mapstochar\Rightarrow$} % for the reader's guide
\def\etapa{\inmmode$\mathrel{\vrule height 3.5pt depth-1.5pt}\kern-1pt
  =\kern-1pt\mathrel{\triangleright}$}

% KEYS

\def\meaning#1{{\it#1\aftergroup\itcor}}
\def\latin#1{{\it#1}\itcor}
\def\booktitle#1{{\sl#1}\itcor}

% Index: 0, for cite, 1 for definition, 2 for person, 3 for index.

% VISIBLE       HIDDEN
% ^{index}      _{index}
% ^[Person]     _[Person]
% ^(cite1999)   _(cite1999)
% ^|definition| _|definition|
% ^<booktitle>  _<label>
% ^>ref>        _>ref>   ( V = \S2.1, pág.~123     H = \S2.1 )

\catcode`\^=7 \catcode`\_=8
\def\specialhat{\ifmmode\def\next{^}\else\let\next=\beginvref\fi\next}
\def\specialund{\ifmmode\def\next{_}\else\let\next=\beginhref\fi\next}
\catcode`\^=13 \let^\specialhat \catcode`\_=13 \let_\specialund

\def\beginvref{\futurelet\next\beginvrefx}
\def\beginhref{\futurelet\next\beginhrefx}

\def\beginvrefx{\begingroup
 \ifx\next[\aftergroup\vperson \else
 \ifx\next(\aftergroup\vcite \else
 \ifx\next|\aftergroup\vdefinition \else
 \ifx\next<\aftergroup\vlabel \else
 \ifx\next>\aftergroup\vref \else
 \aftergroup\vindex \fi\fi\fi\fi\fi \endgroup}
\def\beginhrefx{\begingroup
 \ifx\next[\aftergroup\hperson \else
 \ifx\next(\aftergroup\hcite \else
 \ifx\next|\aftergroup\hdefinition \else
 \ifx\next<\aftergroup\hlabel \else
 \ifx\next>\aftergroup\href \else
 \aftergroup\hindex \fi\fi\fi\fi\fi \endgroup}

\def\allowhyphens{\nobreak\hskip\z@skip}

\def\vperson[#1]{\leavevmode\ndx{#1}{2}\allowhyphens{\sc#1}}
\def\hperson[#1]{\allowhyphens{\sc#1}}
\def\vcite(#1){\ndx{#1}0\footnote{\ref{#1}.}}
\def\hcite(#1){\ndx{#1}0\ref{#1}}
\def\vdefinition|#1|{\leavevmode\Presignal\ndx{#1}{1}%
 \Green\allowhyphens#1\allowhyphens\Black\Signal}
\def\hdefinition|#1|{\allowhyphens\ndx{#1}{1}}%\tomargin{\it#1}}
\def\vlabel<#1>{{\sl#1}\itcor} % used for book titles only
%\def\hlabel<#1>{\label{#1}}
\def\vref>#1>{la~\S\refsc{#1}, pá\-gi\-na~\refpg{#1}}
\def\href>#1>{la~\S\refsc{#1}}
\def\vindex#1{\leavevmode\presignal\ndx{#1}{3}%
 \Blue\allowhyphens#1\allowhyphens\Black\signal}
\def\hindex#1{\allowhyphens\ndx{#1}{3}\hsignal}

%\def\signal{\hbox to0pt{\Blue\hss\vbox to0pt{
% \hbox{$\longleftarrow$}\vss}\Black}}
\def\Signal{\hbox to0pt{\hss\vrule width10pt height-2pt depth2.4pt
 \vrule width0.4pt height4pt depth2pt}}
\def\Presignal{\hbox to0pt{\vrule width0.4pt height4pt depth2pt
 \vrule width10pt height-2pt depth2.4pt\hss}}
\def\signal{\hbox to0pt{\hss\vrule width10pt height-2pt depth2.4pt}}
\def\hsignal{\hbox to0pt{\hss\vrule width10pt height-2pt depth2.4pt}}
\def\presignal{\hbox to0pt{\vrule width10pt height-2pt depth2.4pt\hss}}


%% OPTION comment out to get signals
\let\Signal=\null \let\Presignal=\null
\let\signal=\null \let\presignal=\null
\let\hsignal=\null

\def\EPA#1{{\sc epa}\ndx{Casares1999}0~\S{\tenrm#1}} %%%%%%%%%% Provisional

% TEXT

\catcode`\@=12

\input of0.tex

\let\par=\destpar

\input of1.tex
\input of2.tex
\input of3.tex
\input of4.tex
\input of5.tex
\input of6.tex

\let\par=\endgraf

\catcode`\@=11
\catcode`\^=7 \catcode`\_=8
\everypar{}\parindent=20pt

\vfill\break %%%%%%%%%%%%%%%%%%%%%%%%%%%%%%%%%%%

\pdflabel
\vbox to22pt{\centerline{\fontone References}\toc9{References}\vss}
\vskip 1pc plus1pc minus6pt

\noindent The editions of the works consulted are given.
The numbers that appear at the end of each entry, after the
triangle~{\teni.}, indicate the pages on which the work is cited. Thus,
{\teni.}\refpg{The Chomsky Hierarchy} means that the reference is on
page~\refpg{The Chomsky Hierarchy}.

\newwrite\bibnotes \immediate\openout\bibnotes=auxiliar.abdb
\iverb\bibnotes"% auxiliar.abdb"
\iverb\bibnotes"\record{Lettvin1959}{INCOLLECTION}{"
\iverb\bibnotes" \AUTHOR{Lettvin, J.\kern-1pt Y\kern-1pt.; "
\iverb\bibnotes" Maturana, H.R.; McCulloch, \kern-1ptW\kern-2pt.S.; "
\iverb\bibnotes" Pitts, W\kern-2pt.H.}}"
\iverb\bibnotes"\record{Rumelhart1986}{BOOK}{"
\iverb\bibnotes" \AUTHOR{Rumelhart, D.E.; McClelland, J.L.; \& the PDP Research Group}}"
\iverb\bibnotes"\endinput"
\immediate\closeout\bibnotes

\newcount\lastpgno \newcount\thispgno
\newif\ifrange

\def\gobblethree#1#2#3{\ignorespaces}

\def\ndxline#1#2{\ifnum#2=0 \let\oldk@y=\newk@y \def\newk@y{#1}%
 \let\next=\bibline \else \let\next\gobblethree \fi \next}

%\def\bibbreak{\ifdim\lastskip<\smallskipamount \removelastskip
% \penalty-200 \vskip3pt plus 2pt \relax\fi} %plus 1.5pt minus 1pt \fi}
%\let\bibbreak=\filbreak
\def\bibbreak{\par\vskip-2pt plus 1fil \penalty-200 \vskip2pt plus -1fil\relax }

\def\bibline#1#2#3{\ifx\newk@y\oldk@y \thispgno=#3
 \ifnum\lastpgno=\thispgno \else \advance\lastpgno1
  \ifnum\lastpgno=\thispgno \rangetrue \else
   \ifrange \advance\lastpgno-1 --\number\lastpgno\fi
   \lastpgno=#3\rangefalse
   , \pdfgoto{#3}{#1}%
  \fi\fi
 \else % \newk@y<>\oldk@y
 \ifrange --\number\lastpgno\rangefalse \fi
 \bibbreak
 \pdflabel \cleantoks \lastpgno=0
 \def\record##1##2##3{\def\secondk@y{##1}\ifx\newk@y\secondk@y##3\fi}
 \input auxiliar.bdb
 \input auxiliar.abdb
 \hang\noindent\rm{\sc\the\AUTHOR}:%
 {\def\em{\string\em\space}\def~{\string~}\def\it{\string\it\space}%
  \def\sc{\string\sc\space}%
  \lbl{\newk@y}{\string\rm{\string\sc\space
  \expandafter\expandafter\the\AUTHOR}\space
  (\expandafter\expandafter\the\YEAR):\space
  {\string\sl\space\expandafter\expandafter\the\TITLE}}}%
 {\sl\null{ }\the\TITLE
 \setbox0=\hbox{\the\SUBTITLE}\ifdim\wd0=0pt\else\the\SUBTITLE\fi}%
 \setbox0=\hbox{\the\YEAREDITION}\ifdim\wd0=0pt
  \YEAREDITION={\the\YEAR}\else\ (\the\YEAR)\fi
 \setbox0=\hbox{\the\EDITOR}\ifdim\wd0=0pt\else.
   Recopilado por \the\EDITOR\fi
 \setbox0=\hbox{\the\BOOKTITLE}\ifdim\wd0=0pt\else
   \ en ``\the\BOOKTITLE''\fi
 \setbox0=\hbox{\the\EDITION}\ifdim\wd0=0pt\else. \the\EDITION\fi
 \setbox0=\hbox{\the\PUBLISHER}\ifdim\wd0=0pt\else, \the\PUBLISHER\fi
 \setbox0=\hbox{\the\ADDRESS}\ifdim\wd0=0pt\else, \the\ADDRESS\fi
 \setbox0=\hbox{\the\JOURNAL}\ifdim\wd0=0pt\else, \the\JOURNAL\fi
 \setbox0=\hbox{\the\VOLUME}\ifdim\wd0=0pt\else, vol.~\the\VOLUME\fi
 \setbox0=\hbox{\the\MONTH}\ifdim\wd0=0pt
  \setbox0=\hbox{\the\YEAREDITION}\ifdim\wd0=0pt\else, \the\YEAREDITION\fi
  \else, \the\MONTH\ de \the\YEAREDITION\fi
 \setbox0=\hbox{\the\PAGES}\ifdim\wd0=0pt\else, pp.~\the\PAGES\fi
 \setbox0=\hbox{\the\ISBN}\ifdim\wd0=0pt\else, {\sc isbn} {\the\ISBN}\fi.
 \setbox0=\hbox{\the\NOTE}\ifdim\wd0=0pt\else \the\NOTE. \fi
 {\teni.}\pdfgoto{#3}{#1}\fi
 \ignorespaces}

\bigskip

\input auxiliar.abc


\vfill\break %%%%%%%%%%%%%%%%%%%%%%%%%%%%%%%%%%%

% Output for INDEX adapted from [417]

\newdimen\fullhsize \fullhsize=\hsize
\newdimen\fullvsize \fullvsize=\vsize
\def\fullline{\hbox to\fullhsize}

\newdimen\gutter \gutter=1pc
\newbox\partialpage
\def\begindoublecolumns{\begingroup
 \output={\global\setbox\partialpage=\vbox{\unvbox255\bigskip}}\eject
 \output={\doublecolumnout}%
 \hsize=\fullhsize \advance\hsize by-\gutter \divide\hsize by2
 \vsize=\fullvsize \multiply\vsize by2 \advance\vsize by2pc}
\def\enddoublecolumns{\output={\balancecolumns}\eject
 \endgroup \pagegoal=\vsize}

\def\doublecolumnout{\splittopskip=\topskip \splitmaxdepth=\maxdepth
 \dimen0=\fullvsize \advance\dimen0 by-\ht\partialpage
 \setbox0=\vsplit255 to\dimen0 \setbox2=\vsplit255 to\dimen0
 \shipout\vbox{\vbox to 0pt{\vskip-22.5pt
  \fullline{\vbox to8.5pt{}\the\headline}\vss}\nointerlineskip
  \vbox to\fullvsize{\boxmaxdepth=\maxdepth \pagesofar}
  \baselineskip=24pt \fullline{\the\footline}}\advancepageno
 \unvbox255 \penalty\outputpenalty}
\def\balancecolumns{\setbox0=\vbox{\unvbox255}\dimen0=\ht0
 \advance\dimen0 by\topskip \advance\dimen0 by-\baselineskip
 \divide\dimen0 by2 \splittopskip=\topskip
 {\vbadness=10000 \loop \global\setbox3=\copy0
  \global\setbox1=\vsplit3 to\dimen0
  \ifdim\ht3>\dimen0 \global\advance\dimen0 by1pt \repeat}
 \setbox0=\vbox to\dimen0{\unvbox1}%
 \setbox2=\vbox to\dimen0{\dimen2=\dp3 \unvbox3 \kern-\dimen2 \vfil}%
 \pagesofar}
\def\pagesofar{\unvbox\partialpage
 \wd0=\hsize \wd2=\hsize \fullline{\box0\hfil\box2}}

%%%

\pdflabel
\vbox to22pt{\centerline{\fontone Index}\toc9{Index}\vss}
\vskip 1pc plus1pc minus6pt
%\vskip 5.87955pt % just to make exact room for columns

\noindent The concepts in boldface are the most important concepts
of the theory of subjectivity; the numbers in boldface indicate the
pages on which they are defined. The other concepts are less important,
and the other numbers indicate on which pages the different concepts and
authors appear, independently of their importance.

\lastpgno=0 \thispgno=0 \rangefalse

\def\ndxline#1{\let\oldkey=\newkey \def\newkey{#1}\defline}
\def\defline#1#2#3#4{\ifnum#1=0 \else % not biblio
 \ifx\newkey\oldkey \thispgno=#4
  \ifnum\lastpgno=\thispgno \else \advance\lastpgno1
   \ifnum\lastpgno=\thispgno \rangetrue \else
    \ifrange \advance\lastpgno-1 --\number\lastpgno\fi
    \lastpgno=#4\rangefalse
    , \pdfgoto{\ifnum#1=1 {\bf#4}\else #4\fi}{#2}%
   \fi\fi
 \else % \newkey<>\oldkey
  \ifrange --\number\lastpgno\fi
  \bibbreak\hang\noindent \lastpgno=#4\rangefalse
  \ifcase#1 {\tt\newkey}\or{\bf\newkey}\or{\sc\newkey}\or{\rm\newkey}\else
   {\it\Red \newkey\Black}\fi
 ,\space\space
 \pdfgoto{\ifnum#1=1 {\bf#4}\else #4\fi}{#2}\fi\fi\ignorespaces}

\begindoublecolumns
\parindent=12pt \rightskip=0pt plus 4pc \hyphenpenalty=250
\input auxiliar.abc
 \ifrange --\number\lastpgno \fi
\enddoublecolumns

\vfill\break %%%%%%%%%%%%%%%%%%%%%%%%%%%%%%%%%%%

\pdflabel
\vbox to22pt{\centerline{\fontone Contents}\toc9{Contents}\vss}
\vskip 2pc plus1pc minus6pt

% \tocline{level}{Title}{dest}{section}{page}
\def\tocline#1{\ifcase#1 \let\next=\toclinezero \or
 \let\next=\toclineone \or \let\next=\toclinetwo
 \else \let\next=\toclineindex \fi \next}

\def\toclinezero#1#2#3#4{\relax}
\def\toclineone#1#2#3#4{\bigbreak
 \line{\bf \pdfgoto{#1}{#2}\hfil}\ignorespaces}
\def\toclinetwo#1#2#3#4{\par
 \contentsline{#1}{#2}{$\S#3$\quad}{#4}\ignorespaces}
\def\References{References}
\def\toclineindex#1#2#3#4{\par
 \def\1{#1}\ifx\1\References\bigbreak\fi
 \contentsline{{\bf #1}}{#2}{}{#4}\ignorespaces}

\newdimen\tridig \setbox0=\hbox{$\S127$\quad}\tridig=\wd0

\def\contentsline#1#2#3#4{\setbox2\hbox{#1}%
 \setbox0\hbox to\tridig{\hfil#3}%
 \dimen0=\hsize \advance\dimen0 by -\wd0
 \multiply\dimen0 by 8 \divide\dimen0 by 10
 \ifdim\dimen0>\wd2 \line{\box0 #1\tocleaders \pdfgoto{#4}{#2}}\else
  \line{\box0 \vtop{\hsize=\dimen0 \raggedright \normalbaselines
   \let\\=\ \noindent #1\strut}\tocleaders \pdfgoto{#4}{#2}}\fi}
\def\tocleaders{\leaders\hbox to\baselineskip{\hss\bf.\hss}\hfil}

\raggedbottom \openup3pt
\input auxiliar.toc
\openup-3pt

\vfill % Colofón

\begincenter \obeylines % \normalbottom \baselineskip=12pt
 This book has been typeset by the author using
 the tools provided by Profesor D.~E.~Knuth (Stanford University).
 I have used {\em his} {\TeX} program
 to compose {\em my} text
 with {\em his} Computer Modern fonts,
 and to set {\em my} figures,
 that I have drawn with {\em his} {\logo METAFONT} program.
\endcenter

\ifportada %%%%%%%%%% Contraportada

\break

\shipout\vbox{\background
\pdfWhite
\hsize=5.5in
\leftskip-0.55in
\vskip-0.55in
\begingroup\pcitefont\parindent=0pt\baselineskip=24pt\obeylines

\kern12pt
``Life is a bubble of knowledge and
\leavevmode\kern12pt freedom'' {\ptextfont(§154 ¶2)}
\kern12pt

\endgroup

\vskip 1pc

\begingroup\ptextfont\parindent=0pt\baselineskip=16pt
 \setbox0=\hbox{0}\obeylines

``On Freedom'' provides a link that connects life to symbolism.
This link sheds light on the true nature of symbolism
and makes clear the relation between its two layers:
semantics and syntax.
Thus, the book gives a Darwinian explanation of symbolism.
So then, where is freedom?
I haven't mentioned yet that the link is a problem,
more specifically the problem of survival,
and there is no problem without freedom.
\null
These are my ten propositions on freedom:
\kern3pt\parindent=\wd0
1 Life is a problem.
2 A problem is freedom and a condition.
3 Semantics cannot represent freedom,
\qquad which is free of meaning.
4 Syntax, with free terms, is needed to represent freedom.
5 A resolution is a syntactical transformation.
6 To represent resolutions, a recursive syntax is needed.
7 A symbolism, with semantics and recursive syntax,
\qquad can represent problems, resolutions, and solutions.
8 A subject is a symbolic live being.
9 Man is the only living subject.
\noindent10 Man is free and conscious.

\endgroup

\kern48bp

\pdfBlack
\input ean13
\hbox{\kern-0.55in
\pdfximage{cc-by-sa-www.pdf}\pdfrefximage\pdflastximage
\kern2.05in
\pdfliteral{q}% save current graphic state in the stack
  \pdfliteral{0 0 0 0 k 0 0 0 1 K}% set black stroke on white fill
  \pdfliteral{-5 -5 m}% moves to the origin
  \pdfliteral{110 -5 l}%
  \pdfliteral{110 92 l}%
  \pdfliteral{-5 92 l}%
  \pdfliteral{-5 -5 l}%
  \pdfliteral{b Q}% close, stroke, fill, and restore graphic state
\ISBN 978-1-4750-4270-2
\barheight=2.5cm % this must be after \ISBN call
\EAN 978-1475042702
\hskip9pt
}
}

\else \let\vfill=\relax \fi

\bye

