% OL1.TEX (RMCG19991123)

\Part Prolegomena

\Section Purpose

The object of this essay is ^{free will}. It deals, therefore, with
^{freedom}; not with any kind of freedom, but with ^{free will}
according to the ^{theory of subjectivity} presented in the book  ^<El
problema aparente>, which I will refer to as {\sc epa}^(Casares1999).
Nevertheless, this text stands by itself, so it is not necessary to have
read ^<El problema aparente> to understand it.

^<El problema aparente> provides a mathematical presentation of the
epistemological question, giving ^{epistemology} the meaning that
^[Descartes] would give it: if a subject receives data that are raw but
have a value, that is, some data are beneficial to him and some are
harmful, and if this subject has no additional \latin{a priori}
knowledge of his exterior ^{environment}, what can this subject know?
Once this ^{problem} is set forth mathematically, ^{mathematics} itself
resolves it.

This way of presenting the reasoning seeks to avoid philosophical
difficulties. Those difficulties, if any, will be limited to asking
whether the mathematical problem is an adequate model for the
epistemological issue, or whether the resolution arrived at, not being
the only possible resolution, actually fits the facts. Thus, technical
obstacles apart, the road is philosophically smooth.

But it is not smooth for everyone. Some will find the conclusions
unexpected or, even worse, ^{absurd}. Those who hold this opinion have
several options. By recurring to the rule of \latin{reductio ad
absurdum}, they may decide that, as the conclusions of a well-developed
reasoning are absurd, the starting postulates are false. Another
possibility is that they may label the situation paradoxical,
considering the premises to be true, the inference correct, and the
conclusion false. Or they may end up by accepting that the conclusions,
although unexpected and surprising, are also true.

To facilitate this last option, we will reason in both directions: in
the ^{entry} direction, that is, from outside to inside, and in the
^{exit} direction, that is, from inside to outside. The idea is, in
short, to present all of the very diverse consequences of the theory of
subjectivity in order to show its full explanatory power. Anyone who is
not inclined to accept the theory will have to substitute another one
that has at least as great a scope as this one.


\Section Some Advice

This essay presents many new concepts and, what is worse, many everyday,
fundamental concepts are interpreted in a peculiar or even
extravagant_{extravagance} way. In addition (and this is exclusively my
fault), I do not idly repeat explanations. If I do repeat something, it
is to introduce new matters and so this essay is conceptually very
dense. In my defense, avoiding repetitions shortens the essay. You,
gentle reader, can still make the repetitions that I omit on your own;
in this way, if you do not need them, you need not put up with them.

The greatest complication probably arises from the variety and diversity
of disciplines involved in this essay:
 ^{ethics}, ^{philosophy}, ^{epistemology},
 ^{linguistics}, ^{logic}, ^{mathematics},
 ^{computation}, ^{cognition}, ^{psychology},
 ^{neurology}, ^{biology}, and ^{physics}.
No one can have a complete knowledge of all of them, and so each of you
will have a different ^{perspective} of the theory, depending on your
formation and temperament.

Seeing things in perspective, which is always unavoidable, distorts
matters even more in this case for two reasons. The first reason, which
has already been noted, is that since the ^{sculpture} has a great
number of dimensions, the number of different views is enormous. This
makes it even more difficult to integrate them into a single coherent
object. The second and more serious reason is that even the sculptor has
limited knowledge, causing the sculpture to have a preferred viewing
point; thus, from other positions, the creation could possibly not
coincide at all with the sculptor's intentions. If this reasoning is
correct, the greater the divergence between your knowledge and mine, the
more errors you will find; I apologize for these errors. Thus, in order
not to vitiate the empirical investigation of this matter, I will not
reveal my interests.

Under these circumstances, there is no magical_{magic} formula to insure
that this essay will be comprehended. It would, however, be impossible
to comprehend if it were read with prejudices. My advice to you, kind
reader, is that you be ^{patient}; open your understanding to novelties
and please suspend judgment until you completely understand the theory,
because the details are less important than the whole. Let yourself be
carried away by the lucubrations, however improbable they may seem.
Discovering unexpected consequences is interesting and, I hope, fun. Let
us begin.


\Section A Small Difference

Which is easier, telling the difference between cats and dogs, or
calculating square roots?

For a person, it is easier to distinguish a ^{cat} from a ^{dog} than to
find the ^{square root} of a number. You don't even have to go to school
to tell a cat from a dog. For a cat or a dog, or even for a ^{mouse}, it
is also simpler to distinguish cats from dogs than to calculate square
roots, a task that is totally impossible for them.

In spite of this unanimity, it turns out that, from the point of view of
an engineer_{engineering} given the job of designing a machine to do
these tasks, it is simpler to calculate square roots than to recognize a
cat. It is easier to build a calculating machine that can do square
roots than it is to build a machine that can tell cats from dogs. To put
it another way: many more computational_{computation} resources are
needed to identify cats than to calculate square roots.

This error of appraisal is one of the first and most interesting
discoveries of ^{artificial intelligence}, the name given to one of the
branches of the new ^{cognition} sciences. The consequences are
immediate: if a dog is as capable of identifying cats as a person, we
deduce that the difference between the computational capacities of
people and dogs is to be found in minor, not major, aspects. There is a
small difference that apparently has very great consequences. This
conclusion agrees with the Darwinian theory of ^{evolution}, and so does
not surprise us.

Throughout the following text, we will try to describe what this small
difference actually is. Our insistence on this difference may make it
seem large. It is not.


\Section Down with Objectivism!

^[Descartes]' return to the origins of knowledge is unobjectionable. It
is difficult to refute the idea that the only immediate and direct
knowledge, the only thing I know without a doubt, is my own ^{self}.
Everything else can only be known indirectly.

But we open our eyes and we can see a ^{stone} clearly. We must reflect
in order to remember ^[Descartes]' prescriptions, and even so it is
difficult to doubt the stone's existence. Further reflection serves to
make us realize that what we really perceive is some colors and lights
that we identify as a stone, not the stone itself. So then we take the
stone and feel it, and the ^{Cartesian doubt} fades away for a second
time. A little further reflection shows us that the situation has not
undergone any fundamental variation, because holding the stone only
provides additional data about its shape, size, weight, roughness, and
temperature. The stone continues to be the result of a deduction made
from the data.

What is confusing about this process is that the deduction, even though
it is the most costly part computationally speaking, is unconscious and
automatic. It is easier for us to make a deduction than it is to
discover that we are making a deduction, and so the deduction goes
unnoticed if we do not pay close attention. Even so, the situation is so
strange that it seems too much to doubt the stone's existence.

Our brain carries out this imperceptible process for judging objects
because of its aptitude for surviving. It forms part of our genetic
inheritance and need not be learned. It is important to observe that the
process of objectification is previous to the process of
^{symbolization} that makes ^{speech} and symbolic ^{consciousness}
possible; symbolization also makes ^{square root} calculations possible.
It is important because it explains that, for the symbolic substrata of
our thinking, the object, in this case a stone, is a piece of data and
not the result of a deduction. Therefore, although for the brain as a
whole, the data are the colors our eyes capture and the shapes our hands
feel, for our symbolic consciousness the data are the objects.
$$\hbox{Phenomenon} \underbrace{\longrightarrow \hbox{Object}
 \longrightarrow
 \overbrace{\hbox{\strut Word}}^{\hbox to 0pt{\hss
  Consciousness\hss}}}_{\hbox{Brain}}$$

To be coherent with the previous conclusion, we must completely abandon
^{ontology}. The existence of objects is a construction of the brain.
Objects and each and every one of their properties depend on the subject
who perceives them. Subjectivism_{subjectivism} becomes the only
possible alternative.

The basis for this proof of subjectivism is that, of the two cognitive
processes considered, objectification is previous to symbolization.
Anyone who is not clear on this point has only to think that you cannot
speak of something that you haven't thought of yet; therefore, in order
to be able to speak about objects, these objects must be previous to
speech. One special case of this that is handy here is the case of the
illusions_{illusion} that occur when conscious symbolic processes
discover an error in other cognitive processes; these other cognitive
processes necessarily occur previously to the ones that discover them to
be erroneous. These illusions are disturbing because they reveal that
what we see may not be the way we are seeing it. And this is precisely
what subjectivism affirms: there are no objects out there.

The theory of subjectivity affirms, then, that of everything that is not
my own self, what we could call the external ^{universe}, the only truth
we have is a torrent of raw data. The data that appear to be immediate
to our symbolic conscience are already elaborated data. The preparation
of these data follows recipes that, on one hand, have favored the
survival of our predecessors and, on the other hand, turn sensations
into objects. And that is all there is to it.

I know that, in spite of its apparent logic, it is hard to accept all of
this. It requires us to understand that things are not as they appear to
us consciously; things are not obvious. But even though knocking down
the objectivist theory seems to leave us without any ground under our
feet, it is advisable to take note of  two arguments, a theoretical one
and a practical one. To all practical effects, we can continue to reason
as objectivists, with the certainty that objectification has worked
successfully for millions of years with no catastrophic failures. To all
theoretical effects, and if all this is correct, the ground that the
objectivist theory puts under our feet is illusory. We may as well be
faithful to our principles and adopt subjectivism fearlessly, if we hope
to achieve an exact understanding of what the ^{self}, ^{consciousness},
and the ^{world} are.


\Section Objective Reality Is Subjective

What do you see?  A child on a swing. No, an
Impressionist_{Impressionism} painter would answer, you see colors,
spots of color. It is as if we have glasses that add labels_{label} to
what we see. % [Cover: A child on a swing]

Everything out there changes_{change}. If we saw the image that our
retina captures on television, we would get dizzy. Because ^{dizziness},
in non-pathological conditions, happens when uncontrolled movement
occurs; for example, the movement of the waves when we are on ship, or
of a car when we are not driving, or even the image on the television if
someone else changes channels too quickly. This means that we get dizzy
when our stabilization system doesn't have the data it needs to get
ahead of our perceptions, that is, when we cannot stabilize what we are
looking at.

Let us distinguish sensation from perception. We will call the sensorial
impression ^{sensation}. An example would be the image captured by the
eye's retina. We will use the words ^{sense} and ^{sensation} with this
meaning exclusively. Perception_{perception} is the process that takes
the sensation and produces the stabilized, labeled things that we call
objects_{object}. Sensations change, but what we perceive doesn't. We
will frequently use `^{see}' as a synonym of `perceive', although there
are other perceptive modes such as ^{hear} or ^{taste}.

We will call the things that we see reality, not the changing sensations
that make us dizzy. So ^{reality} is what we perceive, not what we
sense. The impression on our retina, which we cannot know, is not real;
proof of this is that we ignore the existence of the blind spot, see
^[Resnikoff]^(Resnikoff1989). The stone that we see, the object seen, is
real and, consequently, the reality of the objects is processed.

We cannot avoid seeing objects, even though these objects do not exist
outside of our heads. We do not see the universe as it is, or, rather,
we do not see the universe as our senses capture it. We see it as we see
it. The reality of the objects, objective reality, is a construction of
the subject; that is, it is subjective. In short, objective reality is
subjective.
$$\overbrace{\hbox{\strut Phenomenon}}^{\hbox{Universe}}
 \underbrace{\longrightarrow
  \overbrace{\hbox{\strut Object}}^{\setbox0=\hbox{Reality}\dp0=0pt\box0}
  \longrightarrow
  \overbrace{\hbox{\strut Word}}^{\hbox to 0pt{\hss\quad
   Consciousness\hss}}}_{\hbox{Subject}}$$


\Section Dreams

Dreams reveal labels. Excerpt from a ^{dream}: `You were there,
^[Piripili],  but you looked and talked like your mother'. Because
dreams label things incorrectly, they reveal what the labels are and
that it is the labels that are important, not the appearances. But it
isn't exactly the label that is important, either. What is important is
not the  ^{label}, ^[Piripili], but its meaning, \meaning{you}.

^[Eco]^(Eco1997) ends up pointing out the surrealist_{surrealism} and
oniric character of rebus puzzles, because they also confuse sensations
and words. Note that in the dream about ^[Piripili] and her mother,
there is no way to visualize the scene. It must be explained with words.
Just as if it were a ^{rebus}, we have to put the label ^[Piripili] on
what is shown, to all effects, as the mother.

I have no ^{will} in my dreams, in contrast to my waking, conscious
state. It seems to me that evolution, having undergone no adaptational
pressure on this point, has not bothered to adequately distinguish the
role of symbolism, that is, labels, in dreams, as it has had to do in
our waking moments. It was this unsettling aspect of dreams that allowed
^[Freud]^(Freud1900) to discover the error by which the ^{subject}
identifies with his conscious ^{self}.

The ^{subject} does not see himself as a subject but as a ^{self}; this
means that the subject identifies with the conscious part of himself and
with his will. This error of perspective explains why the subject
understands the objects to be external, not internal. And this error is
perverse because it is in the subject's own interest: if the subject
were identical with his will, he would not have to die_{death}.


\Section Reality Is Involuntary

But just as the ^{will} is conscious, or else it is not will, the
process of objectification, on the other hand, is not conscious, but
previous to ^{consciousness} and automatic. As we saw in ^>Down with
Objectivism!>,  the object is the result of a process with an
evolutionary design. To put it another way, the ^{program} for
objectification is encoded in the genes. That is why the object, even
though it is subjective, is not at the mercy of the subject's will.
Thus, objective reality is subjective, but involuntary. Reality is
involuntary.

Even ^{physics}, when it manages to describe objects, is necessarily
part of ^{psychology}. Such is the case of quantum mechanics, which
reaches the limits of the object. It can be no other way, if objects are
the products of cognition, that is, if objects are subjective. The
objectivity of physical science, which seems to lift it above the
uncertain and whimsical subjective ^{world}, is not due to the objects'
autonomous existence, but to their being beyond the reach of the
subject's will.

\endinput
