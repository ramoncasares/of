% OL2.TEX (RMCG20000412)

\Part Entry

\Section The Adaptor

Up to this point we have noted that objects intervene in the processes
that determine how we see, that is, in our ^{perception}, and we have
accepted the idea that such processes were designed by ^{evolution}. We
will now investigate why this happened, and to that end the first aspect
that we must elucidate is what purpose the objects serve.

Objects_|object| simplify sensation. The quantity of data that our
senses capture is enormous, too large to treat fully. Fortunately, the
purpose of getting these data is to decide what is the right thing to do
according to the circumstances, and that generally depends on a small
amount of data. The strategy consists, then, of taking the data only to
decide if these few but important things are present or not.

In order to focus the explanations that follow, we are now going to
establish some basic definitions. Sensation_{sensation} is the
impression of an exterior phenomenon on the ^{body}, on the senses. To
sense is to receive or capture the sensation. To perceive or see is to
recognize the objects present through the signs_{sign} or indications
detected in the sensation. So ^|perception| is what we call the process
of converting the phenomenon into an object that is ^{present}, of
converting what is captured by the senses into what is perceived.

This simplifying perception has been exploited by evolution from time
immemorial. Thus, objects mediate between the phenomenon and the action
of some living beings that we will call adaptors_|adaptor|. These
adaptors are animals that have a nervous system. The ^{nervous system}
has the job of deciding, at each moment, the action that will actually
be taken from among all the possible actions of its body. This decision
uses the objects that are present as data.
$$\hbox{Phenomenon}
  \underbrace{
   \longrightarrow
   \hbox{\strut Object}
   \longrightarrow}_{\hbox{\strut Adaptor}}
  \hbox{Action}
$$

Such is the case of the ^{frog}, see ^[McCulloch]^(Lettvin1959)
\latin{et alii}, which interprets any dark point that moves rapidly in
its field of vision as a ^{fly} which it will try to eat. This
interpretation has proven useful and has been preserved in the frog's
genetic code.

This object that the frog sees, and that I have called a fly, does not
coincide with any objects from a person's perspective. It is more than a
fly and less than an insect. It isn't a ^{bug} either, that people find
repugnant and frogs tasty. In all purity, this object only exists within
the frog's nervous system. If any correspondence at all, even partial,
can be made with our bugs, it is because the difference between frogs
and us is not as great as we think (see ^>A Small Difference>). The
frog's fly exists_{existence} because it is useful to frogs and the
person's bug exists because it is bothersome for people.

According to these ideas, the frog generalizes and uses universals,
without needing metaphysical_{metaphysics} abstractions_{abstraction} in
order to do so. This capacity for ^{generalization} rests on the
perception process that groups different phenomena together in the same
object. For a frog to survive, all it needs is to be able to distinguish
cats from flies.


\Section Objects

The primary characteristics of the ^{object} are already present in the
adaptor. The object becomes ^|present| when the nervous system has
enough signs_{sign} to make it present. These signs originate directly
from ^{sensation}, but they can also come from other objects. This
allows us to take advantage of objects' ^{contiguity}, since some, like
^{smoke} and ^{fire}, usually appear together, while others never do so.
The presence of objects can provoke actions and it can lull or awaken
the presence of other objects, as we have just seen.

The presence_{present} of the object has no meaning beyond the adaptor.
The presence of the object is simply the result of certain calculations
carried out by the adaptor's nervous system; these calculations are, in
any case, vouched for by its evolutionary efficacy. Present objective
reality is completely inferential and is built partly by genetically
coded information and partly by calculations made by the nervous system,
calculations that we have called ^{perception}.


\Section Reality

As soon as an object becomes present, the ^{process} associated with
this object starts up. These processes have two kinds of effects: they
can influence other objects or they can influence the rest of the body.
If these processes influence other objects, they can do so in two
manners: positively, if they make them become present, or negatively if
they return them to a latent state. With this terminology, we can affirm
that objects constitute a network of ^{concurrent processes} (see
^[Pdp]^(Rumelhart1986)), that we call ^|reality|.
$$\hbox{Reality} = \hbox{Network of Objects}$$

We can also distinguish between ^{action} and ^{behavior}. To simplify,
you can assimilate a computer's behavior to the ^{program} that it is
executing. Thus, the key used to mark the end of a line when using a
program for editing texts may be the same key that starts the
calculation of a mathematical expression in an arithmetic program. The
same action receives a different response according to the program,
which we assimilate to behavior. Therefore, the description of a
computer's behavior is the program that it is executing. In other words,
the ^{reaction} depends as much on the action as on the ^{state} of the
machine. The next state also depends as much on the action as on the
actual state. For example, the key that allows the text editor to go to
the upper case state for letters actually goes to the upper case state
if it was previously in the lower case state, but goes to the lower case
state if it was previously in the upper case state.

I will repeat the basics of the adaptor more precisely using the
terminology we have just introduced. The nervous system's task is to
discern, in sensations, what behavior, or program, is good for the
^{body} to carry out at each moment; in order to do this, all the
nervous system need do is determine which objects are present.
$$\hbox{Perception} \longrightarrow \hbox{Present Reality}
   \longrightarrow \hbox{Behavior}$$

For example, according to these definitions, ^[Brooks]'^(Brooks1999)
robots_{robot}, which connect perception to behavior, are adaptors.


\Section Nouns and Verbs

The ^{frog}'s perceptive apparatus, identifying here and now the objects
^{fly}, ^{cat}, and ^{water}, is providing the conditions_{condition}
that must be fulfilled in order to resolve the ^{problem} of which
^{behavior}, of all the possible ones of which the frog's ^{body} is
capable, is the right one at the moment. In this case, the right
behavior is to ^{flee}. The frog ignores anything that is not
^{present}, and even part of what is present; in our example, it ignores
the fly, because the cat's dangerous presence prevails over the fly's.
The action executed in this situation for flight may be jumping into the
water.

In order to simplify the explanation, we have up until now only paid
attention to one type of object, such as `cat', that corresponds with a
^{noun} and that we shall call a ^{nominal object}. But the adaptors'
nervous systems also use other types of objets, such as `to flee', that
correspond with verbs_{verb}. When the ^{verbal object} `to flee'
becomes present, the frog's nervous system emits a well synchronized
series of executive orders to different body muscles in order to carry
out the jump adequately.


\Section The Adaptor's Reality Is Objective

We can draw two conclusions about the adaptor's_{adaptor} ^{reality}:
\point What mediates between the ^{phenomenon} and the adaptors'
actions_{action} is a ^{network of objects} called reality that
simplifies the ^{sensation} captured by the senses.
\point The adaptor always lives in the ^{present} reality, that is to
say, it lives in the reality of the objects that are present.


\Section The Learner

The ^{network of objects} that constitutes ^{reality} can be fixed or
changing. We will give the name of ^{simple adaptor} to the adaptor with
a genetically fixed reality and that is capable of adapting to present
circumstances, but that cannot learn to get along in new situations
because it is incapable of modifying its network of objects.

A ^|learner|, on the other hand, is an adaptor capable of modifying
reality. That is, the learner's network of objects is flexible and can
be tuned to its external ^{environment}. We shall call this process by
which the learner fits its objective reality to external circumstances
^{modeling}.

Going back to the simple adaptor, this adaptor has a rigid network of
objects. It could be said that this adaptor's model of the exterior is
rigid, but with one qualifier. With a rigid network, it is not useful
for the network to go through the intermediate step of calculating the
forecast of the model in order to decide upon the behavior to be
executed. In this case, it is more efficient and rapid to have the
appropriate ^{behavior} predetermined for each present situation. Thus,
although there is a model, there is no ^{foresight}.

Given evolution's_{evolution} ^{opportunism}, the first learners must
have been very similar to the adaptors. They may have had a nervous
system similar to that of a simple adaptor, but with the possibility of
varying its network of objects slightly. This may have been sufficient
for the learner to learn to live in several slightly different
environments. In these first stages, the learner did not yet need to
calculate forecasts.

The learner's situation is different when the plasticity of its modeling
increases. The critical point is reached when the variability of the
models is such that it is no longer practical to use predetermined
behaviors for each possible network of objects with each possible
configuration of objects that are present. That is, when the number of
realities and present situations that the learner is capable of
producing passes a certain threshold, it becomes much less efficient to
have the response to each rigidly coded. This is when evolution obtains
an advantage if it designs learners that foresee the consequences in
order to decide which behavior to execute. These learners are capable of
internally simulating_{simulation} the result of behaving in different
ways in the present reality, and they only actually carry out the
behavior that is most favorable, according to their simulation. I would
remind you again that `^{present} reality' is shorthand for saying
`network of objects found by modeling, in which the objects that
perception determines are now present'. With simulations based on
present reality, the learner is entering into the ^{future}.


\Section Modeling

Modeling_|modeling| modifies the ^{network of objects}. If the network
of objects_{object} is made up of objects, which are the nodes in the
network, connected by stronger or weaker relations, which can be
positive, with an awakening effect, or negative, with a lulling effect,
then modeling can vary all of these elements. Modeling can create,
eliminate, reinforce, or weaken connections. It can even make them
change the direction of their effects, turning connections that have an
awakening effect into connections that have a lulling effect, and
\latin{vice versa}. It can also create, eliminate, unite, or separate
objects.

The purpose of modeling is that the resulting network of objects can
foresee_{foresight} exactly which reactions_{reaction} the external
^{environment} will have to the ^{learner}'s actions_{action}; in this
way the simulation will be precise. This transaction with the exterior
does not assume the ^{existence} of objects outside the learner's
nervous system. It is sufficient if the external environment's reaction
coincides with the reaction that the internal network of objects
anticipates. This conclusion corroborates one of the theses I uphold
here, that is, that objective ^{reality} is subjective.


\Section Simulation

Simulation_|simulation| can consist of internally closing a loop that,
without simulation, is closed in the exterior by the environment. That
is to say, the learner should be capable of making the ^{verbal object}
that is present go back to the network of objects that models present
reality, in order to foresee the reaction of the environment, instead of
provoking the direct execution of actions on the outside. Thus, the
actions are actually executed only when the prediction is favorable.
$$\hbox{Phenomenon}
   \underbrace{\strut
    \longrightarrow
    \onitself{\strut Object}
    \longrightarrow}_{\hbox{\strut Learner}}
  \hbox{Action}
$$

These learners with foresight need a much more computationally powerful
^{nervous system} than the adaptors. For the purposes of this
superficial description, we will say that this new simulation function
is carried out by a part of the nervous system called the ^{brain},
which is complex enough to do this.

Careful!  The fact that complex learners can internally simulate the
result of their actions does not mean that they always do so.
Evolution_{evolution} is opportunistic and makes its modifications upon
an existing basis; it doesn't design things from scratch. Within the
learner, then, predetermined behavior coexists with simulation.
Opportunism_{opportunism} is a constant pattern in Darwinian evolution
and, even when we do not make explicit reference to it, it must always
be taken into consideration.


\Section The Learner's Reality Changes

Two conclusions can be reached about the learner's reality:
\point The ^{learner} changes not only the ^{present} but
^{reality} itself. The objects of the learner's reality, besides making
themselves present or absent, can be modified, created, and eliminated.
\point The learner foresees_{foresight} the effects, that is, the learner
sees_{see} ^{future} reality.


\Section The Knower

The learner's_{learner} ^{network of objects} can be modified. If I call
the part of the ^{brain} in charge of modifying the network of objects
`^|mind|' (and whenever I write `network of objects' it can be read as
`^{reality}'), then we have, once again, two possibilities: the mind is
rigid or the mind can be modified. We can call the learner with a rigid
mind `^{simple learner}' and the learner with a flexible mind
`^|knower|'.

I would like to make one comment on ^{flexibility}. It may seem that the
more flexibility the better; taken to the extreme, it may seem that
complete flexibility would be best. This would not, however, be the
best, because the result would be shapeless, it would be chaos. At some
level, there must be a rigid layer that gives shape to the more flexible
layers. The workings of the most flexible of machines, the ^{computer},
can be enlightening. I am referring, in particular, to the most
interesting aspect of these machines, that is, how it is possible to
build a computer with logic gates. How can a computer that will do
anything we want be built with tremendously simple elements that always
do exactly the same thing? The answer awaits us in _>The Universal
Turing Machine> and, at a deeper level, right at the end of _>Logicism>.

Going back to the principal discussion, we saw that a simple learner has
a flexible network of processes to foresee_{foresight} events, but that
its other mental processes are rigid. The processes that determine how
to modify the network of objects in order to foresee events better are
rigid in the simple learner, as well as ^{perception} and the processes
that determine which ^{behavior} is adequate for the present reality. In
the simple learner, reality can be shaped, but the processes that use
reality are fixed and, consequently, they use it in one fixed way.

The knower, on the other hand, can use the network of objects in
different ways. One way is to use it in its totality, as does the simple
learner, but other ways use only parts of reality. In order to do this,
other mental processes of the knower must have dynamic access to the
network of objects. On another level, the presence of the objects must
be internally controllable, that is, the determination of which objects
are present should depend on perception, as in the learner, but also on
other mental processes. In the end, ^{attention} mechanisms reappear in
the knower.


\Section Sentiment

In order to carry out these requirements, there must be two kinds of
processes that determine which objects are present. Apart from
perception, inherited from the learners and originating in the
phenomenon, another path is added. This path is similar to an interior
perception because it permits the knower itself to make the objects that
are of interest to it present and to ignore the ones that are not of
interest. I repeat, if in the simple learner the presence of objects is
the result of the interaction between the ^{sensation} received from the
exterior and the learner's own network of objects, in the knower, other
additional mental processes intervene. We call these
sentiments_{sentiment} because they are mental sensations.
Feelings_{feeling} are sentiments.

Sentiments_|sentiment| are those states that determine how ^{reality}
should be used at each moment, and so modify the ^{present}. A
thirsty_{thirst} knower will use reality in a different way than a
satiated one. The problems of each, and therefore the solutions or
adequate behaviors, are different.

Objects acquire ^{meaning} when sentiments spread throughout the
^{network of objects}. I must admit that this definition of meaning is
obscure; you can postpone adherence to it, and treat it as a technical
term, until we see where it will take us. So let us continue. Sentiments
are primitive semantic terms. Meaning will explain why the knowers act
one way and not another, why one behavior does not have the same value
as another. This depends, in the end, on the sentiments, also known as
feelings.

The first meanings are sentiments, basically \meaning{^{pleasure}} and
\meaning{^{pain}}. All other feelings are derived from these.
Evolution is what determines what \meaning{pleasure} is and what
\meaning{pain} is. For evolution itself, the first meanings are
\meaning{^{life}} and \meaning{^{death}}. So \meaning{pleasure} and
\meaning{pain} are substitutes, discovered by evolution,
for \meaning{life} and \meaning{death}. But only we symbolic subjects
know this; simple knowers_{simple knower}, I mean those that are not
subjects, have no meaning for life and death because the primitive
meanings, pleasure and pain, are enough for them.

The actions that the ^{knower} associates with obtaining pleasure or
pain, and the verbal objects_{verbal object} that are present,
responsible for these actions_{action}, thus obtain a second meaning.
The processes that make these objects present, as well as the objects
that launched these processes, get a third meaning, and so on until this
wave, which updates meanings, reaches the perceived nominal
objects_{nominal object} that started the actions.

Thus, the meanings spread throughout the network of objects, so that all
of the knower's objects have, at every moment, meaning; this is why we
say that the knower's network of objects is a ^{semantic network}. To
sum up: the reality of the knower is semantic because it has meanings.
And these meanings, that color the objects of sentiment, are what direct
the knower's behavior towards life and away from death.


\Section Emotion

The process that associates meanings_{meaning} to signs_{sign} is very
general. When the ^{dog}, in ^[Pavlov]'s^(Pavlov1904) classic study of
^{conditioned reflex}, salivates upon hearing a bell, it is giving the
meaning \meaning{food} to the sound of the bell. \meaning{Food}, vital
for survival, will have a meaning, given directly by ^{evolution}, very
close to \meaning{pleasure}. Eating is one of life's pleasures. But the
situation is much more general. The knower needs to give everything it
sees, every object, a meaning. This is because sight neither feeds nor
kills. What I want to say is that if an ^{antilope} doesn't give the
meaning \meaning{dangerous lion} to certain spots that grow larger in
its field of vision, then the ^{lion} that gets close to it will kill
it. Nor would the lion run if it didn't give the meaning \meaning{tasty
antilope} to what it perceives.

This means that the antilope's ^{object} lion must be associated with
the meaning \meaning{dangerous}. Evolution has found it useful to add a
meaning to each object. We can also say the same thing if we say that
evolution perpetuated the species of knowers_{knower} that increases the
^{network of objects} with meanings that color the objects with
^{desire}, appetites, usefulnesses, uses, and needs that the ^{emotional
system}, which we will abbreviate as ^|emotion|, derives from
sentiments.
$$\hbox{Emotion} = \hbox{Emotional System}$$

Up to now, the processes described have begun with the ^{phenomenon},
from which ^{perception} extracts signs that make the objects
^{present}; these objects, in turn, set off the sequences of executive
orders that determine the ^{behavior}. A ^{process} beginning with a
^{sentiment} will help us to distinguish both entry paths to the
^{network of objects}. For example, a sharp sentiment of ^{thirst} will
find, in the network of objects, that it is necessary to launch the
search for signs of ^{water}. This is, without a doubt, related to
^{attention}, which, if there is nothing more urgent, will ignore any
objects whose ^{meaning} is not related to thirst.

Because the knowers use meanings, they can use the network of objects to
resolve concrete problems to which the emotional system gives priority,
such as slaking thirst. They have a double entry path to ^{reality}: one
path is the ^{perception} inherited from the learners and the adaptors
and the other is ^{emotion}, specific to knowers. This emotion is goal
oriented and modifies the present by paying attention to the most
pressing aspects at each moment.
$$\hbox{Phenomenon}
   \underbrace{
    \longrightarrow
    \column{
     \onitself{\strut Object}\cr
     $\uparrow$\cr
     \hidewidth\strut Sentiment\hidewidth\cr
     $\uparrow$\cr
     \strut Libido\cr}
    \longrightarrow}_{\strut\hbox{Knower}}
  \hbox{Action}
$$

The root of the ^{emotional system} is the ^{libido}, the incarnation of
the ^{survival instinct} in the ^{brain}. With the ever-present survival
instinct and with the propioceptive_{propioception} information that it
receives, emotion gives priority to sentiments at every moment. That is,
emotion directly uses information about the body to determine which
problem is the most urgent. The sentiment selected is a primary meaning
that, using the meanings spread throughout the network of objects, gets
the knower to attend to the objects whose meaning is of interest.


\Section Pain

Pain_{pain} requires all of the ^{attention}, thus proving that it is a
primary ^{sentiment}, that \meaning{pain} is pure ^{meaning}. A clear
proof is the measurement of one's own willpower_{will} against pain. If
a ^{migraine} is intense, it is impossible to think about anything but
one's own headache. It just happened to me yesterday. % [1999.12.16]

Nothing is worth much when the pain is great. Because, if the pain is
great, it absorbs all meaning and all else becomes meaningless.


\Section Arbitrary Signs

In the case of the adaptors_{adaptor}, meanings are
genetically_{genetics} associated with perceptions_{perception}, as in
the case of the ^{frog}, for which every small thing that moves quickly
is an edible ^{fly}; it is impossible to separate the ^{noun} (fly) from
the ^{adjective} (edible), the ^{object} from its meaning. So adaptors
do not distinguish objects from meanings; they are the same to them. But
other more complex species are capable of learning to give ^{meaning} to
new objects. For example, a ^{dog}, which, according to our
classification, is a ^{knower}, is capable of learning to distinguish
its favorite dog biscuits from other food.

The advantage that the dog has over the frog is that, for the dog, the
relationship betweeen the ^{sign} and its meaning is arbitrary, as
^[Pavlov]^(Pavlov1904) demonstrated. What I mean is that any object can,
in principle, have any meaning. Oranges taste very good, but they could
be poisonous. If they were, the same sign would have a different
meaning.

The species that have genetically determined meanings are born with this
knowledge, they have ^{inspired knowledge}. This is, without a doubt, an
advantage, but only as long as the ^{environment} does not change. If,
for example, a fruit similar to an ^{orange} but poisonous appeared in
their environment, they would die because they would fail to learn the
difference between an orange and a pseudo-orange. If the oranges and the
pseudo-oranges were indistinguishable, that is, if half of the oranges
suddenly became poisonous, the proper action would be not to eat
anything that looked like an orange. If the knower is capable of
changing the meaning of the object orange from \meaning{tasty} to
\meaning{poisonous}, it will reduce its risks.

The ^{behavior} of the knowers depends more on the meaning than on the
sign. Assigning meaning incorrectly to an object can be a fatal mistake.
This would, for example, be the case if the meaning of the orange is
assigned to the poisonous pseudo-orange. What we perceive is less
important than the meaning that we give to what we perceive.


\Section Adjectives and Adverbs

The knower's_{knower} ^{network of objects} receives data from the
exterior and from the interior, that is, from the phenomena_{phenomenon}
and from the sentiments. Both, ^{sensation} and ^{sentiment}, determine
^{present} ^{reality}; we have already seen this. Just widening the
entry path does not modify the ^{learning} mechanism. Therefore the
learning mechanism doesn't distinguish one from the other either, but it
takes both into consideration in order to modify reality, which will now
fit the exterior as well as the interior. Thus, simply taking advantage
of the mechanism of learning inherited from its learner predecessors,
the knower can learn that the indistinguishable oranges_{orange} and
pseudo-oranges are \meaning{poisonous} instead of \meaning{tasty}.

But that is not all. The learner modifies reality by creating, varying,
and eliminating the relationships between the different objects and also
by creating, modifying, and eliminating the objects themselves. Thus,
the learning that the knowers have is capable of creating objects from
external phenomena as well as from internal sentiments.

So then other objects_{object} corresponding to adjectives_{adjective}
and adverbs_{adverb} appear. An ^{adjectival object} such as
\meaning{poisonous} will be awakened by inedible nominal
objects_{nominal object} and will veto behaviors_{behavior} that would
cause their ingestion. Adverbial objects_{adverbial object} do the same
modifying or modulating verbal objects_{verbal object} that, when they
become present, launch behaviors in the knowers.


\Section Meaning

The ^{behavior} of the knowers_{knower} does not depend directly on
^{perception}, but on the meaning given to perception. The meaning, in
turn, depends on ^{emotion}, so that the immediate or primary ^{meaning}
is the ^{sentiment}. The other non-primary meanings are found by ^{back
propagation}: first, the verbal objects whose presence has launched the
actual behavior catch the pure meaning of the sentiment that they have
obtained; afterwards, the meaning of the present objects propagates
itself epidemically to those other objects whose presence had caused the
presence of the previous objects; and so on until the wave finally
reaches those nominal objects that perception itself had made present.
In this way, all of the knower's objects_{object} acquire meaning, and
their ^{reality} becomes a network of meanings, that is, a ^{semantic
network}.

\breakif1

For example: An ^{antilope} flees and manages to save its ^{life} from a
predator's attack. Upon saving itself, it will feel relieved,
\meaning{happy}_{happiness}, a primitive meaning. It saved itself by
fleeing_{flee} from the presence of a dangerous ^{lion}, and so the
connecting line between the objects `to flee', `dangerous', and `lion'
is reinforced so that the next time the antilope will once again be able
to follow the path from `lion' to `dangerous' and from `dangerous' to
`to flee'. Another way to put this is that, for the antilope after a
successful flight, the lion maintains its meaning of dangerous animal,
and dangerous maintains the meaning of something from which it is
preferible to flee. On the contrary, a behavior that results in failure,
\meaning{painful}_{pain}, will weaken the existing relationships between
the objects that cause the behavior, even to the point of changing their
meaning.
$$\hbox{Perception}\longrightarrow
  \column{\strut Reality\cr
               $\uparrow$\cr
  \hidewidth\strut Emotion\hidewidth\cr}
  \longrightarrow\hbox{Behavior}$$

Thanks to meaning, reality can be used simultaneously by ^{perception},
which brings data from the exterior, and by ^{emotion}, which uses
internal information. Meaning_|meaning| puts external conditions
together with internal conditions.
$$\hbox{Conditions}
  \left\{ \vcenter{\nointerlineskip\halign{#\hfil\crcr
   Internal\cr
   $\quad\;\bigm\updownarrow\hbox{Meaning}$\cr
   External\cr
  }}\right.
$$


\Section  The Knower's Reality Is Semantic

We can state two conclusions about the knower's reality:
\point The knower's_{knower} ^{reality} is ^{semantic}, it has ^{meaning}.
The meaning orders the access to reality that ^{perception} of the
exterior and internal ^{emotion} dispute.
\point The knower's reality is situated_{situation}, it is centered,
that is, it is spatial_{space}, because it distinguishes between
internal and external: inside and outside, here and there.


\Section Words

Let us suppose that an ancient prehistoric predecessor had a nervous
tick that made him exclaim `leo' when he recognized a ^{lion}. The other
members of his tribe would eventually learn that the sound `leo',
pronounced by this tribal comrade, was an indication that a lion was
present nearby. This learning is in no way different from that studied
by ^[Pavlov]^(Pavlov1904). Any ^{sign} that helps to identify an
^{object}, is associated with that object. Our ability to recognize
objects in phenomena improves with experience because of this procedure
that we have acquired through evolution.

Another inherited behavior, that we share with monkeys, is ^{imitation},
especially in infancy. This imitation allows us to avoid unfruitful
attempts and go directly to trying the solutions that our elders'
experience has proven useful. So we can suppose that, in the following
generation, all of the tribal members said `leo' upon identifying a
lion. The advantage for the tribe was that it was enough for one of the
members to see a lion so that all of them could perceive it without
needing to capture its image on their retinas, that is, without needing
to sense it visually.

This is how the word `leo' came to mean \meaning{lion} in this tribe. It
was just chance that it was this specific word, it could have been any
other, since the mechanism for associating signs with objects has only a
utilitarian requirement; that is, it allows the association of any
^{name} with any object, as long as it is a useful association, as we
saw in ^>Arbitrary Signs>.
$$
 \hbox{Phenomenon}
 \underbrace{\longrightarrow
  \column{
   \onitself{\strut Object$'$}\cr
   $\uparrow$\cr
   \hidewidth\hbox{\strut Sentiment$'$}\hidewidth\cr
   $\uparrow$\cr
   \hbox{\strut Libido$'$}\cr}
  \longrightarrow}_{\strut\hbox{Knower$'$}}
 \hbox{Word}
 \underbrace{\longrightarrow
  \column{
   \onitself{\strut Object}\cr
   $\uparrow$\cr
   \hidewidth\hbox{\strut Sentiment}\hidewidth\cr
   $\uparrow$\cr
   \hbox{\strut Libido}\cr}
  \longrightarrow}_{\strut\hbox{Knower}}
 \hbox{Action}
$$

For the moment, for our tribe, the word `leo' is only a sign. But it is
a peculiar sign because whoever pronounces the word mediates between the
^{phenomenon} and the knower who interprets the sign. It is in the
ability to take advantage of this new situation that the origins of man,
the only living ^{subject}, are to be found. In the following sections,
we will develop the process that, through words, takes us from the mute
knower to the symbolic subject.


\Section Signic Language

A single word such as `leo' accompanied by a finger indicating a
direction would be sufficient to cause the entire group to ^{flee} in
the other direction. This use of the spoken word as a ^{sign} is useful
enough to have a selective value in the ^{evolution} of our species. It
is not yet ^{symbolic language}. The ^{word}, in this preliminary state
of language, is a sign, which means that for any speaker of this
^{signic language}, the word is another property or indication of the
object. Thus, the word `leo' is taken into consideration in the same way
as the lion's_{lion} ^{color}, aspect, or ^{odor}.

Please forgive me for introducing this new word, `signic'. My aim is to
differentiate `signic language' from `sign language', which is a
well-known kind of gestural language that is not, according to our
definition of sign, signic. For example, American Sign Language is a
symbolic language.

^[Vygotsky]^(Vygotsky1934) observed that, for children, names are
attributes or properties of things, just like their color, not
conventions; this is easy to check. And it's fun, too!  The language
that small children speak is, then, an example of a signic language;
that is, it is not symbolic. Transferring this ontogenetic proof, we
obtain the corresponding phylogenetic proof; in other words, if each
person as a child passes through this stage, it is permissible to
suppose that humankind as a species also passed through the stage.

In a signic language, names can be given to objects that can be, as we
have seen, nominal_{nominal object}, adjectival_{adjectival object},
verbal_{verbal object}, and adverbial_{adverbial object} objects. The
word is a sign that makes the object to which it refers ^{present}.
Summarizing, signic language only adds another attribute, the ^{name},
to entities provided by other cognitive processes. The word is always a
reference to a given, to something external to it.

The limitations of a signic language are evident. Just take any sentence
in this book, this very sentence for example, to discover what is beyond
its expressive reach, because ^{self-reference} is impossible in a
signic language. Questions are also impossible in a signic language.


\Section Truth

The ^{word} is a peculiar ^{sign} because it permits a ^{knower} to
mediate between the ^{phenomenon} and another knower.  And so it could
happen that, in the tribe in which `leo' meant \meaning{lion}, someone
could say `leo'_{lion}, supposedly by mistake. Since hearing the sound
of the word `leo' was, for the other members of the tribe, another way
of detecting a lion's presence, the result was that, for them, the lion
was ^{present}.

To begin with, this mistake doesn't seem to be advantageous. It creates
a new ^{world} in which the lion is present here and now even though, in
^{truth}, it isn't. The important thing to realize is that, with words,
a knower has the possibility of directly influencing the ^{reality} of
another knower. We can rest assured that ^{evolution} soon took
advantage of this extraordinary power.

The word, being a mediated property, works to bring the ^{object} into
the present, the here and now, even if it isn't actually present or,
being meticulous, even if only one of the object's signs, its ^{name},
is present. The use of the word as a sign has the power of making
present that which, without the word, is not present. The word goes
beyond the ^{attention}, which modifies the present; the word invents
the present. And with this invention, truth and falsehood appear.
Without words, there are no lies_{lie}.


\Section Communication

Communication_{communication} between knowers is based on the
intromission of a ^{knower}, by way of a ^{word}, in the perceptive
process of another knower. But to take advantage of this situation, one
additional revolution was needed; we will call it the ^{proto-subject}
revolution. I believe, however, that once mediation was achieved in
^{perception}, it was only a question of time before ^{evolution} found
a way to exploit this, it being enough for the proto-subject to dispense
with the source of perception, that is, the ^{phenomenon}.
$$\underbrace{
 \phantom{\longrightarrow}
 \column{
  \onitself{\strut Object$'$}\cr
  $\uparrow$\cr
  \hidewidth\strut Sentiment$'$\hidewidth\cr
  $\uparrow$\cr
  \strut Libido$'$\cr}
 \longrightarrow}_{\strut\hbox{Proto-Subject$'$}}
  \hbox{Word}
  \underbrace{\longrightarrow
   \column{
    \onitself{\strut Object}\cr
    $\uparrow$\cr
    \hidewidth\strut Sentiment\hidewidth\cr
    $\uparrow$\cr
    \strut Libido\cr}
  \longrightarrow }_{\strut\hbox{Proto-Subject}}
  \hbox{Action}
$$

Once this was achieved, if the speaker proto-subject gains the
^{attention} of the listener, then the ^{action} that the listener
carries out has its origin in the libido of the speaker. The other
proto-subject's ^{behavior} can be controlled with words.

The word has the same place in communication as the phenomenon does in
perception. The difference between perception and communication is that,
in perception, the phenomenon is the origin of the data, while, in
communication, the origin is not the word but the ^{object}. The fact
that only we subjectivists observe this difference proves that the
simplification of applying the communication model to perception has had
enormous success historically. Objectivism_{objectivism} postulates
that, in perception as well, external objects exist that are the causes
of the perceived phenomena. This hypothesis is unnecessary, so its
effect can only be distorting.


\Section Symbols

The ^{word} goes on to carry out two functions: the original one of
serving as a ^{sign} of an external phenomenon, even if it is a sign
obtained by someone else's mediation, and another new function by which
the word refers to an object belonging to the speaker who pronounces it,
where it is not a sign of an external phenomenon, but of an interior
object belonging to the speaker. In the first case, we will say that the
word is a sign, and in the second case, that it is a ^|symbol|. In these
conditions, it is already in the interest of the proto-subject to
distinguish these uses of the word, so that in one case he could say
`There's ^{water}' and in another, `Want water'.

What is revolutionary about this is that, with the word, the object,
from being a mere mental construct for one's own use, achieves an
external character of shared usage, and, especially, it turns into part
of a widened ^{reality}. But this requires more explanation.


\Section Self-Absorption

As we have seen, a ^{proto-subject} can control the ^{behavior} of
another proto-subject through words; if one talks to oneself, then, one
can control one's own behavior. This might seem superfluous at first,
since the proto-subject already controlled its own behavior (who else if
not the proto-subject itself?). However, in the proto-subject, as in the
^{knower}, behavior depends as much on ^{perception} as on ^{emotion}.
By talking to oneself, and since the words pronounced out loud are heard
and listened to, the ^{emotional system} manages to effectively occupy
perception. With both entry paths to ^{reality} monopolized, the
^{libido} completely dominates ^{cognition}.

So the next step was to use the spoken ^{word} to listen to oneself.
This way the new ^{subject} achieved the same control over himself that
the proto-subject had achieved over others.
$$
 \vbox{\halign{\hfil#\hfil\cr\hfill$\swarrow$\cr Word\cr}}
 \underbrace{\longrightarrow
  \column{
   \onitself{\strut Object}\cr
   $\uparrow$\cr
   \hidewidth\hbox{\strut Sentiment}\hidewidth\cr
   $\uparrow$\cr
   \hbox{\strut Libido}\cr}
  \longrightarrow}_{\strut\hbox{Subject}}
 \vbox{\halign{\hfil#\hfil\cr$\nwarrow$\hfill\cr Word\cr}}
$$
By paying ^{attention} to what one says to oneself, the ^{sentiment}
generated by the libido completely dominates cognition. For example, a
child hurts itself and cries, but it doesn't see its mother anywhere, so
after a while it calms down and continues playing. When the mother
returns, the child pronounces the word `^{owie}' and starts to cry
disconsolately again. Paying attention to the word `owie', perception
supports emotion, which thus completely occupies the child's cognition.

We will call this situation in which the ^{subject} has broken the
connection with the exterior ^{self-absorption}. This is a conversation
with oneself that serves to achieve complete domination of oneself. In
other words, while the ^{simple knower} is necessarily hooked into the
^{exterior}, the subject can, thanks to the ^{word}, unhook itself from
the exterior.


\setbox0=\hbox{$\underbrace{\phantom{\longrightarrow}%
   \subjectdia
   \phantom{\longrightarrow}}_{\hbox{Subject}}$}
\dimen0=\hsize \advance\dimen0 by -\wd0 \advance\dimen0 by -1em
\nointerlineskip
\hbox to0pt{\kern\dimen0\kern1em\vbox to0pt{\kern2pc\box0\vss}\hss}
\nointerlineskip

\Section Thought

\parshape1 0pt \dimen0
Soon the word was interiorized. We will call the interiorized word
^{idea}. We will call the loop that goes from object to object through
the idea ^|thought|, or ^|reflection|. To ^{imagine} is to go from the
object to the idea, and to ^{conceive} is to return from the idea to the
object. If communication builds new worlds_{world} and different
present_{present} times in the brains of others, thought does the same
in one's own ^{brain}.

\breakif3

Thought is mute, interiorized speech. This agrees with the findings of
^[Vygotsky]^(Vygotsky1934), who observed that, in children, speech is
previous to symbolic reasoning.
$$\hbox{Thought} = \hbox{Mute Speech}$$

We only have one ^{mouth}, that is, we can only say one thing at a time.
Besides, it is convenient to have a single process that directs all
other processes; when this does not occur, as in the case of
^{schizofrenia}, the ^{subject} tends to behave paradoxically. This
explains why thought is sequential, even though it takes place over a
maze of simultaneous, or parallel, cognitive processes.


\Section The Will

Subjects filter the data they receive three times, because each entry
path into the network of objects that we have called ^{reality} selects
one part.
$$\hbox{Phenomenon}
  \underbrace{\longrightarrow\subjectdia
   \longrightarrow}_{\hbox{Subject}}
  \hbox{Action}
\abovedisplayskip=\abovedisplayshortskip
\belowdisplayskip=\belowdisplayshortskip
$$

Perception_{perception} is the oldest filter, inherited from the
adaptors, and it only catches the objects that are present. Besides, the
^{emotional system}, or ^{emotion}, inherited from the knowers, only
attends to the objects that are present that are of interest to the
subject, the significant objects. And ^{thought}, characteristic of
subjects, has a path by which the ideas also condition ^{reality}. This
last path is called the ^|will|.

Then there is ^{learning}, which provides not a filter for reality, but
the possibility of redefining one's own reality. So that if, in the
short term, the three entry paths into reality filter or select objects,
in the long term, and thanks to learning, they contribute to modifying
the ^{network of objects}, creating, destroying, uniting, separating,
strengthening, or weakening the objects and the relationships between
the objects.

Of all of the entry paths, thought took on the role of behavior
controller at the highest level because it is the way the
subject's_{subject} ^{libido} found of occupying its own perception in
order to broaden ^{reality} at will. If speaking works to control
another subject, thought and will work for ^{self-control}.

When any of the four faculties that control the network of objects
---perception, learning, emotion, or thought---undergoes atrophy or
hypertrophy, dysfunctional behavior occurs. Pathological psychology
studies these cases.


\Section Consciousness

Words can be spoken and heard. This is why ^{thought} goes from the
^{object} to the object and is recursive right from the start; that is
also why it is called reflection. This ^{recursivity} makes
^{introspection} possible. Let us see how.

Since signs allow us to recognize an object in a ^{phenomenon}, when the
^{word} is a ^{symbol}, that is, when the word is a ^{sign} of an
interior object, then what it does is recognize an object in another
object. And because recognizing objects in phenomena is the way,
designed by evolution, of seeing_{see} the external phenomena or, in
general, of perceiving them, when we recognize objects in objects, we
subjects are seeing the internal objects. The difference between sensing
and perceiving, or seeing, should be very precisely observed here;
remember what was said in ^>Objective Reality Is Subjective>, and in
^>The Adaptor>. The objects that we see are the ones that are present,
so that we subjects see ^{present} ^{reality}. What is remarkable is not
this, however, but that the simple knowers_{knower} do not see present
reality. What is surprising is that only we subjects_{subject} see
present reality. Simple knowers are in present reality, but they do not
see it.

This is as true for the ^{idea}, which is a mute word, as for the spoken
word; if we see our own objects with ideas, we see another's objects
with the spoken word. The spoken word can, however, be a ^{lie}. This
happens when the subject discovers that it is in his interest for the
listener to believe that the subject thinks something he does not
actually think. Self-deception_{self-deception} is less frequent but
more dangerous; in spite of its enormous interest, we will not study it
here.

So if the adaptors_{adaptor}, the learners_{learner}, and the
knowers_{knower} have eyeglasses that add labels_{label} to what their
senses capture, and this is seeing, we subjects can, in addition, see
the labels because we can label the labels. Words_{word} are labels for
objects, and objects_{object} are labels for phenomena. Thus, we can see
our own thoughts. This does not mean that we can see all of our
cognitive processes, but the ones that we do see are what we call our
^|consciousness|.

Just as an ^{eye} can sense itself in a ^{mirror}, so an object can see
itself reflected in thought. Using this analogy, by means of which we
have already made `thought' synonymous with `reflection', we can give
the following definition with precision, and with ^[Plato]'s permission:
an ^{idea} is a ^{virtual object}.
$$\hbox{Idea} = \hbox{Virtual Object}$$


\Section The Unconscious

Consciousness_{consciousness} is that part of ^{cognition} that we can
^{see} thanks to its ^{symbolization}, but there is another part that we
cannot see, as ^[Freud]^(Freud1900) discovered. We cannot see what
happens before objects_{object} are constructed. Nor can we see what
happens before sentiments_{sentiment} are formed. There are certainly
many other cognitive processes that we do not see, either, given the
late date of the appearance of symbolization in Darwinian ^{evolution}.

Why don't we remember the first year of our lives?  Because we can only
consciously remember what has been symbolized, that is, what has been
spoken, thought, seen, but never what has been sensed but not seen. We
cannot remember what has been captured but not labeled, perhaps because
it is not possible to bring it back into consciousness without its
^{label}. Conscience and ^{symbolization} are two sides of the same
coin.


\Section Things and Concepts

In the ^{subject}, ideas_{idea} permit the ^{reflection} that, taken as
data, broadens the entry and exit paths from the ^{network of objects}.
The subject's network of objects can produce behaviors such as
^{speech}, and also thought, that is, mute speech. Moreover, ideas, just
like phenomena and sentiments, determine which objects are present and
modify the network of objects. Thus, the network of objects is the
subjective center of two loops: the new ^{theoretical loop}, that we
denominate reflection or thought; and the old ^{practical loop}, that,
ever since the ^{adaptor}, goes through action, the exterior
environment, and the phenomenon.

The word that is heard is a ^{phenomenon}, and as such uses the
mechanisms of ^{perception}. The spoken word is no different from the
other actions that are executed as a result of cognitive processes. We
can assume that interiorized words, ideas, use the same channels as
spoken words. The first point in favor of this hypothesis is that
^{thought} is very recent, in the evolutionary sense, and has not had a
chance to diverge from speech. The second point is empirical. When
someone speaks to a person who is thinking about something unrelated to
what this person hears, the hearer doesn't understand what is being
said. The typical excuse, ``Sorry, I was thinking about something
else'', shows that words that are heard and thoughts have one sole entry
path into the subject's ^{present}.

Therefore, although the theoretical loop and the practical loop are
different, they receive the same treatment by the network of object's
^{modeling} processes, that is, by reality's modeling processes. This
^{opportunism}, typical of Darwinian ^{evolution}, allows thought to
broaden ^{reality}. That is, the same ^{learning} processes that
determine that \meaning{oak} should be a different object than
\meaning{tree}, also establish that \meaning{prime} should be a
different object than \meaning{number}.

Of all of the subject's objects_{object}, some are constructed by
^{perception} and by ^{learning} from phenomena that are experienced
practically. These are the nominal objects_{nominal object}. Among these
primitive objects, already in use by the adaptors, we have the verbal
objects_{verbal object}, that control ^{behavior}. With the appearance
of the emotional system of the knowers, nominal and verbal objects
acquired ^{meaning} and so the adjectival objects_{adjectival object}
and adverbial objects_{adverbial object} appeared, owing their existence
to sentiments. With the subject's thoughts, all of these previous
nominal, verbal, adjectival, and adverbial objects, that already had
meaning, acquired conceptual weight. In addition, other objects whose
existence is the exclusive product of the thought's ideas appeared. We
shall call the nominal, verbal, adjectival, and adverbial objects
things_|thing|; the others, which are theoretic objects_{theoretic
object}, we shall call concepts_|concept|.
$$\hbox{Object}\cases{\hbox{Concept}\cr\hbox{Thing}}$$


\Section The World

Concepts are peculiar objects because their existence is based on the
objects_{object} themselves. Nominal objects_{nominal object} owe their
existence to ^{perception}, verbal objects_{verbal object} to
^{behavior}, adjectival objects_{adjectival object} to perception and
^{emotion}, and adverbial objects_{adverbial object} to emotion and
behavior. But concepts exist because of the ideas_{idea} of thought,
which are merely labels for objects, so conceptual objects owe their
existence to the objects. While things_{thing} are given to us whether
we want them or not, concepts_{concept} are, just as their name says,
conceived at ^{will}.

For example, according to these definitions, ^{stone} is a thing, as is
rough, to ^{flee}, and now, although the words `stone', `rough', `flee',
and `now' are concepts, as we will explain in the next section,
_>Existence and Reference>.

To establish the difference between real things, which are given to us,
and theoretic concepts, which we can use to broaden reality at will, we
shall distinguish between ^{theory} and ^{reality}. World_|world| is the
name we give to the broadened reality that includes things, that is, the
objects that conform reality as such, as well as concepts, which are the
theoretic objects that thought conceives.
$$\hbox{World}\cases{\hbox{Theory}\cr\hbox{Reality}}$$

When we make this distinction, we lose the strict equivalence,
established in ^>Reality>, between the network of objects and reality.
The subject's network of objects is the same as his world. And since the
subject's world includes reality as such, reality turns out to be only
one part of the subject's network of objects.
$$\hbox{World} \left\{ \vcenter{\def\:{\hskip0.65em}\halign{#\cr
    Theory\:\hfil Concepts\cr
    Reality\:\hfil Things\cr} } \right\}
  \hbox{Network of Objects}$$

The ^{theoretical loop}, even though it is nothing more than a third
entry path to and a second exit from the ^{network of objects},
complicates reality enormously, and broadens it. The theoretic loop acts
as a ^{mirror} that reflects objects back onto the very objects. With
this help, the apparatus destined to see, sees itself.
$$\hbox{Perception}\longrightarrow
   \vbox{\everycr={}\tabskip=0pt \lineskip=0pt
    \halign{\hfil\strut#\hfil\cr
     \hidewidth Thought\hidewidth\cr
     $\downarrow$\hfil$\uparrow$\cr
     \column{\strut World\cr$\uparrow$\cr
      \hidewidth\strut Emotion\hidewidth\cr}\cr}}
  \longrightarrow\hbox{Behavior}$$


\Section Existence and Reference

When the ^{network of objects} is broadened, new objects obtain
^{meaning} by the process of back propagation we already saw in
^>Sentiment>, and in ^>Meaning>; this process goes from sentiments to
behaviors and from behaviors to objects. This is a general process and
it also works if the new object is a ^{concept}. However, something that
is impossible with things can happen with concepts: there may be no way
to reach the concept from the sentiments. The reason is simple, there
being concepts that do not provoke any behavior, but only reflection.
These purely theoretic concepts that do not give rise to actions cannot
cause ^{pain}, or ^{pleasure}, and for this reason they have no meaning.

The matter becomes more complicated because reflection allows us to see
our own network of objects. In this way, it is possible for the word
`^{water}', which was, at the start, only one of the signs of the object
water, to become an object in itself. Because this is possible, we can,
for example, see that the word `water' has two syllables. In order to
note the difference between the ^{object} water, which is a thing, and
the object that is the ^{word} `water', which is a concept, we will put
the word `water' in quotation marks, and refer to the thing that is
water as the thing water. The thing water gets things wet, but the word
`water' has two syllables and doesn't get things wet!

Under these circumstances, we say that the word `water'
refers_{reference} to the thing water, or that the word `water' takes
the meaning of the thing water, or to summarize, that water
exists_{existence}. Both objects, the thing water and the word `water',
are very closely related in the network of objects; if we hear the word
`water' we bring the thing water directly into the present, and if we
perceive signs that make the thing water present, we can immediately
pronounce the word `water'. Both objects become present simultaneously.

Up to this point, reference is a binary operation, because it uses two
objects, one of which is an objectified word. But this is not true, in
general, for symbolic language. Each word is, in effect, a conceptual
object, that is, a concept, but it is not true that each concept
necessarily refers to a thing. Later, in ^>Paradoxes>, we shall speak
about words without referent or meaning.


\Section A First Hint of Freedom

By being able to objectify the ^{word}, which was a ^{sign}, we free
ourselves, quite literally, from the ^{present}, from the here and now;
we saw this in ^>Self-Absorption>. This ^{freedom} cannot be in the
finished ^{past}, or in the present, which is nothing but a limit, but
it is not the same as the ^{future}, either. Objects can contain
predictive_{foresight} models, like ^[Pavlov]'s dog's_{dog} bell that
announces the imminent, but future, arrival of ^{food}. The future is a
necessary condition for freedom, but it is not sufficient.

On the other hand, the word made symbol is a sufficient condition for
freedom. This is because there is no ^{freedom} unless there are various
possible worlds, and the word, by interfering between the ^{phenomenon}
and ^{reality}, constructs these worlds_{world} (see ^>Truth>, and ^>The
World>; and it wouldn't hurt to read ^[Goodman]^(Goodman1978)).


\Section Symbolic Language

The ^{word} was a ^{sign} first and a ^{symbol} afterwards. This
corresponds with the two uses of the word, that is, as a sign of an
exterior ^{phenomenon} and as the symbol of an interior ^{object}. This
is why we distinguish two stages in the development of language: the
^{signic language} that the tribe used where the word `leo' was one more
sign of the ^{lion} (as seen in ^>Words>), and the ^{symbolic language}
in which there are words, such as the word `verb', that are signs of
words themselves.

The difference between a signic language and a symbolic language is that
the ^{reference} for the words in the second of these has no limitation.
Specifically, a symbolic word can refer to another word; more
importantly, a word can have no referent whatsoever.

According to ^[Vygotsky]^(Vygotsky1934), the process of intellectual
maturity in children, that concludes around the age of twelve, consists
of symbolizing everything, even the very process of symbolization.
Translated to our terminology, and if ^[Vygotsky] is right, symbolic
language establishes itself at around the age of twelve.

Surpassing the stage of signic language also allows words to indicate
the uses of the object that go beyond their referential value. For
example, it is not the same to affirm that ^{water} is present, `There
is water', as it is to express the ^{desire} that water be present, `I
want water', or to ask where water might be found, `Where is there
water?' Only in the first case is the word `water' used as a sign; in
the others, it is used as a symbol. To distinguish which use is being
made of the word, subjects need to add other words or modifiers to the
word that refers to water. Some of these words, such as the word `where'
have no referent; we shall say that they have a purely syntactic value.

Syntax_{syntax} is what distinguishes symbolic language from signic
language. Its origin, leaving the reference without limits, is not very
spectacular, but it is enormously important. For example, since a
symbolic word, or ^{symbol}, can refer to another word, the word itself
becomes an ^{object}. As such, the symbol admits different signs_{sign}
to identify it. That is why the symbolic word can be written, not only
spoken.


\Section Writing

For the symbolic ^{subject}, the word is a conceptual object. Any object
can be recognized by different signs; so the subject's word can be, too.
This is why the subject can use spoken or written words, or even words
captured by touch, as ^[Braille] has demonstrated.

The ^{written word} has a characteristic that the ^{spoken word} does
not: it is lasting. That is why it can be used to direct ^{attention}
for a longer time. ^[Socrates], who has been dead for two thousand five
hundred years, still gets our attention because ^[Plato] wrote down what
his teacher said.


\Section Sentences

In a ^{signic language}, pronouncing a ^{word} was sufficient to say
that the ^{object} the word referred to was ^{present}, because the
word's only use was to serve as a ^{sign}. But when ^{speech} began to
transmit other aspects of ^{cognition}, its ^{expressiveness} grew, at
the expense of the growth of the speech unit, which up to that moment
had been the word and then became the sentence.

The ^|sentence| is the unit of speech, and it consists of a sequence of
words. The condition of the sentence being a sequence is due to the
limitation resulting from the inability to emit more than one vocal
sound at a time. Intonation_{intonation} can be used to differentiate
various uses of the same word; ^{Chinese}, for example, uses intonation
profusely; but this method has its limitations. That is why the sentence
uses words, modified or not, in the form of intoned sequences to try to
express the subject's cognitive state.

Because the sentence is a speech unit, as a rule, one ^{concept} is
expressed with at least one sentence. But the concept can became an
^{idea} and, after ^{reflection}, it can be named with a single word
whose ^{definition} is the concept that it names_{name}.

Thus, the word `water' that, in the ^{signic language} served to say
that there was ^{water} present, becomes the sentence `There is water',
in ^{symbolic language}, to differentiate it from `I want water', that
expresses that my emotional system has determined that I am
thirsty_{thirst} and calculates that the thing water would solve the
^{problem}. An imperative sentence, such as `Bring me water', would
solve the problem definitively if it managed to convince my
interlocutor, although this person might say `There might be water',
expressing his doubts as to a happy ^{solution} to my problem. If
instead of asking directly for a solution, I ask for help to
resolve_{resolution} my problem, then I should express it explicitly
with an interrogative sentence which could, in this case, be `Where is
there water?'  When I finally find water, I could exclaim `Water!',
because I have solved the problem. Exclamative sentences are vestiges of
the ancestral signic language.


\Section Syntax

In the previous sentences, the ^{word} `water' is the only word that
refers_{reference} to a ^{thing}; the other words or phrases, such as
`there is', `might be', or `where' are concepts_{concept} that do not
refer to anything. Besides words, written sentences use some signs, such
as question marks_{interrogation} or exclamation marks_{exclamation},
that make a note of the special ^{intonation} used for speaking the
words. Last of all, words are ordered in the sentence according to their
classifications, such as verb or noun, because, at times, the order
serves to distinguish between different uses of the words. Everything
related to the sentence, as a sentence, is called ^|syntax|, and it is
different for every ^{language}.

The ^{sentence} expresses, in part, the subject's_{subject} cognitive
state. But while the subject's cognitive state is made up of several
processes that act simultaneously, or in a ^{parallel} manner (see
^>Reality>), the sentence is a single sequence, or ^{series}, of words.
And even the words themselves are constructed by pronouncing sounds
sequentially. So imagining_{imagine}, as defined in ^>Thought>, is
basically a serializing or sequencing process, while
conceiving_{conceive}, the complementary process, works to make objects
parallel or concurrent. We say that the ^{syntax engine} carries out
these two processes.


\Section Problems

Symbols_{symbol} are freed signs_{sign}. Freeing a sign from exterior
and interior ^{perception} allows us to enunciate problems and
resolutions. This is an unexpected consequence, as all of the
discoveries of evolution_{opportunism} usually are. But it has proven
valuable (for the moment) for survival. Symbolic language can express
problems, such as `How can I eat a ^{nut}?', and resolutions such as
`Hit it with a stone until it breaks open'. It also lets us express
solutions, but this is no novelty, because solutions such as
`Flee!'_{flee} can also be communicated in a signic language. Some bird
calls that make the whole flock take flight also express, `Flee!' (see
^[Lorenz]^(Lorenz1949)). Syntax is not necessary in order to express
solutions, as this example shows.

We have distinguished the resolution of a ^{problem} from its solution.
Resolution_{resolution} is the process that permits the ^{solution} of
the problem. If the problem is how to eat a nut, hitting it with the
stone until it breaks open is the resolution, while the open nut that
you eat is the solution.

But why is symbolic language capable of expressing problems?  With
^{symbolic language} we ^{see} our conscious thoughts_{thought}; this is
part of the cognitive process, and the brain's_{brain} purpose is to
solve problems. We must remember that the purpose of the ^{nervous
system} is to determine, given what it perceives and its own state,
which bodily behavior is currently proper (see ^>The Adaptor>). Thus,
the reason that symbolic languages allow us to express problems,
solutions, and resolutions is that, through symbolic language, we can
partially see the brain's machinery for setting up and resolving
problems.

The semantic ^{reality} inherited from the knower fits in the subject's
symbolic ^{world}. But there is also room for many other concepts:
problems, desires_{desire}, doubts, questions, and resolutions, tools,
algorithms, plans, and solutions, behaviors, processes, actions. These
and none other are the ingredients from which the theoretic part of the
world is made (see ^>The World>), because problems, resolutions, and
solutions can be expressed in symbolic language.

In a symbolic language, questions such as `Why will I die?' or `What is
^{life}?' can be expounded, questions that do not exist outside of the
world of syntax. Freedom_{freedom} doesn't exist outside of this world
of syntax, either.


\Section Pronouns

In order to express a ^{problem}, one has to be able to express its two
components: ^{freedom} and ^{condition}, as we will see in _>The Theory
of the Problem>. The primary conditions are given by perception,
behavior, and emotion. Perception_{perception} presents the external
conditions, that is, the state of the universe, and ^{emotion}
determines the internal conditions, that is, needs and desires_{desire}.
The other condition is that the ^{behavior} the body will carry out to
solve a problem must be a behavior that is possible.

Symbolic language_{symbolic language} uses empty words, with no
referent_{reference} or ^{meaning}, to express ^{freedom}. That is how
it has to be if these words need to represent the freedom of the
problem. In the interrogative sentence `What should be done?', the word
`what' is a ^{pronoun} that doesn't refer to any specific behavior and
therefore has no meaning. It is necessary for it not to refer to
anything, or there would be no way to express the problem, which
consists precisely in that what should be done is unknown.

The fact that the word `I' is a pronoun means that it is used to mark
the freedom of a problem. The freedom of the ^{problem of the subject}
is expressed in the word `^{I}'. The name for the pronoun `I' is
`^{self}'.


\Section Articles

In English, the ^{article} specifies the ^{noun}, that is, it expresses
whether it is a definite noun or if it should be treated almost like a
^{pronoun}. Thus, the expression `a stone' tells us that the reference
is indefinite, although it is less indefinite than if we used an
interrogative pronoun such as `what'.


\Section Grammar

There are various types of sentences_{sentence, types} that are
different because they originated in different evolutionary moments. In
the first place, as we have seen (in ^>Problems>), there are exclamative
sentences that we subjects have inherited from the knowers. The results
of perception are expressed with enunciative sentences, or statements,
that describe the state of things. For referring directly to behavior,
one should use imperative sentences. Feelings use desiderative sentences
to express desires_{desire}, or statements to suggest that the needs are
imposed on the subject as if from outside himself. Finally, we have two
types of sentence that are specific to the subject: the dubitative
sentence, that reflects distancing between the subject's ^{thought} and
his ^{reality}, and the interrogative sentence, which is the kind that
best expresses the inquisitive nature of reflection.

The different types of words are also related to cognitive evolution.
Nouns_{noun} come from nominal objects_{nominal object} and verbs_{verb}
from verbal objects_{verbal object}, that have their distant origin in
the adaptors; adjectives_{adjective} and adverbs_{adverb} come from the
adjectival objects_{adjectival object} and adverbial objects_{adverbial
object} of the knowers. Pronouns_{pronoun} and articles_{article} appear
when the subject wants to express problems. Other types of words serve
to shape the sentence itself; in English, these are the
conjunctions_{conjunction} and prepositions_{preposition}, that try to
express the ^{concurrence} of reality and of the ^{world}, something
that sequential speech cannot do without these devices.

Given the necessarily recursive_{recursivity} nature of reflection,
syntax is also recursive. This makes it possible for a sentence to
contain other sentences, called subordinate clauses, that take the place
of nouns or adjectives or adverbs.

These affirmations should not be taken as strictly grammatical
affirmations. What I mean is that, in a sentence such as `I want water',
the grammatical verb `want' acts as an adjective because the whole
sentence is equivalent to the phrase `desirable water', where the
grammatical verb `want' has become the adjective `desirable'. Similarly,
the sentence `I assure you that your daughter is lying' is dubitative
because it has the same cognitive structure as `I believe your daughter
is lying'; both express a reflexive evaluation of reality.


\Section Everything Changes

The difference established between the ^{permanence} of the ^{thing} we
see and the ^{change} in ^{behavior} allows us to distinguish between
nouns_{noun} and verbs_{verb}. This difference can become conventional.
For example, `fire' is a noun, and that means that it is something that
is permanent; however, ^{fire}, as ^[Heraclitus]^(Kahn1981) liked to
observe, is a continually changing process. Then again, `to burn' is a
verb and therefore denotes change. `To burn' and `fire' are semantically
synonymous and that is why the sentence `the fire burns' is a
^{tautology}. The words are not redundant, because `fire' can occupy the
syntactic position of subject, and `burn' that of predicate. This proves
that the difference between permanence and change established by
distinguishing between nouns and verbs can be merely grammatical, that
is to say, conventional; that is why it doesn't work for distinguishing
between what is permanent and what changes.

``Everything changes''
 ($\pi\acute\alpha\nu\tau\alpha \; \acute\rho\epsilon\tilde\iota$)
said ^[Heraclitus]. Fire and rivers_{river} are ^[Heraclitus]' two
prototypical examples, but it is the same for everything, everything
changes even if things keep their names. For example, people age, just
like all other living beings, and ^{aging} is the same as ^{burning}; it
is oxidation, although it is slower, that is, aging is merely a less
perceptible change than burning, at least for us. And what about
stones_{stone}?  They also change if we can observe them for long enough
or closely enough; but even if the stone didn't change, we would never
see it twice with the same light or from the same perspective. We
construct the stone out of our perceptions. And so, since things change
too, can we conclude, along with ^[Heraclitus], that nothing remains the
same?

In order to clear this matter up for once and for all, we have to go
back to the beginning. Even though the fire changes, while we perceive
signs_{sign} of fire, the object fire is still present. In this sense,
fire behaves perceptively just like other objects. That is, all objects
remain present as long as ^{perception} detects sufficient signs of
their presence_{present}. And this happens even if the sensory stimuli
vary from one instant to the next. This is so much the case that the
first stages of perception ignore what does not vary because they only
attend to change (see ^[Resnikoff]\footnote{_(Resnikoff1989),
$\S5.5$.}).

So that, in practice, what is useful to say about things is that they
change or remain the same, if it does any good. That is why, even if we
usually speak of solid ^{land}, we know that, geologically speaking, it
is more correct to speak of continental drift. Does the land change? No
and yes, depending on where our interest lies when we say it. As
^[Galileo] said about the apparent fixedness of the Earth:
``\latin{Eppur si muove}''.


\Section Syntax Is What Is Permanent

That is enough, for now, of practical questions; sometimes it is
convenient to speak of ^{change} and other times it is convenient to
speak of ^{permanence}. Let us now attack the theoretic question that
^[Heraclitus] and ^[Parmenides] discussed at the start of Greek
^{philosophy}. Because there is another more radical way of
understanding this matter, and it is the way that I prefer.

\breakif1

The ^{self} is the arquetype of ^{existence}---I am---because, for those
of us who believe ^[Descartes] in this matter, this is what is
immediate. Thus, we apply the qualities that we attribute to our own
existence to the existence of things, not the other way around. Since
the self is what is immediate, it is previous to everything, even to
time. And because the self is anterior to time, we also suppose that
things exist outside of time, by themselves, unalterable, as
^[Parmenides] does. But the self is free and, therefore, it is
syntactic; see ^>Pronouns>. And so we propose a generalization, that
syntax is what is permanent. We will now show the pertinence of this
generalization because it is interesting in itself, even though its
veracity or falsehood does not affect the nucleus of this theory.

Syntax is what is unchanging. Outside of ^{syntax} everything changes.
But it is only from syntactic permanence that change can be observed.
Once again, the subject's symbolism allows it to step back from change,
even while it is within the process of change, in order to observe it.
This abstract observation of change is what we call ^{time}.

A sentence, such as `The ^{dog} is playing with a ball', tells us that
the dog is moving, it is changing. But the dog in the sentence `The dog
is still' is also changing. The person who says the sentence is not
intentionally lying; on the contrary, rather, because his purpose may be
to make us notice that the dog is not bothering us, perhaps that its
movement is imperceptible. As we saw in the previous section,
_>Everything Changes>, it can be quite interesting, practically
speaking, to affirm that `The dog is still'. But the sentence `The dog
is still' would not be completely true for a physicist even if the dog
were dead, unless it was at a ^{temperature} of absolute ^{zero}, which
is, incidentally, impossible to reach.

The dog's impossible stillness contrasts with the permanence of
sentences such as ``Everything changes'' which, contrary to what it
affirms, has remained unaltered ever since ^[Heraclitus] said it.
Besides, when a sentence refers to a syntactic matter, such as `The verb
of this sentence is the previous `is'{}', then it does describe
something permanent.

And if these reasonings in favor of the equality of syntax and
permanence were not definitively convincing, further on, in _>The
Chomsky Hierarchy>, we shall see that syntactic expressions of
symbolisms have to be analyzed by a back-and-forth movement, that is,
without temporal restrictions.


\Section Definitions

After this journey through change, we should return to our path, and we
will do so at a place where two paths cross, the path of ^{reference}
and the path of the ^{problem}. These paths were initiated,
respectively, in ^>Existence and Reference> and in ^>Problems>.

Objects can be constructed starting from any expression that symbolic
language makes possible. That is, thanks to ^{reflection}, an object can
first be conceived starting from a sentence; a word can afterwards be
used to refer to this object, so that the final word summarizes the
original sentence. We would say that the original sentence is the
^|definition| of the final word. Thanks to definitions, ^{symbolic
language} is extensible_{extensibility}.

So objects that are constructed starting from a ^{problem}, that is to
say, starting from an interrogative sentence, exist. Also, taking
advantage of syntax's_{syntax} ^{recursivity}, there are objects that
are constructed by means of their being solutions to a problem. We call
these objects abstract objects_{abstract object}. All abstract objects
are concepts, not things, because their construction is theoretic; that
is why the term abstract object is synonymous with ^{abstract concept}.

Abstraction_{abstraction} seems a bit elaborate, and so it is, but for
this very reason, it is even more surprising when we realize that an
abstract concept is simply an object defined by its properties. This can
be deduced quite easily, because a problem is freedom and condition, and
the solution to the problem is whatever use of freedom satisfies the
condition, as we will see in _>The Theory of the Problem>. Therefore,
when I say that I am referring to the solutions of a problem, I mean
that I am referring to whatever fulfills the condition of the problem.

So when I speak of self-luminous celestial bodies, for example, I am
proposing a problem with the condition of being in the sky, because that
is what a celestial body is, and of being a source of light, not a
reflector of light. This is how I can refer to everything that fulfills
these two properties: being in the sky and emitting light. Once the
abstract concept is built, I can give it the name `star'. And so we
conclude: the definition of ^{star} is a self-luminous celestial body.

So each time we define something by referring to its properties, we are
using an abstract object. Is there any other way to define concepts? No.
The conditions can come, according to our sketch, from perception, from
behavior, from emotion, and from thought. We therefore have four pure
types of definition: descriptive definition, using qualities, when all
the conditions come from perception; genetic definition, how something
is done, which limits the precise behaviors necessary to obtain the
object; final definition, what I can use something for, if all of the
conditions refer to its utility and thus derive from the emotional
system; and theoretic definition, which establishes conditions that come
from other definitions, and whose recursivity is a product of thought.
In this way, thanks to theoretic definition, compound definitions can
also be created, crossing the pure types, if the properties come from
different type sources.


\Section Paradoxes

If the ^{problem} that defines an ^{abstract object} has no ^{solution},
then we have a ^{paradoxical object}. Since paradoxical objects are a
type of abstract object, all paradoxical objects are concepts, not
things, and ^{paradoxical concept} is synonymous with paradoxical
object. Paradoxical concepts have no referent_{reference} nor, as a
result, ^{meaning}. For example, just as the problem of finding the set
of round things that are square has no solution, the abstract object a
`round square' is a ^{paradox}.

Abstract concepts are independent from ^{perception}, from ^{behavior},
and from ^{emotion}, because they can be defined whether they have a
referent or not. Thus, a `^{horse}' can be defined as the quadrupedal
animal that fulfills a series of conditions, and a `^{unicorn}' as the
quadrupedal animal that fulfills the conditions that define a horse
along with an additional condition, having a ^{horn} in the middle of
its forehead. Keep in mind that, just as the unicorn is, for us, a
paradox, and we say that it doesn't exist, in the case of the horse,
there is a thing that is a horse, and there is an abstraction that is a
horse. This thing `horse' becomes present when we see a horse; the
abstraction `horse' is made present when the conditions laid out in its
^{definition} are fulfilled. The coincidence of a thing and an
abstraction is usually intentional, but not always successful. Consult
^[Eco]^(Eco1997) for more details about the peculiarities and
difficulties of ^{reference}, of ^{definition}, and of ^{abstraction}.

By the way, the fact that we believe the ^{sentence} `Unicorns don't
exist' to be true, being a sentence that doesn't deal with what is real,
proves that ^{truth} is the conformity of ^{syntactic expression} with
the ^{world}, not just with ^{reality}.
\breakif1


 \MT:def tribar(expr alpha) =
 \MT: pickup med_pen;
 \MT: save u, v; u = w/2; v = 12;
 \MT: z1 = (w/2,h/2) + u*(right rotated alpha);
 \MT: z2 = (w/2,h/2) + u*(right rotated (alpha+120));
 \MT: z3 = (w/2,h/2) + u*(right rotated (alpha-120));
 \MT: z1r = (w/2,h/2) + u*(right rotated (alpha+v));
 \MT: z2r = (w/2,h/2) + u*(right rotated (alpha+120+v));
 \MT: z3r = (w/2,h/2) + u*(right rotated (alpha-120+v));
 \MT: draw z1 .. z1r; draw z2 .. z2r; draw z3 .. z3r;
 \MT: draw z1 .. z3r; draw z2 .. z1r; draw z3 .. z2r;
 \MT: z1m = whatever[z1,z2r]; z1m = whatever[z1r,z3];
 \MT: z2m = whatever[z2,z3r]; z2m = whatever[z2r,z1];
 \MT: z3m = whatever[z3,z1r]; z3m = whatever[z3r,z2];
 \MT: draw z1 .. z2m; draw z2 .. z3m; draw z3 .. z1m;
 \MT: z1ra = z1 reflectedabout (z1r,z1m);
 \MT: z2ra = z2 reflectedabout (z2r,z2m);
 \MT: z3ra = z3 reflectedabout (z3r,z3m);
 \MT: draw z1m .. z1ra; draw z2m .. z2ra; draw z3m .. z3ra;
 \MT: z3x - z1x = whatever*(z3-z1m);
 \MT: z2x - z3x = whatever*(z2-z3m);
 \MT: z1x - z2x = whatever*(z1-z2m);
 \MT: z1ra - z1x = whatever*(z3-z1m);
 \MT: z3ra - z3x = whatever*(z2-z3m);
 \MT: z2ra - z2x = whatever*(z1-z2m);
 \MT: z1y = whatever[z1m,z3]; z1y = whatever[z1x,z2x];
 \MT: z2y = whatever[z2m,z1]; z2y = whatever[z2x,z3x];
 \MT: z3y = whatever[z3m,z2]; z3y = whatever[z3x,z1x];
 \MT: draw z1x .. z3y;
 \MT: draw z2x .. z1y;
 \MT: draw z3x .. z2y;
 \MT:enddef;

 \newdimen\tribar \tribar=75pt

 \MTbeginchar(\the\tribar,\the\tribar,0pt);
 \MT: tribar(-36);
 \MTendchar; \setbox0=\box\MTbox
 \MTbeginchar(\the\tribar,\the\tribar,0pt);
 \MT: tribar(-21);
 \MTendchar; \setbox2=\box\MTbox
 \MTbeginchar(\the\tribar,\the\tribar,0pt);
 \MT: tribar(-6);
 \MTendchar; \setbox4=\box\MTbox
 \MTbeginchar(\the\tribar,\the\tribar,0pt);
 \MT: tribar(9);
 \MTendchar; \setbox6=\box\MTbox
 \MTbeginchar(\the\tribar,\the\tribar,0pt);
 \MT: tribar(24);
 \MTendchar; \setbox8=\box\MTbox

 \dimen0=\hsize \advance\dimen0-\tribar

\noindent\parshape1 0pt \dimen0
  \hbox to0pt{\kern\dimen0\vbox to0pt{%\kern-9pt
  \box0\kern-10pt\box2\kern-9pt\box4\kern-9pt\box6\kern-12pt\box8\vss}\hss}%
\indent
Even some optical illusions_{illusion} are paradoxes, for example,
^[Penrose]'s tribar. It is perfectly possible to represent the tribar on
a flat surface, as the figures, adapted from
^[Resnikoff]^(Resnikoff1989), show. But the tribar concept defined as
the solution to the problem of constructing the three-\allowhyphens
dimensional thing whose two-\allowhyphens dimensional representations
are the flat tribar figures is paradoxical.

\strut\parshape1 0pt \dimen0
A paradox appears, then, when a ^{concept} becomes trapped in the
^{theoretical loop} without any chance of reaching the ^{practical
loop}. That is why theoretic systems, with their theoretic definitions,
run the risk of being paradoxical. Let us think, for example, about
^{phlogiston} and the ^{electron}. When the ^{physics} theory that
maintained phlogiston's existence and considered it to be the thing that
explained thermic phenomena (see ^[Kuhn]\footnote{_(Kuhn1970),
pages~99--100.}) was invalidated, pholgiston was considered nonexistent.
The electron, on the other hand, although it has been discovered to act
as a particle and as a wave, is at present considered to exist.
According to our jargon, phlogiston is paradoxical because it is not
held to be the cause of any practical effects, while the electron is not
paradoxical, no matter how unintelligible its behavior is, because it is
considered to be the explanation of certain physical phenomena,
principally electrical phenomena.

\indent\parshape4 0pt \dimen0 0pt \dimen0 0pt \dimen0 0pt \hsize
On the other hand, in a symbolic language, it is impossible to elude
paradoxes, because thanks to abstraction I can refer to what cannot be
referred to, as I have just done, and I can define what cannot be
defined, which is, by definition, that which cannot be defined. The
impossibility of eliminating paradoxes is precisely one of the
characteristics of symbolic languages, as will be proven in
_>Reflexive Paradoxes>.


\Section Tools

The subject's_{subject} symbolic apparatus is given the objective of
fulfilling, for example, a feeling of ^{thirst}, as well as the external
conditions for its satisfaction that surround it, in the manner of
^{present} ^{reality}, in which the subject will search for signs of
^{water}, rivers, or fountains.  That is to say, it is presented with a
^{problem}, just as the ^{knower} was. What is new about the subject is
that it treats the problem as an object, as if it were real, even though
it may have no meaning. It is because of this, and also because it can
treat the resolution as an object, that it can make abstractions,
^{reason}, and even construct the resolution of the problem.

This, which may seem very theoretic and not very practical, explains why
the subject is capable of designing tools, or of dressing itself, and
why simple knowers cannot do these things. A ^{tool} is a resolution
made into a thing; before it is made, the subject needs to ^{imagine}
it, that is, represent it to itself internally, and only a subject is
capable of imagining, in its ^{symbolic logic}, a resolution. I am using
the word `represent', even though its objectivist etymology bothers me.


\Section The Subject

In dealing with the ^{subject}, what there is to be said is too
overwhelming, so it will be better to stop now and finish with a summary
of the most important aspects.

Due to their evolutionary history, subjects use two types of
representations or objects: things and concepts. Things_{thing} are the
old objects constructed by perception, learning, and emotion, as was
already the case with knowers. Concepts_{concept} are the new objects
that thought produces voluntarily starting from other objects, which can
be things or concepts. That is why ^{symbolism}, which is the logic, or
system of representation, of the subject, has two layers: ^{semantics},
which is the old logic, with real_{reality} things, and ^{syntax}, which
is the new logic, with theoretic_{theory} concepts.

The novelty of symbolism is, then, the new syntactic layer that
originates in the ^{word}; the word becomes interiorized as the ^{idea},
and the idea turns into an ^{object}, to become the concept. We are
particularly interested in stressing a point: that ^{syntax} makes
conscious reflection and problem representation possible. Let us go over
how this occurs.

\breakif1

If seeing_{see} the exterior consists of recognizing objects in the
exterior phenomena, then when the subject recognizes objects among the
interior objects, the subject sees its own interior. It is, therefore,
the recursive_{recursivity} nature of the concept that permits cognition
to see itself; this is why it can be called reflected vision or
^{reflection}.

Concepts can refer directly or indirectly to things, from which they
take their meanings, or they can be without ^{meaning}. These concepts
that are free of meaning, that is, that are purely syntactic, are the
ones that allow problem representation, because they can express the
^{freedom}, or indetermination, that any ^{problem} sets forth.


\Section The Subject's World Is Symbolic

We can come to two conclusions about the subject's world:
\point The subject's ^{world} is symbolic, that is, it is reflexive,
discursive, and linguistic, and includes semantic ^{reality}.
\point The subject, when it considers problems with their solutions
and resolutions, evaluates different possible worlds. The subject
inhabits a world of possibilities. The subject is free.


\endinput
