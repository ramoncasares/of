% OL4.TEX (RMCG20000611)

\Part Exit

\Section What Am I?

The ^{exit path} goes from inside to outside. It begins with the
^{self}, but what am ^{I}?

For ^[Descartes], the self is what is certain, what is indubitable, the
only thing that is known with absolute ^{certainty}, and which,
therefore, must serve to know everything else. That is, that I, by
definition, am who defines things. This is, of course, a circular
definition that simply reveals that ^[Descartes]' self is atomic; in
other words, it cannot be analyzed.

Our exit path begins at the self, with ^[Descartes], so that we could
actually agree upon the self as a primitive, non-analyzable term.
Nevertheless, there are some qualities that can be affirmed about the
self, and others that cannot. For example, `I am free' is perfectly
valid, while applying physical qualities to the self is more
controversial. To begin with, it can be and is said that `I ^{weigh}
eighty kilos', but in fact what weighs is my ^{body}, not my self; this
is demonstrated when I lose ten kilos, because I continue to be the same
self, even though I weigh only seventy kilos. Besides which, of course,
the fact that I weigh eighty kilos is not a piece of knowledge that I
can reach by simple ^{introspection}, so it is not an indubitable piece
of knowledge, and is thus not a part of the self.

We already know that this procedure of eliminating physical properties
from the self ends up negating that the self is physical. And the
consequence is that the self is not scientifically explainable. This
limitation of present-day ^{science} is annoying, and some people,
^[Minsky]^(Minsky1985), for example, think that the self is an
^{illusion} and that ^[Descartes] must necessarily have been mistaken.
But ^[Descartes]' arguments are, on this point, unassailable, to the
point that not even their disqualification as illusory discredits them.
Because even if the self were merely illusory, science would remain
incomplete if it did not explain the nature of such an oft-repeated
illusion. I, on my part, believe that in order to overcome the
difficulty of the self it is necessary to broaden science's explanatory
power, not deny the fact. Let us continue, then.

The definitive factor, in this exit path, is what we know
introspectively about the self. As the great Irish mathematician
^[Hamilton]^(Webster1913) states:

 \null
 {\sl\leftskip=\parindent \rightskip=\parindent \noindent
The self, the~I, is recognized in every act of ^{intelligence} as the
^{subject} to which that act belongs. It is I~that perceive, I~that
imagine, I~that remember, I~that attend, I~that compare, I~that feel,
I~that ^{will}, I~that am conscious_{consciousness}.
 \par}
 \null

%The definitive factor, in this exit path, is what we know
%introspectively about the self. Introspectively, as the dictionary
%(^<Diccionario> de la ^[Real Academia Espa�ola]^(RAE1970)) sets forth,
%the self is the ``affirmation of consciousness_{consciousness} of the
%human personality as a rational_{reason} and free_{freedom} being''. In
%order to establish the semantic field of the self, we shall consult the
%Ideological Dictionary (^<Diccionario ideol�gico> de
%^[Casares]^(Casares1959)), that takes the ^[Academia] definition, and
%refers us to the analogic group of {\em consciousness}. This group has
%six sections. It is in the first that we find {\em self}, {\em
%^{subject}}, and {\em ^{person}}; the second is headed by {\em
%morality_{ethics}}; the third includes{\em perceive_{perception}}, and
%{\em reflect_{reflection}}; the fourth, {\em inflame}; the fifth {\em
%voluntary_{will}}; and the last one, {\em interiorly}.

\noindent
Any theory that explains the self should construct a coherent whole that
assimilates, in one way or another, this jumble of concepts that are
certainly related. And that is what we are doing.


\Section I Am Freedom to Not Die

The ^{subject} does not see itself as a subject, but as its ^{self},
that is, it sees itself as free. ^{I} am free to decide what to do.

What one does at every moment is done, and there is no ^{freedom} in
what is already done. That is why ^{freedom} is not to be found in what
is done, but in the ability to consider different possibilities of
doing, as many as the subject's imagination can produce. A ^{slave} can,
thus, be as free as his or her ^{master}, even though the slave must
immediately discard many options that the master must evaluate. The
serious problem is that the slave, due to a matter of mere mental
efficiency, ends up by not even considering (what for?) the socially
impossible possibilities.

The subject's freedom is not complete because it is limited, in the end,
by its own ^{death}. Put simply, dead people don't make decisions. They
are not free to act, so the freedom of live people only lasts as long as
they are alive. We can say, following this reasoning, that the subject's
self is free with the condition that it does not die. And with these two
pieces of news, we have already reached a concise, but sufficient,
definition of the ^|self|: I am freedom to not die.

\breakif1

But only the full development of the definition will show if it is
sufficient and correct or not. We will start to investigate this matter
by supporting the definition with an example, so that later we can
propose an equivalent definition of the self.


\Section The Penal System

I_{I} am, by definition, ^{freedom} to not die. The ^{penal system}
makes an {\em adjusted} use of this definition. The maximum
^{punishment} is the ^{death} penalty, and after that ^{life
imprisonment} which, although it does not completely eliminate freedom,
certainly diminishes it forever. A lesser punishment is ^{imprisonment},
which deprives the ^{prisoner} of ^{freedom} temporarily and partially.


\Section The Problem of the Subject

Insofar as the ^{self} is conscious of being free and mortal, it is
conscious of a ^{problem}: what to do in order not to die_{death}? We
will call this the ^|problem of the subject|.

The definition of self that we have proposed, I am ^{freedom} to not
die, and the problem of the subject, what to do in order not to die, are
equivalent. Because in one direction, if the problem is actually a
problem, it is because I have freedom to act; in the other direction, if
there is freedom and there is a condition, then the problem of what to
do with that freedom in order to fulfill the condition of not dying
arises immediately.
$$\hbox{Self} = \hbox{Problem of the Subject}$$

To summarize, if the self is, by definition, freedom to not die, then
the self is also the problem of the subject, what to do in order not to
die. Thus, the investigation of the self must be pursued by studying the
^{theory of the problem}, as ^[Dewey]^(Dewey1941) wisely advises.
$$\hbox{Self} \etapa \hbox{Problem}$$


\Section The Theory of the Problem

Every ^|problem| is made up of ^{freedom} and of a ^{condition}
 (see {\sc epa}\kern-2pt^(Casares1999)\kern-1pt~$\S4$).
There have to be possibilities and freedom to choose among them, because
if there is only ^{necessity} and ^{fatality}, then there is neither a
problem nor is there a decision to make. The different possible options
could work, or not, as solutions to the problem, so that in every
problem a certain condition that will determine if an option is valid or
not as a solution to the problem must exist.
$$
\centerline{$\hbox{\rm Problem}
 \left\lbrace\vcenter{\hbox{Freedom}\hbox{Condition}}\right.$}
$$

In addition, there ought to be certain ^{information} that should help
to make a decision, because if there isn't, the problem would have to be
resolved by ^{chance}. One type of problem, which we will call the
apparent problem, provides no information whatsoever. We will return to
the apparent problem later because it is right at the center of the
issue. For now, however, we will only make a note of how information
marks the difference between the two types of problem: the ^{apparent
problem}, without information, and the non-apparent problem, with
information. For example, the ^{problem of the subject} is not an
apparent problem, because the ^{subject} has an enormous amount of
information, conscious and unconscious, available about what favors
^{life} and retards ^{death}. We will later see where all this
information comes from, in _>The Apparent Problem>, _>The Universe>,
_>Knowledge Is Provisional>, _>The Problem of the Learner>,
_>A Priori Synthetic Knowledge>, and _>Culture>.

A fundamental distinction that we must make is between the solution and
the resolution of a problem. Resolving is to searching as solving is to
finding, and please note that one can search for something that does not
exist. Thus, ^|resolution| is the process that attempts to reach the
solution to the problem, while the ^|solution| of the problem is a use
of freedom that satisfies the condition.
$$ \hbox{Problem} \longrightarrow
 \hbox{Resolution} \longrightarrow \hbox{Solution}$$

We can explain this with another analogy. The problem is defined by the
^{tension} that exists between two opposites: freedom, free from any
limits, and the condition, which is pure ^{limit}. This tension is the
cause of the resolution process. But once the condition is fulfilled and
freedom is exhausted, the solution annihilates the problem. The
resolution is, then, a process of annihilation that eliminates freedom
as well as the condition of the problem, in order to produce the
solution.

\bigskip
\centerline{$\underbrace{\vcenter{
  \halign{\strut\hfil\rm#\crcr Freedom\cr Condition\cr}}
    }_{\hbox{\rm\strut Problem}} \, \Bigr\rbrace
 \mathop{\hbox to 80pt{\rightarrowfill}}\limits^{\hbox{\rm Resolution}}
 \hbox{\rm Solution}$}
\bigskip

A couple of mathematical examples may also be useful in order to
distinguish resolution from solution. In a problem of
arithmetical_{arithmetic} calculation, the solution is a ^{number} and
the resolution is an ^{algorithm} such as the algorithm for division,
for example. And in an ^{algebra} problem, the resolution is a chain of
equivalencies that transform the original expression, representing the
problem, into another expression that, in order to be accepted as the
solution, must be an ^{axiom} or, failing that, an expression already
proven by a previous ^{theorem}.


\Section Symbolic Logic

If we want to find the ^{self}, and the self is a ^{problem}, then we
must investigate which environments problems can inhabit. This question
opens up one of the most important areas of the ^{theory of the
problem}, because it links it to logic and language thanks to an
unexpected relationship the details of which can be found in \EPA5. If
the condition of allowing the representation of problems, resolutions,
and solutions is imposed as a design requeriment on a logic, then what
we obtain is a ^{symbolic logic}. To put it another way, problems,
resolutions, and solutions can be expressed in a ^{symbolic language}.
This discovery sheds light on the true nature of symbolisms, because it
lays the foundation for the strong relationship observed between
symbolism and the self, with this self understood as the problem of the
subject.

You can, in order to follow these explanations, interpret logic and
language as synonyms, or, more exactly, consider a ^|logic| to be a
system of ^{representation}, while a ^|language| is a system of
^{communication}. The relation between both is very strong, since only
what can be represented can be communicated or expressed. Besides, in
our case (see ^>Thought>), ^{thought} is mute symbolic speech, so surely
our symbolic logic is an interiorized symbolic language; but let's not
get distracted right now with these disquisitions.

To give an idea of why a symbolic language is necessary for expressing
problems, we will go back to the example in ^>Pronouns>. When we express
a problem, such as `what should be done?', the interrogative ^{pronoun}
`what' does not refer to anything. It must remain undefined in order for
the problem to be a problem. If, for example, in the case in question,
`what' refers to the action of walking, and not to any undetermined
action, then the expression `what should be done?'\ would mean
\meaning{walk}, and that is no longer a problem.

In the expression of every ^{problem} there must be an ^{unknown}, a
word with no ^{meaning} or, to express it more opportunely, a word that
is free of meaning, that represents the freedom of the problem. The rest
of the words express the conditions and the additional information. This
is, of course, a first approach to the matter, because many of the
conditions that must be taken into account do not need to be expressed.
In `what should be done?', it is a question of doing, but this single
explicit condition is certainly not the problem's only condition.
Gravity and death are certainly some of the tacit conditions. In the
same way, we can do without the word that designates freedom, the
interrogative pronoun `what', if the tone or the context are sufficient
to indicate that we are dealing with a problem.

Despite their interest, we will now leave these issues of the ^{economy
of language} which, on the other hand, do not change the basic fact: in
order to represent freedom, we must do without meaning. Or, to put it
another way, ^{semantics} is insufficient for representing freedom. This
gives us the key to the fundamental characteristic of symbolisms: that
all symbolism has two layers, semantics and ^{syntax}. Going back to our
example, `what' is merely a syntactic artifact.


\Section Semantics and Syntax

The image of a ^{symbolism} set up in layers explains that a symbolism
can be built by adding a syntactic layer to an anterior semantic logic,
this surely being the process by which our symbolism developed from a
purely semantic signic language, as we saw in ^>Symbolic Language>.

One way to make the new syntactic layer more capable than the semantic
layer that sustains it consists of representing, in the syntactic layer,
all the objects of the semantic layer and adding, besides, other
objects, some of them free of meaning. In these circumstances, we can
distinguish between two types of syntactic objects: we call a syntactic
object that directly represents a semantic object a ^|thing|, and an
object that is not a thing, that is, an object that has no immediate
^{meaning}, a ^|concept|. This implies that, if the concept has meaning,
this meaning is constructed from the meaning of the things. In other
words, the concept has a manufactured meaning, or it has no meaning.

As semantics is incapable of representing freedom, problems_{problem},
that combine a condition and freedom, can only be expressed
syntactically. This is why the problem must be a concept and must reside
in syntax.

Freedom_{freedom} is in the problems, but it cannot be in the solutions,
because the solutions must be completely determined, with no ambiguity
and with no degrees of freedom. Besides, solutions were already
represented in semantic logic (see ^>Problems>), so that, in syntax,
solutions will be things. In other words, solutions have meaning.
Obviously, even though solutions are things, not all things are the
solution to the problem.

The resolution process takes the problem, expressed syntactically, and
returns the syntactic expression of its solution, which is a thing. The
resolution is, then, a syntactic transformation. And therefore the
syntactic representation of a ^{resolution} will be an ^{algorithm},
which is what we call a syntactic expression that represents a syntactic
transformation. In short, algorithms have to be syntactic and
recursive_{recursivity} concepts.

Thus, while the problem and the resolution have to be syntactic and
recursive concepts, the solution has to be a semantic thing (the
technical details of this conclusion can be found in \EPA{5.9}). And so
it turns out that the ^{solution} cannot remain only in the syntax, but
must also transcend it; otherwise symbolism, converted into pure syntax
and trapped within itself, would be meaningless and useless. Seen in
this light, the resolution process moves the problem from syntax into
^{semantics}. In other words, the resolution searches for the meaning of
the problem.

What is new about recursive ^{syntax} is that it permits the complete
representation of problems, resolutions, and solutions, and so serves to
resolve problems. For example, the systematic design of tools_{tool},
which are resolutions made into things, needs symbolic elaboration (see
^>Tools>).


\Section Abstraction

Syntax_{syntax}, because it does not have to give each ^{object} a
^{meaning}, has greater expressive power than ^{semantics}. For example,
we have already seen that, though syntax permits us to express problems,
it is not possible to express a problem semantically, because the
^{unknown} of the problem must necessarily be free of any meaning. But
there's more.

When a problem has one single solution, we can use the ^{problem} to
refer to its solution, because its solution coincides with its meaning,
as we saw in the previous section (_>Semantics and Syntax>). This device
is called a ^{periphrasis}, because the problem refers, in the first
place, to itself, and only in a roundabout way to its solution.
Initially, this periphrasis may not seem very interesting, except for
making up riddles_{riddle} or metaphors_{metaphor}. Nonetheless, the
periphrastic use of problems to refer to some or all of their solutions
is an enormously fruitful procedure that is called ^|abstraction|.

If we want to refer to all the things that have a certain shape and
certain uses, what we do is construct a problem whose condition is the
conjunction of the conditions consisting of having this certain shape
and these uses. This way, the solutions of the problem we have
constructed coincide with those things to which we wish to refer. If we
look at it calmly, this artifice is simple; as long as something is
defined by its properties, then a problem whose condition is the
conjunction of these properties is constructed. This is how abstract
concepts are constructed.

Since there is a problem behind every abstract concept, there can be no
abstract concepts without a ^{symbolic language} capable of expressing
the problems.


\Section I Am in Syntax

In order to pursue the matter without forgetting that this whole quick
investigation about the ^{theory of the problem} has the purpose of
elucidating the self, we will stop to consider some new conclusions that
can already be applied to it.

The ^{self}, being a ^{problem}, is not a ^{thing}, but a ^{concept}.
And because it is a concept, the self inhabits the syntactic layer, not
the semantic layer. This may not seem very surprising, even though it
demonstrates that the self is intimately linked to ^{symbolic language}
and that the self is neither physical nor real. But, when we apply
^{abstraction} to the problem of the subject, that is, when we consider
its solutions, we immediately realize that the problem of the subject
has no solution. The ^{problem of the subject}, what to do in order not
to die?, has no solution because it is certain that, whatever I do, I
will die_{death}. And this is enigmatic. The self, which we make
equivalent to the problem of the subject and which we assimilate, by
abstraction, to its solution, is now an ^{enigma}. What is the
^{solution} to a problem with no solution?


\Section I Am Paradoxical

One way to construct a paradoxical concept consists of referring to the
solutions of a ^{problem} with no ^{solution}. So the self is
paradoxical, but let's not panic. All right, so I am paradoxical, but
what exactly is a paradox?

Technically (see \EPA{5.7.1}), a ^|paradox| is a syntactic ^{object}
with no semantic referent, that is, a paradox is a concept without
meaning. Unconditioned ^{freedom} is, according to this definition, a
paradox. An expression such as `this sentence is false' is a paradox
too, because if what it affirms is true, then it is false, but if it is
false, then what it affirms is true, and so it is false, and then true,
and so on forever, without ever reaching the final ^{meaning} of the
sentence.

To understand what consequences affirming that the self is paradoxical
has, we have to make a couple of deductions.

For the first deduction we have to remember that:
\point The ^{self} is equivalent to the ^{problem of the subject}.
\point We use problems to refer, by ^{abstraction}, to their solutions.
\point A solution is a meaning of the problem.
\par\noindent From which we can conclude that, if the self is paradoxical,
it is because the problem of the subject has no solution. This is
nothing new---I already know that I will die---but it ratifies the
foundations of the paradoxical self.

In order to make the second deduction, first we need to make a previous
observation.
\point In any problem there is a ^{tension} between freedom and the
condition, and it is the resolution process that eliminates this
tension, annihilating freedom and the condition, and producing a
solution (as shown in ^>The Theory of the Problem>).
\par\noindent Put negatively, this means that:
\point For any problem without a solution, the resolution process cannot
culminate in a solution, thus maintaining the tension between freedom
and the condition.
\par\noindent This, I suppose, is the reason why paradoxes cause anxiety.
Since a problem with no solution cannot stop being a problem in order to
be a solution, we can affirm that the problems with no solution are the
only necessarily stable problems.

The second deduction is now simple. Given that:
\point All problems with no solution are stable.
\point The problem of the subject has no solution.
\par\noindent We deduce that the problem of the subject is stable and that
the freedom of self cannot be eliminated.

In short, the paradoxical problem with no solution is the only system
capable of confining freedom; otherwise, this freedom is resolved and
disappears, annihilated, along with the condition. The self is
problematic, and it cannot stop being so, because it is paradoxical. The
paradoxical self maintains the tension between freedom and ^{death}.


\Section Immortality

I suspect that all of these deductions may confuse the issue more than
they clarify it. Because one can think, correctly, that when the
^{subject} dies_{death}, ^{freedom}, the corresponding ^{tension}, and
even the paradoxical ^{self} also disappear. This is true, but it does
not mean that the ^{problem of the subject} has a solution; in fact, it
means the very opposite. In order to solve it, the subject has to reach
absolute immortality, with no conditions, and what these deductions
affirm is that this necessarily immortal subject, with no vital
problems, will neither have a self, nor be free.

If the subject is necessarily immortal, the subject will have no
worries, will no longer be inquisitive (what for?), and its self will
become paralyzed; it will simply live eternally. Although I do not know
if you could really call living_{life} what the necessarily immortal
subject does, since it doesn't depend on eating, breathing, or any other
conditioning factor. If this ^{immortality} were not a necessary aspect,
if it were not unconditional, then the subject would have to maintain
certain conditions of immortality, such as having ^{food} and ^{air}
available, and the problem of the subject would still be valid, even
though it had been partly solved.


\Section I Am Alive

The true ^{solution} to the ^{problem of the subject} effectively
annihilates the ^{freedom} and the condition that constitute the
problem, but by fulfilling the condition. That is, the solution must
annihilate the problem of the subject, it must annihilate the self, by
making the subject immortal. Once the problem of the subject is solved,
there is no self, but there still would be a subject.

This explains why ^{suicide} is not the solution to the problem of the
subject, although it does effectively eliminate the problem. As we can
see, in order to realize that suicide is not the solution, but rather a
failure, it is necessary to distinguish between the self and the
subject.

These reflections finally allow us to step out of the primitive self.
The primitive ^{self} is the only ^{certainty}, but it is not the only
thing that there is, because it needs, at the very least, a subject that
sustains it. In order to establish this first step outwards, we must
study the subject. And we have some bits of information to study the
subject from the starting point of its self.

One is that the ^{subject} identifies with its self. Since the self is a
^{problem}, the solution to the problem must be of maximum importance to
the subject. And the solution, as we know, must fulfill the condition,
which is to not die. So that the subject's greatest interest is to live:
the subject is alive.

Besides, we know that the subject must have ^{symbolism} available in
order to sustain the self, which is syntactic. This means that it has a
syntactic layer that can represent problems, resolutions, and solutions,
a layer that the self inhabits, and a semantic layer in which it
executes the solutions that help it to not die, that is, to live: the
self is part of the subject.

In short, the self is part of a subject that is alive. And if the self
is part of life, it must be because the problem of the subject is part
of a more general problem, specifically the ^{problem of survival},
which is the name we give to the problem that defines ^{life}.
$$\vbox{\halign{\hfil#&\quad\hfil#\hfil\quad&#\hfil\cr
 \vtop{\halign{\hfil#\hfil\cr Self\cr $\parallel$\cr}}& %$\Updownarrow$\cr}}&
 $\subset$&
 \vtop{\halign{\hfil#\hfil\cr Life\cr $\parallel$\cr}}\cr
 \strut Problem of the Subject&
 $\subset$&
 Problem of Survival\cr}}$$


\Section The Problem of Survival

Just as by ^{introspection} we could find something out about the
^{problem of the subject}, we know nothing about the ^|problem of
survival| other than that it is a ^{problem}. And since we know nothing,
we will take nothing for granted, we will suppose nothing.

This not supposing anything is, in particular, the basis of the ^{theory
of subjectivity}, and, besides being reasonable not to suppose anything
when we know nothing, it is also consistent, as the development of the
idea will show. I will repeat this another way in order to highlight its
importance. Life is a problem, and it is nothing more than a problem.
All of the theory of subjectivity derives from this postulate, and that
is why {\sc epa}^(Casares1999) merely presents the problem and resolves
it.

Thus, according to the classification of problems that we saw in ^>The
Theory of the Problem>, the problem of survival is an ^{apparent
problem}, because it does not provide any ^{information}. The problem of
survival is the universal problem, of which all other problems form a
part, and it consists solely of ^{freedom} and of condition, because it
is an apparent problem.
$$\hbox{Problem of Survival} = \hbox{Apparent Problem}$$

The problematic nature of the self, which makes it equivalent to the
problem of the subject, caused us to study the theory of the problem.
Now the apparent nature of life, which makes it equivalent to the
problem of survival, causes us to study the apparent problem.
$$\hbox{Life} \etapa \hbox{The Apparent Problem}$$


\Section The Apparent Problem

The essential characteristic of an ^|apparent problem| is that it does
not provide any ^{information}. Nothing at all is known, either about
which resolutions are the most favorable or about the very condition of
the apparent problem. That is, the condition is unknown and therefore
cannot even be enunciated. In other words, faced with an apparent
problem, the only permissible action is to try to resolve it. That is
why the apparent problem is a pure problem, or a minimal problem, or, to
put it even another way, it is the problem without information. To be
exact, the only information that an apparent problem provides is that it
is a problem, that it is not anything else.

Apparent problems are so peculiar that our first impression may incline
us towards not paying them the attention they deserve. Since there is no
information available, any ^{resolution} is equally valid, and, at
first, it seems that there is no more to say. It is true that it seems
so, but it is false, because there are ways of going further.

If we only have one opportunity to resolve an apparent problem, then we
can choose the resolution by ^{chance}, because no other choice is
reasonably better, or worse, either. Once the resolution is executed, we
may have solved the problem, or we may not have solved it. In either
case, we now have one bit of information about the problem, that is,
whether the resolution we carried out solved it or not. Therefore, if we
can execute other resolutions, and pass the information obtained from
the executed resolutions to the new ones, then we have achieved a way of
resolving the problem that is better than pure chance, because it uses
more information.

^[Darwin]'s^(Darwin1859) evolutionary_{evolution} process uses this
method of repeating resolutions to face the apparent problem of
survival. Basically, each living_{life} being is a resolver of the
^{problem of survival} that, before it fails and dies,
replicates_{replication}, by itself or in couples. The resulting
replicas are not always perfect and include information about its way of
resolving the problem of survival. In order for the replicas, in turn,
to make replicas, the first replicas need to defeat ^{death}, at least
until the new replicas are alive; this screening process is called
^{selection}. The distinction between ^{solution} and ^{resolution} is
crucial here, because all living beings are mortal, so they are not
solutions, but they are resolvers.

The name of the apparent problem makes sense because, as we can see, the
only thing that we know about it is its ^{appearance}, that is, its
external_{exterior} reaction to our actions to try to resolve it; it
shows us nothing of its ^{interior}. It is as if we were trying to open
a ^{safe} by manipulating its external devices but without any
information about the opening mechanism. This manner of speaking can,
nevertheless, lead to error, because it takes for granted that the
apparent problem has an interior that is responsible for its external
appearance. This supposition, although it seems inevitable, is illicit
and is called ^{logicism}; we will take a look at this later on, in
_>The Problem of the Learner>.

The apparent problem is what ^[Klir]^(Klir1969) called the pure ^{black
box} problem.


\Section Evolution and Resolution

The ^{apparent problem} models the epistemological_{epistemology}
aspects of ^{life}. This means that it does not take into consideration
anything that does not affect ^{knowledge}, no matter how important it
may be for life itself. For example, it does not take into consideration
the details related to how the replicas_{replication} are made. But at
the same time, when we define life as an apparent problem, we are
generalizing life, because life is not committed to the ^{organic
chemistry} that makes it possible in the form that we know it.

So, if all of this is correct, there ought to be a correspondence
between the theoretical ^{resolution} of the apparent problem and
Darwinian ^{evolution}, in which each theoretically favorable resolution
corresponds to some step that life has actually taken. For example, if
making models_{modeling} of the exterior is shown to be better than not
making them for resolving the apparent problem, then we must conclude
that the evolutionary process will favor those individuals that
genetically code the mechanisms for forming models of the environment.
$$\hbox{Resolution of the Apparent Problem} \iff
  \hbox{Darwinian Evolution}$$

If the ^{apparent problem} generalizes ^{life}, the apparent problem's
^{resolution} will generalize Darwinian ^{evolution}. We will now try to
prove the validity of this correspondence between the theoretical
resolution of the apparent problem and Darwinian evolution, but in order
to prove it we will have to develop a theoretical resolution for the
apparent problem. The first thing we need to do in order to develop this
theoretical resolution is to formalize the apparent problem. The
^{formalized apparent problem} is even farther from the life that it
defines than the apparent problem is, so that formalization may
introduce distortions in the definition. Despite this reservation, the
next step for moving outward from the primitive self will consist of
formalizing the apparent problem in order to attack it theoretically.
$$\hbox{Apparent Problem} \etapa
 \hbox{Formalizing the Apparent Problem}$$


\Section The Universe

As we have presented it, an apparent problem is a ^{problem} in which
there is no ^{information}, but we should make this a bit more explicit.
In an apparent problem, there is ^{freedom} to act_{action} and the
^{condition} that the reactions_{reaction} be good, not bad. That is,
the condition of the apparent problem is the minimum condition possible,
and the relationship between the actions executed and the reactions
received is completely unknown. We will call the hypothetical object
that relates the actions with the reactions the ^{environment} or
^|universe|, and we can thus rewrite the previous sentence: in the
apparent problem the environment or universe is completely unknown, and
could even be non-existent.

An apparent problem cannot be solved \latin{a priori}, that is,
theoretically, because, as we saw in _>The Apparent Problem>, in
principle and because there is no information available, any resolution
is equally reasonable. In other words, faced with an apparent problem it
is impossible to design a solution and argue reasonably that it is a
solution, because you simply do not have information to make any kind of
argument. In order to obtain information about the apparent problem, as
we also saw, you have to face it repeatedly.

What reliable information can be obtained about an apparent problem? In
principle, what we obtain is the reliable information that when the
resolution that we will call $\Re$ has been executed, after the series
of resolutions  $\Re_0, \Re_1, \ldots \Re_t$, the problem is solved, or
not; one or the other depending on what happened to be the case. If we
repeat the resolution $\Re$, however, we cannot insure that the result
will repeat itself, because now the series of resolutions already
executed is not
 $\Re_0, \Re_1, \ldots \Re_t$, but $\Re_0, \Re_1, \ldots \Re_t, \Re$.
That is, the environment may have ^{memory} and react differently to the
same actions depending upon its state. And besides, it could also happen
that the relationship between the actions executed and the reactions
received had an aleatory_{chance} component, second reason why the
repetition of the actions does not insure a repetition of the reactions.

Thus, the information about the environment obtained when we face an
apparent problem takes the form of a probabilistic finite ^{automaton}.
It is an automaton, not a ^{function}, because the environment may have
memory. It is finite, not because of the universe, which may not be, but
because of the limitations of the apparatus of representation itself.
And it is probabilistic, once again, because the environment may be. The
universe, then, can be any probabilistic finite automaton; or at least,
with these specifications, the apparent problem can be formalized.

Now that we have reached this point, we need to stop once more to admire
the scenery before we continue.


\Section Time and Space

In order to formalize the apparent problem we have subreptitiously
introduced two concepts: ^{time} with ^{memory} and ^{space} with
^{action} and ^{reaction}.

We have already seen that, in order to resolve the apparent problem
better than just by chance, it was necessary to pass ^{information}
about the resolutions already made, the past resolutions, to the
resolutions that are being tried, that is, the present resolutions. This
is why a first temporal distinction between ^{past} and ^{present} seems
necessary.

Action_{action} and ^{reaction}, which together we will call
^{interaction}, need an ^{inside} and an ^{outside} in order to
distinguish the two directions, because the action goes from inside to
outside, it goes outward, and the reaction comes from outside inside, it
enters. This is why a first spatial distinction between inside and
outside, between the ^{interior} and the ^{exterior} seems necessary.

In the original definition of the apparent problem, we used concepts
such as ^{freedom}, ^{condition}, and ^{information}, but not time or
space. I believe, however, that the spatial-temporal concretion of the
apparent problem is what best formalizes the problem of survival because
it allows us to frame fundamental concepts such as ^{thing} and
^{death}, or ^{noun} and ^{verb}, with space and time, respectively.
Besides, ^{meaning} appears to relate the resolver's internal conditions
with the conditions that are external to it. And the reasoning that
fills the next two sections builds a foundation for the
^{irreversibility} of time, and, by doing so, gives the ^{future} its
open character. But we must not forget that, at least in theory, the
apparent problem could be specified in other ways.


\Section Knowledge Is Provisional

Repeating a ^{resolution} that previously solved an apparent problem
does not insure solving it this time. We have already seen that this
could happen if the ^{universe} is now in a non-propitious
state_{memory}, or just because ^{chance} goes against it. No matter how
shrewd we are, we will never be able to be certain of being right with a
^{prediction} about an apparent problem outcome. The ^{information} that
can be obtained from an apparent problem is provisional.

The information obtained from the ^{apparent problem} corre-\break
sponds, in Darwinian ^{evolution}, to the ^{knowledge} about the
^{universe} that ^{life} can achieve. Consequently, what this abstract
property, obtained in the theoretical resolution of the apparent
problem, means for life is that knowledge is provisional, hypothetical,
tentative, and never certain.

There is no knowledge that is absolutely certain_{certainty}, and even
though the ^{sun} appears every day in the east, and our prediction that
tomorrow the sun will come up in the east has been correct a thousand
times or more, even so, we cannot claim with assurance that ^{tomorrow}
the sun will come up in the east.

Even genetically coded information is provisional. This is why genetic
information can become dysfunctional, causing the species' ^{extinction}
in serious cases. This same conclusion also applies to our perception,
which we inherit genetically, and which determines the things that we
see, and to our emotional system, which we also inherit genetically, and
which gives meaning to the things. Forgetting this causes the error that
we call ^{objectivism} or, more generally, ^{logicism}. We will present
this in _>The Problem of the Learner>.


\Section Life Is Paradoxical

We cannot insure that the ^{resolution} that we are now going to attempt
will solve the ^{apparent problem} that we face, as we have just shown.
And because we can never insure the solution, it turns out that the
apparent problem has no definitive solution. And if the solution of a
problem does not annihilate it definitively, then it is not, in all
purity, a ^{solution}. And so we arrive at the most summarized
formulation of this property: apparent problems have no solution.

This means that every apparent problem is paradoxical_{paradox} and
consequently, according to the conclusions reached in ^>I Am
Paradoxical>, every apparent problem is stable because it cannot stop
being a problem.

Transposing, once again, from the ^{theory of the problem} to ^{life},
we conclude that life is paradoxical and, as a result, even if it is a
^{problem}, it remains unresolvable as such. Life is problematic, and
besides, it cannot stop being so, because it is paradoxical.


\Section Automatic Algebra

We will not present any mathematical formulation of the ^{apparent
problem} here, because it is too technical for the more philosophical
ends of this essay; if you are interested, you can consult
\EPA{1.4}, where you will find this formulation. Even so, we will
describe the formulation of {\sc epa} sufficiently to capture the most
interesting epistemological aspects.

Because we formalized the universe as a probabilistic finite
^{automaton} in ^>The Universe>, the formalized apparent problem will be
set forth in a logic that will allow us to represent probabilistic
finite automata, in addition to problems, resolutions, and solutions.
Our formalization uses the algebra of automata, or ^|automatic algebra|,
which is a ^{symbolic logic} and which allows us to represent binary,
synchronous, and probabilistic finite automata (\EPA{A}).


\Section Automata

But what is an automaton? An ^|automaton| takes ^{data} from the
exterior and produces data that it emits to the exterior. We will call
the first type of data ^{input} and the second type ^{output}. The
output depends as much on the input as on the ^{state} of the automaton.
Besides, the automaton changes states, and the state transitions also
depend on the state and on the input.

The description we have just given of an automaton is somewhat
theoretical, and so we will give an example that will help to pin down
the concept. We can say that a ^{calculator} with ^{memory} is an
automaton, because it produces data, the numbers it shows on its screen,
and it takes in data, the numbers and operations that we punch in.
Besides, the results depend, at times, on the contents of the memory,
and the contents of the memory depend on what we punch in and on its own
content.

There are more examples. An animal's ^{nervous system} takes data from
the exterior through the animal's senses and produces data that its
muscles and glands transform into actions. In addition, the data
produced depend as much on ^{perception} as on the internal state of the
animal; an animal does not act the same way if it is thirsty_{thirst} as
if its thirst is slaked.

The automaton is a very general model, especially if we keep in mind
that an ^{automaton without memory} is still an automaton. If we take a
good look at it, an automaton without memory is an automaton with a
single state, that is, an automaton that simply never changes state. The
automaton's ^{memory} is the measure of its number of states.

It is such a general model that any data processing system can be seen
as an automaton, even computers. In particular, the most common
computers at present, which fit _[von] ^[Neumann]^(Neumann1945)
architecture, are binary and synchronous finite automata. Computer
networks are also automata, ^{Internet} included, which is not
synchronous because it has no reference ^{clock}.

The ^{computer} is especially important because, ignoring physical
limitations, it is an automaton capable of imitating any other
automaton, that is, it can behave like any other automaton if it has the
right ^{program}. Technically, the computer is a {\UP}, that we will
present in _>The Universal Turing Machine>.

Automatic algebra_{automatic algebra} employs binary, synchronous, and
probabilistic finite automata. A binary finite automaton_{binary
automaton} employs a coding of the data based on two symbols that are
conventionally $1$ and $0$. Binary coding is the simplest and that is
why we use it, without suffering any loss of generality. A synchronous
finite automaton_{synchronous automaton} uses a single clock signal as a
reference that marks when all the value changes of the output and state
data happen. Synchrony, which consists of assuming that all operations
have the same duration, is conceptually simpler than asynchrony, which
requires us to keep in mind the different execution times of each
operation; this is why we use synchrony without suffering any loss of
generality, either. In the following sections, we will not repeat that
the finite automata that we refer to are binary and synchronous, as it
does not affect the results.

Having seen this, you can simply remember the following trick to
identify finite automata and forget about everything else that was said
in this section. If it can be programmed on a computer, that is, if a
computer can do it, then a finite automaton that can do it exists, too,
and \latin{vice versa}.


\Section Behavior

`Behavior' is a generic word that we already talked about in ^>Reality>,
but that we use in a technical way here, with a precise definition (see
\EPA{A.5.5}).
%
We say that two automata have the same ^|behavior| if it is not possible
to distinguish them from the outside_{exterior}. In their interior they
may be different, and, for example, one can use more states than the
other, but if it is impossible to distinguish one from the other by
managing the ^{input} and observing the ^{output}, then we say that the
behavior of both is identical. Therefore, if we are indifferent to the
internal construction of the automata, as is the case in this
theoretical investigation, then we are only interested in the automata's
behavior. In conclusion, we will consider `^{automaton}' and `behavior'
to be synonymous words, even though we know that there is a technical
difference between them, which interests engineers_{engineering}, but
not us (see \EPA{1.4.3}).

There is, nonetheless, a matter that may worry you. %the prudent reader.
Sometimes we talk about automata capable of various behaviors, and this
seems to be a contradiction. There is no trick; it is possible and has
been demonstrated mathematically by ^[Turing]^(Turing1936). The palpable
proof is the computer, which is capable of various behaviors. The
subtlety consists of considering there to be two types of ^{input}, the
ordinary kind and another kind, called the ^{program}, that specifies
the behavior. If we take automaton~$\aut A$ and fix its program to some
specified value, and we observe the ordinary input and output, but
ignore the program, then automaton~$\aut A$ will behave like a certain
automaton~$\aut B$. But if we fix the program to a different value and
observe automaton~$\aut A$ the same way, then it will behave, in
general, as a different automaton, let us call it~$\aut C$. Thus, what
the program achieves is that an automaton extended with a program
imitates other automata. Of course, if we look at all the ^{data},
including the program, then the automaton extended_{extension} with a
program also has only one behavior.


\endinput
