% OL5.TEX (RMCG20010122)

\Section Formalizing the Apparent Problem

When we use ^{automatic algebra} to formalize the ^{apparent problem},
the ^{solution} we are searching for must be a probabilistic finite
^{automaton}, that is, a ^{behavior}, that we will call~$\aut A$, that
will occupy the place of the ^{unknown} in the problem, and so we will
call it~$\aut A?$. As far as the ^{universe}, which we will represent
by~$\aut U$, goes, we know that it is another probabilistic finite
automaton, but we know nothing more, so that it could be any
probabilistic finite automaton; we will indicate this with $\forall \aut
U$. In addition, we know that in the apparent problem the unknown and
the universe interact. Finally, we have to check that the universe's
reactions are good, for which we will use a known ^{measurement} that
will be another probabilistic finite automaton {\Metric}.

All of these automata form the following circuit that represents the
^{formalized apparent problem} in automatic algebra (\EPA{1.4.1}).
% Total interaction and external measurement
\MTbeginchar(110pt,70pt,0pt);
 \MT: pickup thick_pen;
 \MT: x1 + x2 = w; x2 = x1 + 50u; y1b = y2b = 30v;
 \MT: rectangle(1)(20u,20v); % 1 is U
 \MT: rectangle(2)(20u,20v); % 2 es A
 \MT: z1lbl = z1; z2lbl = z2;
 \MTlabel(1lbl)"$\forall \aut U$";
 \MTlabel(2lbl)"$\aut A ?$";
 \MT: z3o = (x2r,y2); z3d = (x1l,y1);
 \MT: feedback(3,3o,3d)(h,10u,10u);
 \MT: x3lbl.l = x3br; y3lbl.t = y3br - jot;
 \MTlabel(3lbl)"\no m";
 \MT: x4tl = x1r; x4tr = x2l; y4tl = 2/3[y1b,y1t]; y4tr = 2/3[y2b,y2t];
 \MT: x4bl = x1r; x4br = x2l; y4bl = 1/3[y1b,y1t]; y4br = 1/3[y2b,y2t];
 \MT: arrow(4tl,4tr); arrow(4bl,4br); % arrows from U to A
 \MT: x4tlbl = x4tr - 10u; y4tlbl.b = y4tr + jot;
 \MT: x4blbl = x4tlbl; y4blbl.t = y4br - jot;
 \MTlabel(4tlbl)"\no n"; \MTlabel(4blbl)"\no v";
 \MT: pickup thick_pen;
 \MT: z6 = z2 - (0,30v); z6lbl = z6;
 \MT: rectangle(6)(20u,20v); % 6 is M
 \MTlabel(6lbl)"\Metric";
 \MT: pickup med_pen;
 \MT: z7o = 1/6[z4bl,z4br]; z7d = (x6l,y6); z7m = z7d - (10u,0);
 \MT: fork(7o,7m,7d);
 \MT: x7lbl = x7m; y7lbl.t = y7m - jot;
 \MTlabel(7lbl)"\no v";
 \MT: z8l = (x6r,y6); z8r = (w,y6); arrow(8l,8r);
 \MT: x8lbl = x3lbl; y8lbl.t = y6 - jot;
 \MTlabel(8lbl)"\sevenbf 1";
\MTendchar;
$$\box\MTbox$$

The boldface~$\no n$ indicates the part of the ^{input} that does not
directly influence the solution of the problem; this is the ^{neutral
input}. The input ^{data} that do influence the solution are marked by
the boldface~$\no v$; this is the ^{valued input}. The ^{output} is
indicated with the boldface letter~$\no m$.

The ^{measurement} {\Metric} uses a copy of the valued input~$\no v$. If
its output is set to the value $0$, then the automaton \aut A will have
failed as a solution to the problem. As for {\Metric}, we have only said
that it is a known automaton, and this is too generic. In order to pin
down these explanations, we will take a valid but simple case of
{\Metric}: it sets its output to $1$ if the majority of the input
values, at that moment, are $1$, and otherwise to $0$.

This being the case, the solution to the problem, $\aut A$, should be
capable of generating at every moment, whatever $\aut U$ is, a majority
of~$1$ values in the valued input~$\no v$. This is what we are left with
out of all the ^{formalized apparent problem}, and this is sufficient to
follow the explanations that are coming next.


\Section Notation

The calligraphic letters that accompany some words may be making you
uncomfortable. They are useful for pointing out the formal concepts
defined mathematically: for example, with one glance, you can determine
that the {\body} that I refer to is the body defined in the theory, and
no other. It is worthwhile to put up with them, because they reduce the
text's ambiguity. In exchange, they may hinder reading, but then again,
they may not, if you learn to ignore them after a bit.

\hbox{}\hbox{}\break

% Total Interaction with Minimum Semantics
\MTbeginchar(100pt,40pt,0pt);
 \MT: pickup thick_pen;
 \MT: x1 + x2 = w; x2 = x1 + 40u; y1b = y2b = 0;
 \MT: rectangle(1)(20u,20v); % 1 is U
 \MT: rectangle(2)(20u,20v); % 2 es A
 \MT: z1lbl = z1; z2lbl = z2;
 \MTlabel(1lbl)"$\forall \aut U$";
 \MTlabel(2lbl)"$\aut A ?$";
 \MT: z3o = (x2r,y2); z3d = (x1l,y1);
 \MT: feedback(3,3o,3d)(h,10u,10u);
 \MT: x3lbl.l = x3br; y3lbl.t = y3br - jot;
 \MTlabel(3lbl)"\no m";
 \MT: x4tl = x1r; x4tr = x2l; y4tl = 2/3[y1b,y1t]; y4tr = 2/3[y2b,y2t];
 \MT: x4bl = x1r; x4br = x2l; y4bl = 1/3[y1b,y1t]; y4br = 1/3[y2b,y2t];
 \MT: arrow(4tl,4tr); arrow(4bl,4br); % arrows from U to A
 \MT: x4tlbl = 1/2[x4tl,x4tr]; y4tlbl.b = 1/2[y4tl,y4tr] + jot;
 \MT: x4blbl = 1/2[x4bl,x4br]; y4blbl.t = 1/2[y4bl,y4br] - jot;
 \MTlabel(4tlbl)"\no n"; \MTlabel(4blbl)"\no v";
\MTendchar;
$$\box\MTbox
%\abovedisplayskip=0pt plus 3pt
%\belowdisplayskip=0pt minus 6pt
$$

And with these particulars, we are now ready to resolve the ^{formalized
apparent problem}, as we proposed in ^>Evolution and Resolution>. Let us
keep in mind that we are trying to design an ^{automaton}~\aut A capable
of generating at every moment, and whatever the ^{universe}~\aut U that
it faces is, a majority of~$1$ values in the ^{valued input}~{\no v}.
 $$\hbox{Formalizing the Apparent Problem} \etapa
   \hbox{Resolution}
%\abovedisplayskip=12pt minus 3pt
%\belowdisplayskip=0pt plus 3pt
$$


\Section The Formal Mechanism

In order to formalize the fact that nothing is known about the
^{universe}, we have said that the {\universe} could be any finite
^{automaton}. It can, for example, be the automaton that always produces
$0$~values as its output. We will call this {\universe} that only
produces $0$~values the \corporal universe$U_0$. Since it can be any
universe at all, it could also be the universe that always produces
$1$~values, and we will call this \corporal universe$U_1$.

So then, given that the {\universe} can be any one at all, it could also
be the \corporal universe$U_0$. And if the {\universe} were the
\corporal universe$U_0$, then no automaton~$\aut A$ could solve the
problem, because there would never be a majority of $1$~values in the
valued data. Exactly the opposite would happen if the {\universe} were
the benevolent \corporal universe$U_1$, because in that case any
\corporal automaton$A$ would solve it.

What happens, then, is that the ^{formalized apparent problem} has no
definitive ^{solution}, but that it depends on what the {\universe},
about which we know nothing, is like. This doesn't reveal anything new,
except that formalization conserves this fundamental characteristic of
the apparent problem, that is, that the apparent problem is paradoxical,
as we saw in ^>Life Is Paradoxical>.

Fantastic, but how can we resolve the apparent problem with so much
indetermination? Although there is no definitive solution that is valid
in every possible {\universe}, each concrete \corporal automaton$A$ will
solve it in specific \corporal universes$U$ from the total of all
possible ones. Let us take two automata, for example, $\aut A_a$
and~$\aut A_b$. Suppose that automaton~$\aut A_b$ solves the problem in
all the possible universes in which automaton~$\aut A_a$ solves it and,
in addition, in others. In this case we could affirm that, in spite of
all the indetermination, automaton~$\aut A_b$ is better than
automaton~$\aut A_a$.

In order to keep improving our resolution of the formalized apparent
problem, we will show that, if a specific automaton solves it in certain
universes, then we can construct another automaton that will also solve
it in these universes, and, additionally, in others (\EPA{1.5}). To
start the sequence, we will arbitrarily take an automaton, which we will
call a ^|mechanism|, and which we will indicate as $\aut A_0$ because it
will serve as the reference. The form of the {\mechanism} is the minimum
that is sufficient in order to occupy the place of the ^{unknown}~$\aut
A?$ in the formalized apparent problem (\EPA{1.6}).
% Mecanismo
\MTbeginchar(80pt,20pt,0pt);
 \MT: pickup thick_pen;
 \MT: x1 = w/2; y1b = 0; z1lbl = z1;
 \MT: rectangle(1)(20u,20v); % 1 is A_0
 \MTlabel(1lbl)"${\aut A}_0$";
 \MT: pickup med_pen;
 \MT: z2o = (0,1/3[y1b,y1t]); z2d = (x1l, y2o); arrow(2o,2d);
 \MT: z3o = (0,2/3[y1b,y1t]); z3d = (x1l, y3o); arrow(3o,3d);
 \MT: y2lbl.t = y2.o - jot; y3lbl.b = y3.o + jot;
 \MT: x2lbl = 1/2[x2o,x2d]; x3lbl = 1/2[x3o,x3d];
 \MTlabel(2lbl)"\no v"; \MTlabel(3lbl)"\no n";
 \MT: z4o = (x1r,1/2[y1b,y1t]); z4d = (w, y4o); arrow(4o,4d);
 \MT: y4lbl.t = y4o - jot; x4lbl = 1/2[x4o,x4d];
 \MTlabel(4lbl)"\no m";
\MTendchar;
$$\box\MTbox$$


\Section The Formal Adaptor

The {\mechanism} has a behavior that will solve certain \corporal
universes$U$, so that an {\adaptor} capable of behaving like a
{\mechanism} and in other ways besides, could solve more universes. This
observation permits us to design an {\adaptor} that improves the
{\mechanism}. Let us take a look at this.

The ^|adaptor|~$\aut A_1$ is made up of a {\body}, capable of various
behaviors, and of a {\governor}, that chooses which of the behaviors
will be executed.

\point The ^|body|~$\aut B$ is an automaton capable of behaving like
the {\mechanism} and in other ways besides, so that it is, technically,
an ^{extension} of the {\mechanism}. Although technicalities do not
interest us here, it is important to know that, given any finite
automaton, it is always possible to construct an extension of it, and,
as this does not depend on any condition, we say that the design of the
{\body} is specified.

\point The ^|governor|~$\aut G$ is another automaton whose purpose is
to order the {\bodys} ^{behavior}. The design of the {\governor} is not
specified, but if it fulfills a condition that we will call the
condition of the governor, then the {\adaptor} solves, at the very
least, all of the \corporal universes$U$ that the {\mechanism} also
solves. Consequently, the {\adaptor} will assuredly be as good or better
than the {\mechanism}. The ^{condition of the governor} is verified if,
when the {\adaptor} faces a {\universe} that the {\mechanism} solves,
its {\governor} orders the {\body} to behave in precisely the same way
as the {\mechanism}.

\noindent Finally, in order for the {\governor} to be in the best
position to fulfill the condition of the governor, it must receive all
of the ^{data}. Thus, we reach the definition of the form of the
{\adaptor} (\EPA{2}).
% Adaptor
\MTbeginchar(140pt,60pt,0pt);
 \MT: pickup thick_pen;
 \MT: x1r = w - 20u; y1b = 0; z1lbl = z1;
 \MT: rectangle(1)(20u,20v); % 1 is B
 \MTlabel(1lbl)"\aut B";
 \MT: x2r = x1l - 40u; y2b = y1t; z2lbl = z2;
 \MT: rectangle(2)(20u,20v); % 2 is G
 \MTlabel(2lbl)"\aut G";
 \MT: z3o = (x1r+5u,y1); z3d = (x2l,3/4[y2b,y2t]);
 \MT: forkback(3,3o,3d)(h,10u,10u);
 \MT: x3lbl.l = x3o; y3lbl.t = y3o - jot;
 \MTlabel(3lbl)"\no m";
 \MT: z4o = (x1r,y1); z4d = (w,y1); arrow(4o,4d);
 \MT: z6o = (0,1/4[y1b,y1t]); z6d = (x1l, y6o); arrow(6o,6d);
 \MT: z7o = (0,2/4[y1b,y1t]); z7d = (x1l, y7o); arrow(7o,7d);
 \MT: y6lbl.t = y6.o - jot; y7lbl.b = y7.o + jot; x6lbl = x7lbl = 5u;
 \MTlabel(6lbl)"\no v"; \MTlabel(7lbl)"\no n";
 \MT: z8d = (x2l,1/4[y2b,y2t]); z9d = (x2l,2/4[y2b,y2t]);
 \MT: z8m = z8d - (10u,0); z9m = z9d - (10u,0);
 \MT: z8o = (10u,y6o); fork(8o,8m,8d);
 \MT: z9o = (x8o,y7o); fork(9o,9m,9d);
 \MT: z10o = (x2r,y2); z10m = z10o + (5u,0);
 \MT: z10d = (x1l,3/4[y1b,y1t]); z10n = z10d - (10u,0);
 \MT: arrowww(10o,10m,10n,10d);
 \MT: x10lbl.l = x10m; y10lbl.b = y10m + jot;
 \MTlabel(10lbl)"\no b";
\MTendchar;
$$\box\MTbox$$

We have used an automaton, the {\body}, with various behaviors here, as
we mentioned previously in ^>Behavior>. The data~$\no b$ that the
{\governor} prepares, constitute the {\bodys} ^{program}, while the
^{neutral input}~$\no n$ and the ^{valued input}~$\no v$ are, in this
case, the ^{ordinary input}.


\Section Comparing Adaptors

In order to show that the equivalence between the resolution of the
apparent problem and Darwinian evolution, proposed in ^>Evolution and
Resolution>, is actually verified, we will see that the formal
{\adaptor} corresponds to the evolutionary adaptor presented from
_>The Adaptor> to _>The Adaptor's Reality Is Objective> in the entry
path.

A formal {\adaptor} is a finite automaton divided in two parts, a
{\body} capable of various behaviors and a {\governor} that chooses the
body's behavior. Thus defined, this formalized {\adaptor} does not look
very much like the evolutionary adaptor that placed a network of objects
between the phenomenon and the action. What happens is that the
resolutive formal {\adaptor} is more general than the evolutionary
adaptor of the entry path, which it includes. Because the evolutionary
adaptor chose the behavior according to the objects present and,
therefore, perception, which was the part that determined which objects
were present, was, by this very action of determining the objects, what
chose the adaptor's behavior. So, according to the formalization that is
carried out, perception does the job of the {\governor}, but in a
concrete way, using objects.

\hbox{}\hbox{}\break

The relation between both adaptors is made patent when we indicate, on
the diagram of the evolutionary adaptor shown in ^>The Adaptor>, the
parts of the formal {\adaptor} that carry out the processing of data
represented by each arrow.
$$\hbox{Phenomenon}
  \underbrace{
   \base{$\aut G\;\;$\cr$\longrightarrow\;$}
   \hbox{\strut Object}
   \base{$\aut B$\cr$\;\longrightarrow$}}_{\hbox{\strut Adaptor}}
  \hbox{Action}
$$


\Section Improving the Body

The formal resolution of the apparent problem took us from the
{\mechanism}, with only one behavior, to the {\adaptor}, which improved
the mechanism's action because it was capable of various behaviors. In
the same way, an {\adaptor} can improve another adaptor if it is capable
of more behaviors. There is little more to say about this quantitative
line of resolution improvement, which tends to increase the {\bodys}
versatility; but we must not forget it when we make the next qualitative
leap, even if it will be a much more spectacular one.


\Section Foresight

When the formal {\adaptor} was defined, its {\body} was specified, but
as far as its {\governor} is concerned, we only established a sufficient
condition for improving the {\mechanism}, the ^{condition of the
governor}. So now, in order to design a {\learner} that will improve the
{\adaptor}, we will focus on improving the {\governor}.

The {\governors} task, which consists of choosing the {\bodys} behavior,
can be done in several ways. If the {\universe} were fully known, which
is not the case with an apparent problem, then a {\governor} could be
designed that would systematically choose the optimal behavior, without
ever erring and, therefore, with no need to ever rectify its choice. We
will call any non-rectifying governor a ^{mechanical governor}. At the
opposite extreme, we have ^[Ashby]'s^(Ashby1956) ^{homeostat}, which
chooses behavior by chance, but rectifies the behavior, making another
equally aleatory choice, when the valuation obtained does not reach a
certain threshold. In this way, it is certain that only those behaviors
of the homeostat whose value is above this threshold are stable.

A continuum can be established from the mechanical governor, that never
rectifies the chosen behavior, to the homeostat, that puts all of its
trust in rectification, because it chooses without any criteria. We will
call a {\governor} that tests a ^{tester}; that is, the tester employs a
trial-and-error procedure. The homeostat is then an extreme example of a
tester.

A {\governor} that tests does not foresee, but tries a behavior, and if
it doesn't solve the problem, tries another. The case of the mechanical
{\governor} is worse, because, in addition to not foreseeing anything,
it is incapable of rectifying if the behavior is bad, something that the
tester is able to do. A better way of choosing behavior consists of
foreseeing_{foresight} its effect before executing it, because, if the
prognosis is exact, it avoids suffering the errors inherent to the
trial-and-error procedure. Thus, the difference between a {\learner} and
an {\adaptor} is that the {\learner} foresees the future, and the
{\adaptor} does not. In order to foresee the result of executing a
behavior, it is necessary to have an internal model of the exterior
available; we will call this model {\reality}. And with these
reasonings, we can now design a {\learner}.


\Section The Formal Learner

The ^|learner|~$\aut A_2$ has three parts: a {\body}, capable of various
behaviors, a {\modeler}, that models the exterior, and a {\simulator},
that chooses the current behavior comparing possible behaviors with each
other on the basis of their effects as predicted by the model.

\point The {\learners} {\body} is capable of various behaviors,
the more the better, as we saw in ^>Improving the Body>, but it has at
least all the behaviors the {\adaptor} has, in order to surpass it.
Technically, the {\learners} {\body} is an ^{extension} of the
{\adaptors} {\body} and its design is, therefore, specified.

\point The {\modelers}_|modeler| task consists of searching for
a model of the exterior {\universe}, a model that we will call
^|reality|~$\aut R$. The {\modeler} observes the action that the
{\learner} itself carries out upon the {\universe} and the universe's
reactions, and, with these data, makes reality, which is a behavior,
that is, an automaton~$\aut R$. If {\reality} is indistinguishable from
the exterior {\universe}, then it serves to make accurate predictions
and we say that it fulfills the ^{condition of the modeler}.

\point The ^|simulator|~$\aut S$ orders the behavior and,
in order to choose it, can use {\reality} and so foresee_{foresight} its
consequences before it executes the behavior. The simulation is
completely internal for the {\learner}; it receives {\reality} from the
{\modeler} and emits behavior to the {\body}; that is why its design is
completely specified.

\noindent We will show that the condition of the modeler is a
sufficient condition for the {\learner} to surpass the {\adaptor}. If
the {\learner} fulfills the condition of the modeler, that is, if the
predictions of {\reality} are accurate, then the {\learner} can simulate
perfectly the situation that the {\adaptor} faces. But the {\learner}
has the advantage of avoiding the actual execution in the exterior
{\universe} of the foreseeably worst behaviors, which coincide with the
worst behaviors, because we are supposing that the predictions are
correct.

The form of the {\learner} is shown in the following figure (\EPA3).
% Learner
\MTbeginchar(170pt,60pt,0pt);
 \MT: pickup thick_pen;
 \MT: x1r = w - 20u; y1b = 0; z1lbl = z1;
 \MT: rectangle(1)(20u,20v); % 1 is B
 \MTlabel(1lbl)"\aut B";
 \MT: x2r = x1l - 70u; y2b = y1t; z2lbl = z2;
 \MT: rectangle(2)(20u,20v); % 2 es M
 \MTlabel(2lbl)"\aut M";
 \MT: z3o = (x1r+5u,y1); z3d = (x2l,3/4[y2b,y2t]);
 \MT: forkback(3,3o,3d)(h,10u,10u);
 \MT: x3lbl.l = x3o; y3lbl.t = y3o - jot;
 \MTlabel(3lbl)"\no m";
 \MT: z4o = (x1r,y1); z4d = (w,y1); arrow(4o,4d);
 \MT: z6o = (0,1/4[y1b,y1t]); z6d = (x1l, y6o); arrow(6o,6d);
 \MT: z7o = (0,2/4[y1b,y1t]); z7d = (x1l, y7o); arrow(7o,7d);
 \MT: y6lbl.t = y6.o - jot; y7lbl.b = y7.o + jot; x6lbl = x7lbl = 5u;
 \MTlabel(6lbl)"\no v"; \MTlabel(7lbl)"\no n";
 \MT: z8d = (x2l,1/4[y2b,y2t]); z9d = (x2l,2/4[y2b,y2t]);
 \MT: z8m = z8d - (10u,0); z9m = z9d - (10u,0);
 \MT: z8o = (10u,y6o); fork(8o,8m,8d);
 \MT: z9o = (x8o,y7o); fork(9o,9m,9d);
 \MT: pickup thick_pen;
 \MT: z11 = z2 + (40u,0); z11lbl = z11; rectangle(11)(20u,20v);
 \MTlabel(11lbl)"\aut S";
 \MT: pickup med_pen;
 \MT: z12o = (x2r,y2); z12d = (x11l,y11); arrow(12o,12d);
 \MT: x12lbl = 1/2[x12o,x12d]; y12lbl.b = 1/2[y12o,y12d] + jot;
 \MTlabel(12lbl)"\no r";
 \MT: z13o = (x11r,y11); z13m = z13o + (5u,0);
 \MT: z13d = (x1l,3/4[y1b,y1t]); z13n = z13d - (10u,0);
 \MT: arrowww(13o,13m,13n,13d);
 \MT: x13lbl.l = x13m; y13lbl.b = y13m + jot;
 \MTlabel(13lbl)"\no b";
\MTendchar;
$$\box\MTbox$$

The data~$\no r$ allow the {\modeler} to communicate what {\reality} is
like to the {\simulator}. Since {\reality} is an automaton, the
data~$\no r$ specify a behavior, so they are a program (see
^>Behavior>). So the {\simulator} receives a program~$\no r$ that
describes the behavior of the exterior {\universe}, and in response
emits another program~$\no b$ that describes what the {\bodys} behavior
should be.


\Section Internal Logic

The {\learner} must have various possible representations of the
exterior {\universe}, which can be any kind of universe, the more the
better. The more representations it has, the more probable it is that it
will have one that behaves up to that moment like the {\universe}, and
thus will make better predictions, and work as {\reality}. That is, the
{\learner} needs to have an internal system of representation that we
will call ^|internal logic|. This internal logic must also represent
internally its {\bodys} behaviors in order to simulate their effects.
This explains the existence of somatic maps_{somatic map} in the
learners' ^{brain}.

Just as in ^{automatic algebra} both the {\universe} and the {\body} are
finite automata_{automaton}, the {\learners} internal logic must be
capable of representing finite automata, or behaviors_{behavior}, the
more the better.


\Section The Problem of the Learner

The {\simulator} has to resolve a problem similar to the problem of
survival, but, instead of being up against any universe, $\forall \aut
U$, it is up against the best model found by the {\modeler}, the model
we call {\reality}. This problem, which we will call the ^|problem of
the learner|, is not an ^{apparent problem}, because the {\simulator}
uses {\reality} as ^{information}.

The problem of the learner is not apparent, because the appearance
remains in the problem that the {\modeler} faces. If we forget this, we
may confuse {\reality} with the exterior {\universe}, a sin which we
will call ^{logicism}. It is a sin because, even if {\reality} has
predicted everything accurately up to now, we can never be sure that the
next prediction will be correct. The universe continues to be completely
unknown, that is, it can be any universe, $\forall \aut U$. The
{\universe} can, for example, be one that behaves just like {\reality}
until the following instant, in which it ceases to behave like reality.
Another consequence of this argument is that it is not possible to
verify if the ^{condition of the modeler} is fulfilled or not.

But even if it is a sin, the {\simulator} works with the logicist
hypothesis, that is, as if {\reality} were the exterior {\universe},
because it has no better hypothesis.


\Section Comparing Learners

Both the formal resolutive {\learner} as well as the evolutionary
learner of the entry path (seen from _>The Learner> to _>The Learner's
Reality Changes>) model reality, and both foresee_{foresight} the
future, and so the relationship between them is clear. The evolutionary
learner is a specific case of the resolutive {\learner}, because the
first one uses a reality of objects, while the second does not impose
requirements on {\reality}. Thus, for example, ^{logicism}, which
appeared when we studied the formal {\learner}, takes the form of
^{objectivism} when reality is objective.

In order to demonstrate the correspondence existing between both
learners, we pointed out the parts of the formal {\learner} that carry
out the data process represented by each arrow on the diagram of the
evolutionary learner shown in ^>Simulation>.
$$\hbox{Phenomenon}
   \underbrace{\strut
    \base{$\aut M\;\;$\cr$\longrightarrow\;$}
    \base{\vrule width0pt depth2pt$\aut S$\cr \onitself{\strut Object}}
    \base{$\aut B$\cr$\;\longrightarrow$}}_{\hbox{\strut Learner}}
  \hbox{Action}
%\abovedisplayskip=12pt minus 3pt
%\belowdisplayskip=0pt plus 3pt
$$


\Section The Double Resolution

In order to take the next step for resolving_{resolution} the ^{apparent
problem}, we must go back to the beginning of this resolution. At that
point we saw, in ^>The Apparent Problem>, that the way of attacking an
apparent problem, which has no definitive solution, consists of trying
various resolutions and passing ^{information} from the ones already
tried out to the ones that have not yet been tried. What we are
interested in now is that, given that the apparent problem has no
definitive solution, its resolution process consists of designing
resolutions that keep getting better with the information obtained from
previous resolutions; specifically, the {\mechanism}, the {\adaptor},
and the {\learner} are resolvers, not solutions.

That is why the resolution of the apparent problem has two levels. The
upper level is carried out by the general ^{resolution} of the apparent
problem corresponding to Darwinian ^{evolution}, for which all ^{life},
as a totality, must be considered as a single resolving being, as
^[Lovelock]^(Lovelock1979) proposes. The lower level corresponds to the
resolution that each individual living organism executes, since each one
is a resolver.


\Section The Formal Knower

The {\mechanism}, the {\adaptor}, and the {\learner} are resolvers. Each
one resolves problems in its own way, that is, each one searches, in a
specific way, for a behavior that will solve the problem. Thus, just as
the {\adaptor} is an improvement over the {\mechanism}, simply because
it is capable of more behaviors, the ^|knower|~$\aut A_3$, which is what
we call the next resolver of the formalized apparent problem, will be
capable of more ways of resolving problems.

Because the {\knower} is capable of resolving problems in various ways,
we can divide it into two parts:
\point A ^|mind|~$\syn M$ that is capable of resolving problems in
various ways.
\point An ^|intelligence|~$\syn A$ that decides which way of resolving
the problem should be executed at each moment.

\noindent In order for the {\knower} to improve the {\mechanism}, the
{\adaptor}, and the {\learner}, it is enough for its {\mind} to be
capable of resolving problems like each of these, and for the
{\intelligence} to satisfy the ^{condition of intelligence}, that is, to
choose the {\mechanisms} manner of resolving the problem if the
{\mechanism} solves it, the {\adaptors} manner, if this solves it, and
the {\learners} manner, if this solves it (\EPA6).
$$\hbox{Problem}\,
   \underbrace{\strut
    \base{{\frakx A}$\;\;$\cr $\longrightarrow\;$}
  \hbox{Resolution}
    \base{{\frakx M}\cr $\;\longrightarrow$}}_{\hbox{%
     Knower$\,{\cal A}_3$}}
  \,\hbox{Solution}
$$

The {\minds} design is completely specified and, as a last resort, the
{\mind} could be constructed by simply aggregating the {\mechanism}, the
{\adaptor}, and the {\learner}, and completing this aggregate with a
selector that would allow a choice between these components. The task of
the {\intelligence} would consist, precisely, of managing the selector
in order to choose one of the three. To summarize, we have a {\mind}
that permits the {\knower} to work at will as a {\mechanism}, as an
{\adaptor}, or as a {\learner}. But how should the selector be managed?
In other words, how should the {\intelligence} be designed?

As in the case of the {\governor} (see ^>Foresight>), the knower's
{\intelligence} can choose the resolution either mechanically, or by
testing_{test}. Testing does not insure that the {\knower} is an
improvement over the {\mechanism}, the {\adaptor}, and the {\learner}.
On the other hand, we might think that, because we have shown that the
{\learner} is better than the {\adaptor}, and the {\adaptor} is better
than the {\mechanism}, then the {\learners} way of resolution should
always be preferable. But we must remember that the improvements were,
in each case, conditioned by the fulfillment of some specific
requisites. Thus, the principal task of the {\intelligence} will be to
check to see if the ^{condition of the governor} is fulfilled or not, in
order to discriminate between mechanization and adaptation, and if the
^{condition of the modeler} is fulfilled, to discriminate between
adaptation and learning.


\Section Intelligence

The ^{condition of the governor} discriminates between the mechanism and
the adaptor. We must remember that the {\adaptor} improves upon the
{\mechanism} if it fulfills the condition of the governor, that is, if
the {\adaptor} behaves like the {\mechanism} when the mechanism is the
solution in a {\universe}. What happens in these circumstances is that
the {\adaptor} and the {\mechanism} have identical behavior, with the
mechanism's behavior being simpler. We can deduce the first rule of the
  \vadjust{\break}
knower's {\intelligence} from this. The {\intelligence}, upon realizing
that a mechanized behavior is the solution, will apply this
\hbox{mechanized} behavior to avoid other more costly calculations that,
working as an {\adaptor} or as a {\learner}, it would have to carry out.

If, on the contrary, no mechanized behavior solves the problem (we will
describe this case rather inaccurately as an unsatisfied condition of
the governor), then a more complex resolution is required, either as an
{\adaptor} or as a {\learner}. In this case, the discrimination depends
on the ^{condition of the modeler}. Because, if the condition of the
modeler is fulfilled, that is, if the {\reality} that the {\modeler}
finds foresees_{foresight} things with sufficient precision, then it is
worthwhile for the {\knower} to work as a {\learner}. But if the
condition of the modeler is not fulfilled, if the previsions of the best
{\reality} found are not trustworthy, then it is better for the knower
to do without them and work as an {\adaptor}. This is the second rule of
the knower's {\intelligence}.

Thirst_{thirst}, which indicates that the present behavior is not
solving the problem, is an example of an unsatisfied condition of the
governor. And ^{perplexity} is an example of an unfulfilled condition of
the modeler. Both feelings, because they are originated by unsatisfied
desirable conditions, are painful, but thirst is physical and perplexity
is mental. And so we discover two kinds of sentiments_{sentiment}:
corporal feelings, so classified because they depend upon the goodness
or badness of the behaviors that the {\body} executes, and mental
feelings, described thus because they depend on the goodness or badness
of the resolutions that the {\mind} carries out. So ^{feeling} is a
complex issue that includes internal information about the bodily state
and about the mental state.

We can say that all this information is organized in three maps: the
^{somatic map}, which represents the {\body}, the ^{mental map}, which
represents the {\mind}, and {\reality}, which is the map that represents
the exterior. In order for the three maps to be useful to the
{\intelligence}, they must be conveniently related. We give the name
^{meaning} to the ^{information} that integrates the different mental
and bodily, internal and external maps, and makes them cohere; this
information is fundamental to the unitary working of the {\knower}.

We will not go into greater detail here, such as the interesting illness
called ^[Camus] disease, because a complete description would deflect
our attention from the principal issues (if you are interested, you can
see the discussion of this topic in \EPA{6.4}).


\Section Comparing Knowers

A comparison of knowers maintains the tendency observed in previous
comparisons between the formal resolution of the apparent problem and
Darwinian evolution (see _>Comparing Adaptors> and _>Comparing
Learners>): the formal {\knower} is more general than the evolutionary
knower presented from _>The Knower> to _>The Knower's Reality Is
Semantic>. Both are capable of resolving problems in various ways, and
this is the essence of the knower. But when we described the
evolutionary knower at the entry path, we noticed that, of all the
different ways that reality can be used, one of them consists of using
it as a whole and others consist of using part of it. We saw, for
example, how thirst could be a subproblem of a problem of survival that
the evolutionary knower could face. In the case of the formal {\knower},
the ways of resolving a problem are not limited.

Nevertheless, in order for the comparison to be exact, we must keep in
mind that the evolutionary knower's semantic reality (seen in ^>The
Knower's Reality Is Semantic>) includes, in addition to strict
{\reality}, also the ^{somatic map}, the ^{mental map}, and the
^{meaning}, which is the relationship between the other three, in its
corresponding formal {\knower}.

When we were investigating the feelings of the {\knower} in the previous
section (_>Intelligence>), we discovered that the {\intelligence}
receives internal information, bodily as well as mental, and external
information. Therefore, the ^{emotional system}, that corresponds to the
formal {knower's} {\intelligence}, should be placed at the end of three
arrows in the diagram of the evolutionary knower, although these three
arrows are not shown because they would complicate the representation.
Each of these three arrows would originate at one of the three arrows of
the diagram of the ^{learner}; that is, these new arrows coming from
other arrows would represent data from a new layer.


\Section Improving the Mind

The {\knower} is an improvement over its predecessors because it is
capable of various resolutions, not just one. But by the same token, a
{\knower} can be an improvement over another if its {\mind} is capable
of more ways of resolving problems. Thus, increasing the {\minds}
versatility is the path for improving resolutors that was quantitative
at first and most probably preceded the following qualitative
improvement.


\Section The Formal Subject

The {\subject} is to the {\learner}, as the {\knower} is to the
{\adaptor}, with the {\adaptor} and the {\learner} being in the layer of
solutions or behaviors, and the {\knower} and the {\subject} in the
layer of resolutions. That is, just as the {\adaptor} was capable of
various behaviors and the {\learner} was, too, but with the possibility
of foreseeing the effect of these behaviors, the {\knower} is capable of
various resolutions and the {\subject} is, too, but with the possibility
of foreseeing_{foresight} the effect of these resolutions. We will say
that the {\subject} can ^{reason} about the resolutions.

In order to foresee the effect of the resolutions, it is necessary to
comprehend the problem that the resolutions are facing. Thus, the
^|subject|~$\aut A_4$ consists of three parts: an {\inquirer} that
searches for the best representation of the problem it is facing,
{\reason} that searches for the best possible resolution for the problem
that the {\inquirer} has found, and a {\mind} that resolves the problem
in the way decided upon by {\reason} (\EPA7).
$$\hbox{Problem}\,
   \underbrace{\strut
    \base{{\frakx I}$\;\;$\cr $\longrightarrow\;$}
    \base{\vrule width0pt depth2pt{\frakx R}\cr \onitself{\strut Resolution}}
    \base{{\frakx M}\cr $\;\longrightarrow$}}_{\hbox{%
     Subject$\,{\cal A}_4$}}
  \,\hbox{Solution}
$$

\point The {\subjects} {\mind} must be capable of various
resolutions, the more the better, as we saw in the previous section
(_>Improving the Mind>). But it must at the very least be capable of all
of the resolutions the knower is capable of, in order to surpass the
{\knower}. Just as the {\knowers} {\mind} is specified, so also is the
{\subjects} mind, because in the end they could be the same.

\point The ^|inquirer|~$\syn I$ searches for the best representation of
the problem it is facing. We will call this representation the ^|problem
of the subject|, or simply ^|self|~$\syn X$. If the problem encountered,
the {\self}, were indistinguishable from the problem that the {\subject}
is effectively facing, then we would say that it fulfills the condition
of the inquirer.

\point The task of ^|reason|~$\syn R$, to search for the best possible
resolution for the problem of the subject, is specified because it is
completely internal to the {\subject}. That is, the determination of
which is the best resolution for the \mental problem$X$ found by the
{\inquirer}, from among all the resolutions that the {\mind} is capable
of, can be made, as a last resort, by a systematic search.

\noindent The ^{condition of the inquirer} is a sufficient condition
for the {\subject} to surpass the {\knower}, because, if it is
fulfilled, that is to say, if the {\inquirer} finds a problem that is
indistinguishable from the problem it is facing, then the {\subjects}
{\reason} can calculate ahead of time the goodness or badness of the
resolution before undertaking it, and can thus avoid suffering the
errors that the {\knower} is incapable of anticipating. This is so, even
though it is impossible to verify that the condition of the inquirer is
definitively satisfied; this is impossible because the {\subject} is
facing an apparent problem.


\Section The Symbolic Subject

In order for the {\subject} to foresee_{foresight} the effect of the
possible resolutions on the problems to be solved, it must be capable of
representing problems and resolutions, and to do that, it must have a
^{symbolic logic}, with ^{semantics} and recursive ^{syntax}. We will
now take these two steps, but we will take them one by one.

A first step towards converting the {\knowers} ^{internal logic} into a
symbolic logic consists, as we saw in ^>Semantics and Syntax>, in adding
a syntactic layer to the preexisting logic, which we will call semantics
because the knower's reality is semantic (see ^>Comparing Knowers>).
That is, the {\knowers} logic, capable of representing the exterior
{\universes} behavior and the {\knowers} own behavior, as well as the
conditions that its {\intelligence} evaluates, become the semantic layer
of the {\subjects} logic. And a syntactic layer appears over this
semantic layer to complete the symbolic logic. Once the {\subjects}
symbolic logic is constructed in this way, its solutions have to have
the same nature as those of the {\knower} and of the {\learner}; that
is, they are automata_{automaton}~$\aut A$, or behaviors. More than a
limitation, this is a requisite. We call a logic with semantics and
syntax a ^{grammatical logic}.

But since not just any syntax will do, it is necessary to take the
second step. The more problems the {\subjects} logic is capable of
representing, the more possibilities there are that it can represent the
problem that it is facing at the moment. In addition, given the
evolution of the {\mind} towards increasing versatility (see ^>Improving
the Mind>), it is also necessary for the {\subjects} logic to represent
the greatest possible quantity and variety of resolutions. In these
circumstances, and keeping in mind that resolutions are syntactic
transformations (as we saw in _>Semantics and Syntax>), the syntax must
have the maximum expressiveness and, therefore, it must be
recursive_{recursivity}. In other words, the syntax must be such that
even the syntactic transformations themselves can be expressed. To
conclude, and given that a ^|symbolism| is precisely any grammatical
logic whose syntax is recursive, it turns out that the {\subjects} logic
must be symbolic.


\Section The Chomsky Hierarchy

In order to understand the scope of the symbolisms, with their recursive
syntaxes, we would do well to locate them in the ^[Chomsky] ^{hierarchy
of grammars}. I will attempt to explain the situation without going into
the most technical details, but if you wish, you can skip the entire
trip through the ^{theory of computation} and proceed directly to
^>Algorithms>, where the most important conclusions are summarized.
Another possibility, which I recommend to you, is that you become
interested in the theory of ^{computation}. A solid text such as those
by ^[Fernández] and ^[Sáez Vacas]^(Fernández1987), by ^[Carroll] and
^[Long]^(Carroll1989), or by ^[Arbib]^(Arbib1987) can ease your
introduction to this subject.

\mental Grammar$G$_{grammar} is the name given to the set of rules that
allow us to construct all the syntactic expressions of a certain \Mental
language$L$, which is how the set of correctly constructed syntactic
expressions is defined. It is important to note that, with this
technical definition and contrary to common use, \mental language$L$ is
equivalent to syntax, with semantics excluded. So that the first step in
the analysis of a ^{syntactic expression} consists of deciding if it
belongs to the language or not, that is, if it is correctly constructed
or not. The mechanism that is capable of deciding if a specific
expression belongs to the \mental language$L$ is called a ^{recognizer},
and we say that it accepts_{accept} the expression if it recognizes it
as syntactically correct.

The ^[Chomsky] hierarchy of grammars identifies four types of languages
that, I repeat, should be more correctly called syntaxes, each of which
is associated with a type of grammar and with a type of recognizer. The
difference between the simplest type of language in the hierarchy,
\Mental regular language$L_3$, and the most complex, unrestricted
\Mental grammatical language$L_0$_{unrestricted language}, is the
following. A finite ^{automaton} can, with a single sequential reading,
that is, without needing to retrocede, recognize any syntactic
expression of a \mental regular language$L_3$; but to analyze a
syntactic expression of a \mental grammatical language$L_0$, it
generally needs to transform it into other expressions, whose lengths
are not limited, going forward and backward in the reading and writing
of the expressions as much as necessary for the analysis.

Since the lengths of the intermediate expressions are not limited, an
automaton that analyzes expressions from an unrestricted \mental
grammatical language$L_0$ will need to use a potentially infinite tape
in order to retain them while it analyzes them, because the finite
automaton's memory is, by definition, finite. We call a finite automaton
with an infinite tape a {\TM}. The ^{tape} is, then, another memory that
the {\TM} uses specifically to retain syntactic expressions while it is
analyzing them. The finite automaton of the {\TM} is called the
\Processor$T$.
$$\hbox{^[Turing] Machine~$\syn T$}\left\lbrace
  \vcenter{\hbox{Processor~${\cal P}_{\syn T}$}
           \hbox{Tape}}\right.
$$

In order to indicate that, if we write the syntactic expression~$\frak
e$ on the tape of the {\TM} whose processor is~${\cal P}_{\syn T}$ and
we leave it functioning, then when it stops we will find the syntactic
expression~$\frak r$ on the tape, we will use the following notation:
$${\cal P}_{\syn T}[{\frak e}] \rightarrow {\frak r} .$$
If, on the contrary, the {\TM} does not stop when we write the
expression $\frak x$, then we would say that $\frak x$ is a ^{paradox},
and we would indicate this as follows:
$${\cal P}_{\syn T}[{\frak x}] \rightarrow \infty .$$

Given any expression, the \corporal finite automaton$R_3$ that is the
recognizer for a certain \mental regular language$L_3$ is always capable
of deciding if the expression belongs to it or not, with a single
sequential reading. But the most that we can insure about the ^[Turing]
machine~$\syn R_0$ that is a recognizer of the \mental grammatical
language$L_0$ is that it may recognize the expression as belonging to
this language, that it may recognize it as not belonging to this
language, or that it may never finish analyzing it. In other words,
paradoxes can be expressed in \mental grammatical
languages$L_0$_{grammatical language}.

We owe this important result, known as the halting problem, and closely
related with ^[Gödel]'s^(G\"odel1931) ^{undecidability} theorem,  to
^[Turing]^(Turing1936); we will study it in greater detail in
_>Reflexive Paradoxes>. It is as if the atemporality or reversibility of
non-restricted syntactic expressions, that can be examined with no
limitations, going forward and backward, were to blame for paradoxes.
And, based on the idea that time cannot be seen unless it is from an
atemporal logic, I suspect that here we find the essence of ^{time}. But
vertigo prevents me from continuing to investigate this matter, that was
already examined in ^>Syntax Is What Is Permanent>. Besides, it would
distract us from another, more fundamental, discovery also by ^[Turing]
in 1936.


\Section The Universal Turing Machine

There is a type of ^[Turing] machine, as we already announced in
^>Behavior>, that is called the {\UTM} and that is capable of behaving
like any {\TM}. More specifically, a {\UTM} interprets part of the
^{syntactic expression}, a part that we will call \Mental
algorithm$P_{\syn T}$, as the description of a specific {\TM} to
imitate, which can be any one at all. So the result of analyzing the
other part of the expression, the so-called ^{arguments} or \Mental
parameters$d$, is, in every case, the same result that the {\TM} being
imitated would obtain. That is to say:
$${\cal P}_{\syn U}[{\syn P}_{\syn T}({\syn d})] \equiv
  {\cal P}_{\syn T}[{\syn d}] .$$
It is surprising that the ^[Turing] machine to be imitated can be any
one at all, even the {\UTM} itself. It seems quite simple to say that
the imitator can imitate itself; however, the imitator is an imitator
because it imitates someone, so that, in this case, a part of the
\mental arguments$d$ is the \mental algorithm$P'$ that describes the
machine that the imitated imitator imitates. It's clear, isn't it?
$${\cal P}_{\syn U}[{\syn P}_{\syn U}
( \underbrace{{\syn P'}({\syn d'})}_{\hbox{\frak d}})] \equiv
  {\cal P}_{\syn U}[\underbrace{{\syn P'}({\syn d'})}_{\hbox{\frak d}}]
  \equiv
  {\cal P}'[{\syn d'}] .$$

What happens, then, is that, once the ^{complexity} of the {\UTM} is
achieved, and it is not infinite because its \Processor$U$ continues to
be a finite automaton, we already have reached the maximum power of
syntactic analysis that it is possible to reach. No matter how complex
the {\TM} that we want to imitate is, the {\UTM} can do it. The key
consists in being able to represent the syntactic \processor$T$ itself
syntactically, because the \mental algorithm$P_{\syn T}$ effectively
represents the syntactic transformation that this processor undertakes.
So, within the \mental grammatical languages$L_0$_{grammatical
language}, we will call every language whose recognizer is a {\UTM} a
\Mental universal language$L_{\syn U}$ or, and it is the same thing, a
\Mental recursive syntax$L_{\syn U}$.
$$ {\syn L}_{\syn U} = \hbox{Universal Language}
                     = \hbox{Recursive Syntax} $$

We say that the {\UP}, that is, the processor of a {\UTM}, is a ^{syntax
engine} because it is capable of executing any syntactic transformation.
$${\cal P}_{\frak U}=\hbox{Universal Processor}=\hbox{Syntax Engine}$$


\Section Expressiveness

We must distinguish between two types of set relation that can apply to
languages: inclusion, used in the ^[Chomsky] hierarchy of grammars and
which relates two sets of languages, and ^{expressiveness}, which
relates two languages, a language being the set of its syntactic
expressions.

For example, the language that only accepts the sequence~$\frak abc$ is
a regular \Mental language$L_3$ because there is a finite automaton that
only accepts this sequence. And this same language is also a grammatical
\Mental language$L_0$, because there is also a
 ^[Turing] machine_{\string\string\string\tURING[Turing] machine}
which only accepts the ^{syntactic expression}~$\frak abc$. Due to this
type of reasoning, we conclude that all the regular \mental
languages$L_3$ are \mental grammatical ones$L_0$ and, as a consequence,
the set of regular languages~$\{ {\syn L}_3 \}$ is included in the set
of grammatical languages~$\{{\syn L}_0\}$:
$$ \{ {\syn L}_3 \} \subset \{ {\syn L}_0 \} .$$

On the other hand, though, no \Mental universal language$L_{\syn U}$ can
accept only the sequence~$\frak abc$, although they can all accept an
extended version. The possibility of extension requires them to accept
other sequences. Let us look at this aspect in greater detail.

Let~${\cal P}_{\frak\!abc}$ be the processor of the ^[Turing] machine
that only accepts the expression~$\frak abc$. Let $\syn U$ be a
universal ^[Turing] machine_{universal
\string\string\string\tURING[Turing] machine}, and let ${\syn P}_{\frak
abc}$ be the syntactic expression that represents the processor~${\cal
P}_{\frak\!abc}$ in $\syn U$. With this terminology, $\syn U$ will
accept the expression~${\syn P}_{\frak\!abc}({\frak abc})$:
$${\cal P}_{\syn U}[{\syn P}_{\frak abc}({\frak abc})] \equiv
  {\cal P}_{\frak\!abc}[{\frak abc}] \rightarrow \true .
$$
Thus, the extended expressions have two parts: the ^{algorithm}, in the
example~${\syn P}_{\frak abc}$, and the ^{parameters}, which are~$\frak
abc$ in the example. And we can now affirm that \Mental recursive
syntax$L_{\syn U}$ accepts the expression~${\syn P}_{\frak\!abc}({\frak
abc})$, which should be read: `what~$\frak abc$ expresses in the
language recognized by~${\cal P}_{\frak\!abc}$'. But the \mental
universal language$L_{\syn U}$ accepts, in addition, other expressions,
for example ${\syn P}_{\syn U}({\syn P}_{\frak abc}({\frak abc}))$,
because:
$${\cal P}_{\syn U}[{\syn P}_{\syn U}({\syn P}_{\frak abc}({\frak abc}))]
  \equiv {\cal P}_{\syn U}[{\syn P}_{\frak abc}({\frak abc})] \equiv
   {\cal P}_{\frak\! abc}[{\frak abc}] \rightarrow \true .$$

This is why we say that \mental universal languages$L_{\syn U}$ are more
expressive than \mental regular ones$L_3$, although they do not include
them; $\forall{\syn L}_3,{\syn L}_{\syn U}$:
$$\eqalign{
   {\syn L}_3 &\prec {\syn L}_{\syn U} \cr
   \{ {\syn L}_3 \} &\not\subset \{ {\syn L}_{\syn U} \}  . \cr
 }$$

The \mental universal language$L_{\syn U}$ is the most expressive of the
languages with grammar because it can express everything that any other
\mental grammatical language$L_0$ can. This is because the syntactic
transformations of any other language, whatever they may be, can be
expressed in \mental recursive syntax$L_{\syn U}$ and be interpreted as
such. And, in conclusion, no \mental grammatical language$L_0$ can be
more expressive than a \mental universal language$L_{\syn U}$;
$\forall{\syn L}_0,{\syn L}_{\syn U}$:
$${\syn L}_0 \preceq {\syn L}_{\syn U} .$$

In a way, \mental universal languages$L_{\syn U}$ are too expressive.
They are so expressive that, if the expression~$\frak x$ is paradoxical
in any \Mental grammatical language$L_0$, for example in the language
recognized by the {\TM} whose processor is~${\cal P}_{\syn T}$, then the
expression~${\syn P}_{\syn T}({\frak x})$ is paradoxical in \Mental
recursive syntax$L_{\syn U}$:
$$ {\cal P}_{\syn U}[{\syn P}_{\syn T}({\frak x})] \equiv
   {\cal P}_{\syn T}[{\frak x}] \rightarrow \infty .$$
This proves that there are paradoxes in all \mental recursive
syntaxes$L_{\syn U}$.

\break

Besides, paradoxes_{paradox} that can be expressed in a \Mental
universal language$L_{\syn U}$ are pertinacious, because it is not even
possible to recognize them all. In order to prove this, it is necessary
to show that reflexive paradoxes can be expressed in a
\mental recursive syntax$L_{\syn U}$. A ^{reflexive paradox}, such as `this
sentence is false', is a paradox that refers to itself, so we can also
call it self-referential_{self-reference} paradox, or even
^[Epimenides]'^(Northrop1944) paradox, ^[Epimenides] being a Cretan who
said, as the story goes, that everything Cretans say is a lie.


\Section Recursivity

Reference_{reference} allows us, in syntactic expressions, to use names
as abbreviations of other expressions, generally longer ones. We call
the operation of giving a ^{name} to an expression ^{definition}. Thus,
for example, if we write the definitions after a colon, with~$n$ being
the name of the syntactic expression~$\syn n$:
$$ {\cal P}_{\syn U}[{\frak e} n {\frak o} : n = {\frak n} ]
   \rightarrow
   {\cal P}_{\syn U}[{\frak eno} : n = {\frak n} ] .$$
It is sufficient for us that the definitions of the names are fixed,
that is, that they cannot be changed. To prove that any {\UTM} can use
names, it is enough to show that a ^[Turing] machine exists that is
capable of substituting a name with an expression. That is, we only have
to show that it is possible to design a ^{dictionary}~$\syn D$ ^[Turing]
machine that, given a name, returns its expression:
$$ {\cal P}_{\syn U}[{\syn P}_{\syn D}(n)] \equiv
   {\cal P}_{\syn D}[n] \rightarrow {\frak n} .$$
Thanks to definitions, \mental recursive syntaxes$L_{\syn U}$ are
extensible_{extensibility}.

The other necessary requirement for being able to express reflexive, or
self-referential, paradoxes, is that syntactic transformations
themselves can be syntactically expressed, as is the case in
\mental universal languages$L_{\syn U}$. Because then the ^{algorithm}
that represents a syntactic transformation can be given a name, and this
same name can appear in the definition of the algorithm itself, a
procedure that is called ^|recursivity|, and gives its name to
\mental recursive syntaxes$L_{\syn U}$_{recursive syntax}.


\def\algStop{\inmmode$\syn H$}
\def\algPdox{\inmmode$\syn Z$}

\Section Reflexive Paradoxes

So we can now show that, in a \Mental universal language$L_{\syn U}$, it
is impossible to recognize all of the paradoxes. We will show this by
\latin{reductio ad absurdum}, that is, we will examine what would happen
if an ^{algorithm}~{\algStop} actually existed that always stopped and
said, about any ^{syntactic expression}~$\frak w$, whether the {\UTM}
would stop or not, when $\frak w$ was written on the tape. If the {\UTM}
stopped, the result of {\algStop} would be {\true}; if it did not stop,
then the result would be {\false}, where $\syn D$ is a ^{dictionary}
with definitions:
$$\algStop \cases{
 {\cal P}_{\syn U}[{\algStop}({\frak w}) : {\syn D}] \rightarrow \true
   & if ${\cal P}_{\syn U}[{\frak w} : {\syn D}]$ would stop; \cr
 {\cal P}_{\syn U}[{\algStop}({\frak w}) : {\syn D}] \rightarrow \false
   & if ${\cal P}_{\syn U}[{\frak w} : {\syn D}] \rightarrow\infty$ .\cr
}$$

If {\algStop} existed, then we could also define another
algorithm~{\algPdox}, in the following way:
$$ {\algPdox}({\frak y}) =
  {\bf if\;} {\algStop}({\frak y})
  {\;\bf then\;} {\it loop\,forever}
  {\;\bf else\;} {\it stop}
  {\;\bf end\;} .$$
To prove that, if {\algStop} were an algorithm, we could define
{\algPdox}, we need to show that in a {\UTM} algorithms that use other
algorithms can be defined. This is not difficult, given their
universality. We also need to show that an {\bf if} can be defined, for
which it is sufficient to see that it is possible to construct a {\TM}
that does this. Let us accept these matters as proven.

What would happen if we wrote the expression
 ${\algPdox}(Z):Z={\algPdox}(Z)$, which is self-referential,
 on the {\UTM} tape?
If the {\algStop} algorithm behaves as we have said, then we only have
to study two possible cases.
\point If ${\algStop}({\algPdox}(Z)): Z={\algPdox}(Z)$ is {\true},
that means, according to the definition of {\algStop}, that the expression
 ${\algPdox}(Z): Z={\algPdox}(Z)$
will make the {\UTM} stop, but, according to the definition of
{\algPdox}, it depends on how the {\UTM} resolves the expression
 ${\algStop}(Z): Z={\algPdox}(Z) \rightarrow
  {\algStop}({\algPdox}(Z)): Z={\algPdox}(Z)$,
that we are supposing is {\true}, so that what it will do is
\latin{loop forever}, that is, not stop.
\point If, on the contrary, ${\algStop}({\algPdox}(Z)): Z={\algPdox}(Z)$
is {\false}, then, according to the definition of {\algStop}, with
${\algPdox}(Z): Z={\algPdox}(Z)$ the {\UTM} should not stop, but,
because it is {\false}, according to the definition of {\algPdox}, the
{\bf else} branch would be executed and it would \latin{stop}.

\noindent What this contradiction means is that the algorithm~{\algStop}
cannot exist as we have defined it and, therefore, that within a
universal language it is impossible to definitively determine whether an
expression is paradoxical or not. And with this we have proven that, in
recursive syntaxes, paradoxes are pertinacious.


\Section Recursive Syntax

To end this visit to the ^{theory of computation}, which we could with
all propriety call the ^{theory of syntax}, we will locate the set of
universal languages_{universal language}, $\{ {\syn L}_{\syn U} \}$, in
the more general set of grammatical languages_{grammatical language},
$\{ {\syn L}_0 \}$.

We can divide the set of grammatical languages~$\{ {\syn L}_0 \}$ in two
other disjoint sets: the set of decidable languages_{decidable
 language}~$\{ {\syn L}_{\syn D} \}$
and the set of undecidable languages_{undecidable
 language}~$\{ {\syn L}_{\syn I} \}$:
$$\eqalign{
 \{ {\syn L}_{\syn D} \} \cup \{ {\syn L}_{\syn I} \} &= \{ {\syn L}_0 \}\cr
 \{ {\syn L}_{\syn D} \} \cap \{ {\syn L}_{\syn I} \} &= \hbox{\O} . \cr}
$$
^[Carroll] and ^[Long]^(Carroll1989) use the notation~${\cal
H}_{\Sigma}$ for the set of decidable languages, ${\cal H}_{\Sigma} = \{
{\syn L}_{\syn D} \}$, which are the languages recognized by the {\TMes}
that always stop, and which they place in the ^[Chomsky] hierarchy
between the non-restricted grammatical languages~$\{{\syn L}_0
\}$ and the context-sensitive languages~$\{ {\syn L}_1 \}$:
$$ \{ {\syn L}_1 \} \subset
   \{ {\syn L}_{\syn D} \} \subset
   \{ {\syn L}_0 \} .$$

Paradoxes cannot be expressed in any \Mental decidable language$L_{\syn
D}$, because all of the expressions_{syntactic expression} are
decidable. On the contrary, paradoxes exist in all of the \mental
undecidable languages$L_{\syn I}$. As paradoxes can also be expressed in
all of the \mental recursive syntaxes$L_{\syn U}$_{recursive syntax}, as
we showed in ^>Reflexive Paradoxes>, it turns out that all of the
\mental universal languages$L_{\syn U}$ are \mental undecidable$L_{\syn
I}$, although not all \mental undecidable languages$L_{\syn I}$ are
\mental universal$L_{\syn U}$, because there are {\TMes} that neither
stop always nor are universal:
$$ \{ {\syn L}_{\syn U} \} \subset
   \{ {\syn L}_{\syn I} \} \subset
   \{ {\syn L}_0 \} . $$

We also find among the \mental undecidable languages$L_{\syn I}$ those
that have errors; for example the language that a {\TM} accepts that
does not stop only when it finds a sequence of one hundred consecutive
ones, due, let's say, to a slip in designing the state that corresponds
to one hundred ones.


\Section Algorithms

In ^>Semantics and Syntax>, we defined a ^{symbolic logic} as the logic
capable of representing problems, resolutions, and solutions, using two
layers: syntax and semantics. Semantics_{semantics} has the solutions
and ^{syntax} has the problems and the resolutions. The resolutions take
the ^{problem} expressed syntactically and transform it syntactically
until the syntactic expression of a ^{solution} is reached. Thus, the
^{resolution} is a syntactic transformation.

We give the name ^|algorithm| to the expression of a syntactic
transformation that, in principle, can be a ^{syntactic expression} or a
meta-syntactic expression. The first alternative takes us (as we saw
from ^>The Chomsky Hierarchy>, to ^>Recursive Syntax>) to \mental
recursive syntaxes$L_{\syn U}$_{recursive syntax}, capable of maximum
^{expressiveness}, but which cannot avoid paradoxes. Paradoxes_{paradox}
are syntactic expressions with no meaning, inconclusive, unresolvable:
that is, they are resolutions that do not achieve a solution. So
^[Russell] tried, with his ^{theory of types}, the second alternative to
avoid the ^{reflexive paradox} that he himself had discovered in set
theory (see ^[Quine]\footnote{_(Quine1940), page~163 and following.}):
 $$R = \{ x?\; x \not\in x\} \Longrightarrow
  ( R \in R \Longleftrightarrow R \not\in R ) .$$
But the resulting ^{typed logic} is complicated, which induces error,
limits the logic's expressiveness, and, what is worse, requires an
unlimited number of meta-meta-\dots-syntaxes, so that it is, itself,
paradoxical.

It is better to tolerate paradoxes and use recursive syntaxes, which are
simpler and more expressive because, in practice, neither the ^{tape} of
the {\TM} is infinite, nor is the ^{time} available to reach a solution
unlimited, so that the resolutions that require more tape than is
available, or too much time, even though they would be capable of
reaching a solution in other circumstances, are as useless or harmful as
paradoxes. All in all, paradoxes are no worse than bad resolutions.


\Section Reason

The {\inquirers} job is to describe with the greatest possible precision
the problem that the {\subject} faces. Reason~$\syn R$ receives the
representation of the \Mental problem$X$ from the {\inquirer}; this
representation is necessarily syntactic. And {\reason} produces, as a
result, the representation of the resolution, that is the ^{algorithm}
that the {\mind} has to execute, which is also syntactic. So, if
{\reason} should have the maximum generality, as we concluded in ^>The
Symbolic Subject>, then {\reason} must be a ^{syntax engine}, or
universal processor~${\cal P}_{\frak U}$, capable of transforming any
^{syntactic expression} into any other one, as we saw in ^>The Universal
Turing Machine>:
$${\frak R}={\cal P}_{\frak U} .$$
This means that the {\subjects} ^{internal logic} or at least the
internal logic of its {\reason}, is a symbolic logic, with semantics and
\Mental recursive syntax$L_{\syn U}$.

Finally, the {\mind} has to apply the resolving algorithm proposed by
{\reason} and obtain the solution, in the form of behavior that the
{\body} can execute so that, if everything has gone well, it will
effectively solve the problem that the {\subject} is facing.

In the semantics of the ^{symbolism} of the {\subjects} {\reason}, the
following have ^{meaning}: the symbols that refer to the {\minds}
resources, such as the {\bodys} behaviors, the {\modelers} models, the
forecasts_{foresight} of {\reality}, and the {\simulators} simulations;
and the symbols that refer to the conditions that {\reason} can
consider, as in the goodness or badness of the behaviors and the
forecasts. Other symbols represent concepts
     \vadjust{\penalty-1000}%
of the ^{theory of the problem}, such as, for example, the ^{freedom} of
the problem, represented as `?', which has no meaning, because it
indicates to the {\reason} that the syntactic expression in which it is
found is open, that is, its meaning is not closed, and that, as a
result, it is not ready to be passed on to the {\mind}.


\Section Comparing Selves

We have completed a circle in the ^{exit path}. We started from the
^{self}, and went beyond the self when we discovered, in _>I Am Alive>,
that the self is not all of the ^{subject}, but that the self is only
the ^{problem of the subject}. The subject is alive, it is part of
^{life}, so that the problem of the subject is part of the ^{problem of
survival} that defines life. The problem of survival is an ^{apparent
problem} that, specified in ^{time} and in ^{space}, we formalized in
_>Formalizing the Apparent Problem>, and we resolved it. And now the
resolution of the formalized apparent problem takes us to the
{\subject}, and finally back to the {\self}. But, in order to complete
the circle accurately, we need to establish that the formal {\self} that
we found coincides with that original self that we started out from at
the beginning, in _>What Am I?>, and that we defined as ^{freedom} to
not die, in _>I Am Freedom to Not Die>, and that was syntactic and
paradoxical, as we saw in _>I Am in Syntax> and _>I Am Paradoxical>.

The formal {\subject} found coincides with the original subject, since
we only had two pieces of information about this original subject (see
^>I Am Alive>), and in both pieces it coincides with the formal subject.
We knew that the original subject, like the formal {\subject}, has a
^{symbolic logic}. And we also knew that the problem of the original
subject is its original self, just as the problem of the formal
{\subject} is its formal {\self}; it is the same whenever both selves
coincide, and we will prove this now.

For the formal {\subject}, the formal {\self} turns out to be the best
representation that the {\subject} itself can make of the problem that
it is facing, which is the apparent problem of survival. As freedom is
inherent to every problem, and the final condition of the problem of
survival is to live, then the formal {\self} could be defined in the
same way as we defined the original self, that is, as freedom to not
die.

The formal {\self} is also syntactic, since any representation of a
problem has to be syntactic, as we already know (see ^>Semantics and
Syntax>), because it must represent freedom, which is tautologically
free of meaning.

In ^>The Formal Subject>, we established that the condition of the
inquirer cannot be conclusively verified, that is, that the
representation of an apparent problem, given its nature, is never
definitive. That is why the formal \Mental problem of the subject$X$
must always be kept open to revision. As a changing problem cannot have
a definitive solution, we have proven that the formal {\self} is
paradoxical.


\Section Comparing Subjects

In the previous section we compared subjects, and selves, from the exit
path, and so we have not yet compared the subjects of the exit path with
their homonyms from the entry path.

The essential characteristic of the formal {\subject} is its capacity to
^{reason}, that is, to foresee_{foresight} the result of the different
ways of resolving a problem. A symbolic logic in which to represent
problems and resolutions, as well as solutions, is indispensable for
this. The evolutionary subject, seen from _>Words> to _>The Subject's
World Is Symbolic>, had a symbolic logic available, so it had to be
considered as a specific case of the formal {\subject}, with the
peculiarities derived from its Darwinian heritage. We must remember that
the evolutionary subject's ^{syntax} rested upon a ^{reality} of things
with ^{meaning} inherited from the knower, while the only thing we can
affirm about the {\subjects} syntax is that it rests upon a ^{semantic}
layer inherited from the {\knower} where the solutions should be. Thus,
while in evolutionary syntax we find sentences_{sentence} with
nouns_{noun}, verbs_{verb}, adjectives_{adjective}, and
pronouns_{pronoun}, we know that formal syntax must be
recursive_{recursive syntax}, ${\syn L}_{\syn U}$. A recursive syntax
has a maximum of expressiveness but also pertinacious paradoxes. And we
know that to deal with such a syntax, we need a {\UP}, or ^{syntax
engine}, that is the processor of a {\UTM}.


\Section The Subjective Loop

Having reached the end of the ^{exit path}, we can now make some global
appraisals. The exit path is circular_{subjective loop}. It starts from
the ^{self}, defined as a ^{problem}, then goes to the ^{subject} and,
from here, to ^{life}, defined as an ^{apparent problem}, where the
first part, definitely the exit, finishes. But the ^{resolution} of the
apparent problem of survival, which goes through five stages, ends at
the {\subject} with its {\self}, so that this second part of the exit
path returns to the beginning, as we showed in ^>Comparing Selves>, and
runs in the same direction as the entry path.

$$\vbox{\everycr={}\tabskip=0pt \lineskip=0pt
  \halign{&\hfil#\hfil\cr
   Self& $\,\rightarrow\,$& Subject$\,$& \rightarrowfill&
    $\,$Problem of Survival\cr
   $\parallel$& &$\parallel$&  &$\parallel$\cr
   \frakx X& $\leftarrow$& ${\cal A}_4$& $\leftarrow$%
    ${\cal A}_3$$\leftarrow$${\cal A}_2$$\leftarrow$%
    ${\cal A}_1$$\leftarrow$${\cal A}_0$$\leftarrow$&
    Apparent Problem\cr}}$$

\breakif2

And so the circle is completed, and now we can contemplate as a whole
the fundamental postulate of the ^{theory of subjectivity}, in other
words, that life, or the ^{problem of survival}, which is the same
thing, is an ^{apparent problem} (see ^>The Problem of Survival>). We
will make use of a general observation that applies to any theory that
uses problems for explanations: if any concept corresponds with an
apparent problem, it must be a primitive concept that the theory leaves
undefined, because it cannot provide any information about it. From this
observation, we can see that the need to use problems as explanations as
well as the election of ^{life} as the primitive undefined concept of
the theory of subjectivity are both consequences of defining the self as
freedom to not die. Because the self, being freedom and condition, is a
problem; and being the condition of not dying, it remits us to life. It
remits us to life because the self, as it is already defined, cannot be
the primitive undefined concept that life, ineffable, can be.

But because the path is circular, whoever wishes to can go backwards on
it. If the self is explained starting from the apparent problem of
survival, then the consequence is that the self must be defined as
freedom to not die.

On the other hand, the resolution of the apparent problem reach\-es the
{\subject} with its {\self}, which constructively proves that a
problematic theory is sufficient to explain the nature of the subject
and of the self.


\Section Levels

The five stages of the resolution of the apparent problem can be grouped
in three levels. The {\mechanism} is nothing more than the starting
point and forms, by itself, the reference level. The following level, to
which the {\adaptor} and the {\learner} belong, appears with the
{\body}, which is capable of behaving as various \corporal
mechanisms$A_0$ and, in this way, we can say that it includes the
previous reference level (\EPA2).
$$\hbox{Adaptor $\aut A_1$}\llave{
   Governor $\aut G$\cr
   Body $\aut B$\lopen\langle{Mechanism $\aut A_0$}}$$

\vfill\break

\noindent Finally, the third level, in which the {\knower} and the
{\subject} are to be found, originates in the {\mind} which, being
capable of resolving problems like the \corporal learners$A_2$, like the
\corporal adaptors$A_1$, and like the \corporal mechanisms$A_0$, contains
the two levels that precede it (\EPA6).
$$\hbox{Knower $\aut A_3$}\llave{
   Intelligence $\syn A$\cr
   Mind $\syn M$\lopen\langle{
    Learner $\aut A_2$\cr
    Adaptor $\aut A_1$\cr
    Mechanism $\aut A_0$}}$$

Within each of the two-stage levels, the first stage comprehends the
previous level and the second stage interiorizes it. Thus, the
{\learners} {\simulator} contains interiorizations of the {\bodys}
behaviors and of the exterior, and that is why the {\modelers} task is
to compose {\reality}, which is, precisely, the interior representation
of the exterior behavior (\EPA3).
$$\hbox{Learner $\aut A_2$}\llave{
   Governor $\aut G$\llave{
    Modeler $\aut M$ $\to$ Reality $\aut R$\cr
    Simulator $\aut S$ [Body $\aut B$]}\cr
   Body $\aut B$\lopen\langle{Mechanism $\aut A_0$}}$$
And, similarly, the {\subjects} {\reason} has syntactic and recursive
representations of its {\minds} resolutions and of the exterior problem.
The {\inquirers} task is to look for the problem that it is facing, that
is, the problem of the subject, or the {\self}, which is, precisely, the
internal representation of the external problem (\EPA7).
$$\hbox{Subject $\aut A_4$}\llave{
   Intelligence $\syn A$\llave{
    Inquirer $\syn I$ $\to$ Self $\syn X$\cr
    Reason $\syn R$ [Mind $\syn M$]}\cr
   Mind $\syn M$\lopen\langle{
    Learner $\aut A_2$\cr
    Adaptor $\aut A_1$\cr
    Mechanism $\aut A_0$}}$$

It turns out that the {\subject}, which culminates the resolutive
process of the apparent problem, contains it completely. Because the
{\mind} is interiorized in the {\reason}, and the mind includes all of
the preceding stages, and therefore also contains the {\simulator},
which interiorizes the {\body}.


\Section Layers

Symbolism's construction in two layers (seen in ^>Semantics and Syntax>,
and in ^>The Symbolic Subject>), which is at the origin of the
resolution levels of the apparent problem, allows us to discover some
correpondences in the subject's ^{world}, presented in ^>The World>. The
{\body} corresponds in the layer of behavior, which is the layer of
solutions and which we call semantics, to the {\mind} in the layer of
resolutions, which we call syntax. And the {\self} corresponds to
{\reality}.
$$\vbox{\def\uc#1{\relax\uppercase{#1}}
 \def\:#1#2:#3#4.{\uc#1#2_{#1#2}& $\equiv$&\uc#3#4_{#3#4}\cr}
 \def\;#1 (#2);#3 (#4);{#1_{#2}& $\equiv$& #3_{#4}\cr}
 \setbox0=\hbox{$\mathop{\cal A}\nolimits_4$ }
 \def\.#1#2 $#3#4$.#5#6 #7$#8#9$.{\uc#1#2_{#1#2}%
  \hbox to\wd0{\hfil$\mathop{\cal #3}\nolimits#4$}&
  $\equiv$&
  \hbox to\wd0{$\mathop{\frak #8}#9$\hfil}\uc#5#6_{#5#6} #7\cr}
 \halign{&\hfil#&\hfil\quad#\quad\hfil&#\hfil\crcr
 \.adaptor $A_1$.knower {$\aut A_3$}$E_1$.
 \.governor $G$.intelligence $A$.
 \.body $B$.mind $M$.
 \.learner $A_2$.subject {$\aut A_4$}$E_2$.
 \.modeler $M$.inquirer $I$.
 \.simulator $S$.reason {${\cal P}_{\syn U}$}$R$.
 \.reality $R$.self $X$.
 \:semantics:syntax.
 \:behavior:problem.
 \:solution:resolution.
 \:program:algorithm.
 \:thing:concept.
 \:practice:theory.
 \:finite:infinite.
 \:physics:metaphysics.
 \:data:information.
 \:change:permanence.
 \;Res Extensa (res extensa);Res Cogitans (res cogitans);
}}$$


\Section The World Is an Enigma

The {\inquirer} occupies the highest place in the {\subject} which, in
turn, is the peak of the resolutive process of the apparent problem. And
the {\inquirers} task is to ask questions, the first being `what am I?'
This may be why ^[Aristotle]^(Aristotle-IV) began his ^<Metaphysics> by
declaring that ``all men by nature ^{desire} to know''.

The subject is curious_{curiosity} because he understands the world and
its situations as problems to be resolved. For the subject, the world is
an enigma.


\endinput
