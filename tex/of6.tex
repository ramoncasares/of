% OL6.TEX (RMCG20000731)

\Part Conclusion


% Ideograma de camino (tao) pequeño
\MTbeginchar(9pt,12pt,0pt);
 \MT: save border, ladderw, ladderh, gap;
 \MT: border = w/10; gap = h/20;
 \MT: ladderw = 3w/8; ladderh= h/2;
 \MT: % pickup thick_pen;
 \MT: pickup pencircle xscaled0.75pt yscaled0.4pt rotated 30;
 \MT: z1bl = (w/2,h/8); % One for the ladder
 \MT: z1tr - z1bl = (ladderw,ladderh);
 \MT: z1tl - z1bl = (0,ladderh);
 \MT: z1br - z1bl = (ladderw,0);
 \MT: draw z1tl .. z1bl; draw z1tr .. z1br;
 \MT: draw z1tl .. z1tr; draw z1bl .. z1br;
 \MT: z1mo = 2/3[z1bl,z1tl]; z1md = 2/3[z1br,z1tr] - (gap,0);
 \MT: z1no = 1/3[z1bl,z1tl]; z1nd = 1/3[z1br,z1tr] - (gap,0);
 \MT: draw z1mo .. z1md; draw z1no .. z1nd;
 \MT: z2o = (0,h/8); z2d = (w,0); % Two for the base
 \MT: x2m1 = w/4; x2m2 = 3w/4; y2m1 = y2o; y2m2 = y2d;
 \MT: draw z2o{right} .. z2m1 .. z2m2 .. z2d;
 \MT: y3ao = 4/3[y1bl,y1tl]; % Three for the crown
 \MT: x3ao = x1tl - border; z3ad = (w,y3ao);
 \MT: draw z3ao .. z3ad;
 \MT: z3eo = z1tl+(0,gap); z3em = 1/2[z3ao,z3ad]; z3ed = (w-border,h);
 \MT: draw z3eo .. z3em .. z3ed;
 \MT: z3io = (x3ao,y3ed); z3id = 1/2[z3ao,z3ad];
 \MT: draw z3io .. z3id;
 \MT: x4a = x4i; y4e = 1/2[y4i,y4a]; % Four for the circumflex
 \MT: z4e = 1/2[z3ao,z3ed] - (3w/8,gap);
 \MT: y4a - y4i = h/8; % circumflex height
 \MT: x4e - x4a = w/8; % circumflex width
 \MT: draw z4a -- z4e -- z4i;
 \MT: x5e = x1tl - w/8; % Five for the seven; seven x position
 \MT: x5a = x5u; x5e - x5a = w/5; % seven width
 \MT: x5i = 1/3[x5a,x5e]; x5o = 2/3[x5a,x5e];
 \MT: y5a = y1tl - gap; y5u = y1bl; % seven y position
 \MT: y5a = y5e; y5i = 2/3[y5u,y5a]; y5o = 1/3[y5u,y5a];
 \MT: draw z5a .. z5e; draw z5e .. z5i .. z5o .. z5u;
 \MT: %pickup thin_pen; % The bounding box
 \MT: %draw (0,0) -- (w,0) -- (w,h) -- (0,h) -- cycle;
\MTendchar;
\newbox\taobox \setbox\taobox=\hbox{\raise-1.5pt\box\MTbox}
\def\tao{\copy\taobox}

\Section Tao

The ^{Chinese} ideogram \tao, which sometimes is transcribed `^{tao}',
other times `^{dao}', or even, in ^{Japanese}, `d\=o', means ^{path},
and also method and law. It is the central concept of ^{Taoism}^(LaoZi),
from which it takes its name, and it is fundamental to ^{Buddhism} as
well.

In general, oriental ^{philosophy} is introspective, because it starts
out observing the pure ^{self}. And because it attends solely to the
self, it finds absolute ^{nothingness}, that is, it finds complete
^{freedom} without limits, in which it is even possible to reconcile
opposites, to reconcile yes and no (see ^>I Am Paradoxical>). This is
why the dialectic of \latin{yin} and \latin{yang} is so important, or in
^{Zen} Buddhism, paradoxes are important (see ^[Suzuki]^(Suzuki1934)).

Although later developments of Buddhism and Taoism differ from the exit
path of this ^{theory of subjectivity}, the starting point of the three
paths is the same: self. But if we only see the freedom of the self,
which is its will, then it is impossible to go beyond it. Because if
everything is freedom, if freedom is absolute, then there are no limits,
and you cannot go beyond something that doesn't have limits. Recognizing
limits is enough to make a situation completely different. And we have
already seen where defining the self as freedom to not die takes us.
% Ideograma de camino (tao)
\MTbeginchar(30pt,40pt,0pt);
 \MT: save border, ladderw, ladderh, gap;
 \MT: border = w/10; gap = h/20;
 \MT: ladderw = 3w/8; ladderh= h/2;
 \MT: % pickup thick_pen;
 \MT: pickup pencircle xscaled1.4pt yscaled0.7pt rotated 30;
 \MT: z1bl = (w/2,h/8); % One for the ladder
 \MT: z1tr - z1bl = (ladderw,ladderh);
 \MT: z1tl - z1bl = (0,ladderh);
 \MT: z1br - z1bl = (ladderw,0);
 \MT: draw z1tl .. z1bl; draw z1tr .. z1br;
 \MT: draw z1tl .. z1tr; draw z1bl .. z1br;
 \MT: z1mo = 2/3[z1bl,z1tl]; z1md = 2/3[z1br,z1tr] - (gap,0);
 \MT: z1no = 1/3[z1bl,z1tl]; z1nd = 1/3[z1br,z1tr] - (gap,0);
 \MT: draw z1mo .. z1md; draw z1no .. z1nd;
 \MT: z2o = (0,h/8); z2d = (w,0); % Two for the base
 \MT: x2m1 = w/4; x2m2 = 3w/4; y2m1 = y2o; y2m2 = y2d;
 \MT: draw z2o{right} .. z2m1 .. z2m2 .. z2d;
 \MT: y3ao = 4/3[y1bl,y1tl]; % Three for the crown
 \MT: x3ao = x1tl - border; z3ad = (w,y3ao);
 \MT: draw z3ao .. z3ad;
 \MT: z3eo = z1tl + (0,gap); z3em = 1/2[z3ao,z3ad]; z3ed = (w-border,h);
 \MT: draw z3eo .. z3em .. z3ed;
 \MT: z3io = (x3ao,y3ed); z3id = 1/2[z3ao,z3ad];
 \MT: draw z3io .. z3id;
 \MT: x4a = x4i; y4e = 1/2[y4i,y4a]; % Four for the circumflex
 \MT: z4e = 1/2[z3ao,z3ed] - (3w/8,gap);
 \MT: y4a - y4i = h/8; % circumflex height
 \MT: x4e - x4a = w/8; % circumflex width
 \MT: draw z4a -- z4e -- z4i;
 \MT: x5e = x1tl - w/8; % Five for the seven; seven x position
 \MT: x5a = x5u; x5e - x5a = w/5; % seven width
 \MT: x5i = 1/3[x5a,x5e]; x5o = 2/3[x5a,x5e];
 \MT: y5a = y1tl - gap; y5u = y1bl; % seven y position
 \MT: y5a = y5e; y5i = 2/3[y5u,y5a]; y5o = 1/3[y5u,y5a];
 \MT: draw z5a .. z5e; draw z5e .. z5i .. z5o .. z5u;
 \MT: %pickup thin_pen; % The bounding box
 \MT: %draw (0,0) -- (w,0) -- (w,h) -- (0,h) -- cycle;
\MTendchar;
$$\box\MTbox$$


\Section The Paths

The ^{exit path}, which gives priority to ^{introspection}, is the
^{oriental} ^{path}. It presupposes the ^{subject} and its
solipsist_{solipsism} schools evade the object. The ^{entry path}, which
prefers the ^{phenomenon}, is the ^{occidental} path. It takes the
^{object} for granted and its materialist_{materialism} versions do
without the subject.

If oriental philosophies cannot go beyond the unlimited ^{freedom} of
the pure self, occidental philosophies cannot enter into the freedom of
the self. Thus, for example, occidental natural ^{philosophy}, or to
abbreviate, ^{science}, exclusively explains the physical phenomena that
can be repeated under experimental conditions. In this way, it
proscribes freedom, although it can predict_{prediction} what the result
of a physics experiment will be with precision and profitably. This is
why present-day science cannot explain the self.

The oriental path is insufficient, and the occidental path is, too. One
has freedom without condition, and the other has condition without
freedom. Synthesis is a problem, precisely because the synthesis of
freedom and condition is a problem. This is why only the circular path
is complete: the circular path is the ^{subjective loop} (seen in ^>The
Subjective Loop>), that starts out from the ^{self} and goes beyond it
to arrive at the ^{apparent problem} of survival, which is the source of
meaning, as we will state in _>The Problem Is the Source of Meaning>,
and that, returning to the self, gives meaning to each of its stages
and, finally, to the very self.


\Section The Problematic Explanation

Thinking about recent history can help overcome the explanatory problems
of present-day ^{science}. The ^{material explanation}, seen in ^>The
Material Explanation>, that ruled in science until the quantum
revolution, explains by thing-ifying_{thing-ify}, that is, it keeps
explaining concepts until it reaches things, because things are what
have a natural meaning (see ^>Things and Concepts> and ^>Semantics and
Syntax>). The ^{automatic explanation}, discovering problems in
thing-ifying (see ^>The Automatic Explanation>), does away with the
meanings of the ^{knower} and tries to avail itself solely of the
learner's_{learner} foresights_{foresight}. This step backward avoids
using the inexact, unreasoned models achieved during evolution (see
^>Knowledge Is Provisional>), but it produces explanations with no
meaning, that is, descriptions. True progress is achieved not by going
backward from the knower to the learner, but by advancing to the
^{subject}. This progress needs, then, a theory of the subject. And this
^{theory of subjectivity} works.

\breakif1

The proposal is, first, that science give up the automatic explanation,
that is, that its final products stop being automata, which are systems
of data that predict_{prediction} mechanically, and second, that science
adopt the ^|problematic explanation|, so that science produces, as a
consequence, problems, resolutions, and solutions. The solutions can be
formally indistinguishable from the final products that science produces
today, that is, automata, whether they be in the form of differential
equations or any other; but by framing themselves explicitly in a
^{problem}, that in the end must be the ^{problem of survival}, they
will achieve a ^{meaning} that they do not have at present.


\Section Scientific Theories

This is, of course, one way of overcoming the enormous obstacle that
^[Gödel] discovered at the foundations of ^{mathematics}, and that all
^{science} shares.

^[Gödel]^(G\"odel1931) showed, with his ^{undecidability} theorem, that
there are undecidable propositions in any theory that includes
^{arithmetic}; these are propositions whose truth or falsehood the
theory itself cannot determine, that is, paradoxes_{paradox}. The
corollary for ^{physics} is immediate: physics includes arithmetic and,
therefore, cannot be complete. This same argument can be applied
immediately to all the theories that use arithmetic, and can be
generalized, following ^[Turing]^(Turing1936) (see ^>Reflexive
Paradoxes>), to all theories that have to be expressed in recursive
symbolisms, so that all the sciences are affected. But the case of
physics is especially surprising because, in spite of the famous
mathematical demonstration in the year 1931, there are still important
physicists, such as ^[Hawking]^(Hawking1988), who think that a ^{unified
theory} that describes the universe completely can exist.

On the other hand, for the ^{problematic explanation}, scientific
^{knowledge} remains subordinated to life. ^[Gödel]'s undecidability
theorem is nothing more than the confirmation that mathematics and the
sciences necessarily inherit the paradoxical nature of ^{life} and of
the ^{self} (seen in ^>Life Is Paradoxical> and in ^>I Am Paradoxical>).
So we already know that it is the paradoxical natures of the self, of
life, of knowledge, and of science, that sustain their problematic
character and keep all of these matters inconclusive and incomplete,
that is, open, free, and alive.

It is ^{absurd} to believe that everything that happens is expressible
symbolically and that this expression can completely exhaust the
^{phenomenon}, because it has been proven mathematically that it is
impossible.


\Section The Subject Is Free

For the ^{automatic explanation}, the set of differential equations that
constitutes the theory is all that can scientifically be said about the
phenomena that this theory describes. The automatic explanation is only
good for the ^{entry path}, that is, it is useful to explain
phenomena_{phenomenon}  exterior to the ^{self}, but it is incapable of
turning the interior self into a phenomenon. This is because freedom is
completely excluded from the automatic explanation and, consequently,
the automatic explanation cannot explain the self, which is free; a
conclusion we already reached in ^>Is the Subject Free?>.

Then we asked ourselves if a ^{mechanism} is free_{freedom}. The answer
was no; because the mechanism is the prototype of determinism, it would
not be free. The ^{adaptor} is not free either, we reasoned, because it
is a mechanism. And the ^{learner}, which is an adaptor, cannot be free
either. The ^{knower}, whose virtue is that it is capable of imitating
the learner, the adaptor, and the mechanism, cannot be free either. And
the ^{subject}, we wondered, can a subject be free? According to the
automatic explanation of the entry path, (see ^>Mechanisms>), the answer
was no, because the subject is still a mechanism. Specifically, it is
the mechanism with a nervous system that models reality, a reality that
can be used in various ways according to its feelings, and that has a
symbolic logic with which it can reason.

The situation is very different for the ^{problematic explanation}. It
is true that a mechanism is not free, but it turns out that a living
mechanism is not a whole but a part. A living mechanism is a resolver of
the ^{apparent problem} of survival, and it has no meaning by itself
apart from the problem; that is, it must be defined as part of the
problem. Thus, although the live mechanism may be impermeable to the
freedom of the apparent problem, which it is, it forms part of a system
that necessarily includes freedom. And, in this sense, the sequence that
goes from the mechanism to the subject is a process in which the freedom
of the apparent problem of survival soaks into its resolvers. This
process reaches its culmination with the subject, because the subject is
capable of interiorizing the problem itself and, with the problem, its
freedom. The subject seizes the problem's freedom. The subject is free.

The subject is free, yes, but its freedom is not absolute, it is
conditioned. In other words, freedom is part of the problem. This means,
on one hand, that the ^{future} is not determined but open, because it
depends on decisions made freely; on the other hand, it means that not
all decisions are equally good or bad. That is, it means that the
subject's free decisions have diverse consequences and, as we saw in
^>Knowledge Is Provisional>, these consequences are never entirely
predictable_{prediction}. If things were any other way, life would be
easier.


\Section Consciousness and Self-Consciousness

For the problematic explanation, the ^{subject} is not a ^{mechanism},
simply because the subject is free and a mechanism is not. Adding
^{freedom} makes easy what without it is impossible. The same thing
happens with consciousness and self-consciousness.

In the entry path, specifically in ^>Consciousness>, when we pointed out
the distinction made between sensing and seeing_{sight}, we defined
consciousness as that which the subject sees of the world. Along these
same lines, self-consciousness would be how the subject sees itself. And
as the theory of subjectivity shows us that the subject is a resolver of
the apparent problem of survival, we can make these definitions even
more precise.

Self-consciousness_|self-consciousness| is the faculty thanks to which a
^{resolver} represents itself. And ^|consciousness| is the faculty by
which a resolver represents the complete situation to itself, that is,
the ^{problem} that it is facing and itself as a resolver. It follows
from these definitions that only subjects can have consciousness and
self-consciousness, because they are the only resolvers that, because
their logic is ^{symbolic logic}, can represent problems as well as
resolutions. In other words, in order to see a whole thing, a certain
distance is necessary; and if the thing is one's own situation, then one
needs a mirror, that is, reflection, in order to achieve this distance.

Therefore the subject is conscious when it sees itself as a problem,
that is, when the subject sees itself as its self, and knows itself to
be free. And the subject is self-conscious when it notices that its
problem is unresolvable, because it is paradoxical and has no solution,
when the subject knows itself to be mortal.
\par\breakif3\beginpoints
$$\vbox{\halign{&\hfil#\hfil\cr
  I &know that I &am & free & and & mortal\cr
  I &&am & conscious & and & self-conscious\cr}}$$
\endpoints

These are easy conclusions for the ^{problematic explanation}, but they
are impossible for the ^{automatic explanation} or for the ^{material
explanation}.


\Section Knowledge Is Cumulative

The ^{subject} represents to itself the problem that it is facing. The
problem the subject faces is an ^{apparent problem}; in other words, it
is a problem that does not allow a definitive representation. Therefore,
the representation that the subject makes to itself of the apparent
problem is changing; we have called this representation the ^{problem of
the subject}. When the problem of the subject changes, its resolution
and solution change as a consequence. Translating from resolution to
evolution, as we already did in ^>Knowledge Is Provisional>, this means
that all knowledge is provisional.

And because no knowledge is certain, it is not certain either that new
knowledge is definitely superior to previous knowledge, so just in case,
it is better not to forget any knowledge. Surely it is because of this
that human ^{knowledge} is cumulative, and we forget only occasionally
with disuse and ^{time}. This is the ^{problematic explanation} of why
we cannot ^{forget}, especially when we want to forget.


\Section Intent

Many other comparisons can be made between the ^{problematic
explanation} and the other explanations. You are free to test, with
whatever subject interests you the most, which explanation seems most
accurate. Here we will continue to investigate the most meaningful
matter: meaning.

Meaning appears with the ^{knower}, capable of using reality in
different ways, in order to choose from among them the way that best
serves him. Meaning is useful, in short, in order to be able to use
reality according to his own interests, or, to put it another way,
^{meaning} is the means that the knowers use to integrate external
conditions with internal needs (see ^>Meaning>).

Given that the subjects know that they are facing a problem, they have a
conscious purpose, to solve the problem, which confers ^{intent} upon
everything that they consciously do. They do not do things just to do
them; rather they have an end in view, marked out by the problem that
they know they are facing. This is why their conscious ^{behavior} is
intentional.

Thus, working upon the knower with its meanings, evolution
\hbox{designed} the ^{subject} with its intentions. And if meanings
served the knower for using reality according to its own interests, they
will also serve the subject for doing so according to its intentions.
The difference is that, while the subject sees_{see} the meanings, or is
conscious of the meanings and, because of this, can use them with
intent, the knower uses the meanings, but cannot see them. The subject's
intentions are conscious, that is, they can be expressed, while the
knower's interests cannot.


\Section The Homunculus

As ^[Searle]^(Searle1992) points out, if we study man's_{man} cognition
while rejecting ^{meaning} and ^{intent}, then all of the calculations
that the ^{nervous system} carries out mechanically must be so that in
the end a ^{homunculus} makes the conscious decisions. We still have to
explain this homunculus' cognition, who, in order to make conscious
decisions, must use meanings that direct its intentions. But then
explaining the homunculus' cognition is the same as explaining human
cognition itself, so that we may as well skip this homunculus that
doesn't explain anything anyway. If you are interested in other opinions
on this difficulty, you can consult, among others,
 ^[Dennett]^(Dennett1991),
 ^[Edelman]^(Edelman1992), or
 ^[Crick]^(Crick1994).

^[Searle] is right and there are only two solutions. One is to deny
^{consciousness}, that is, to label it illusory. But here we prefer
neither to reject meaning nor intention. Although there is not a
homunculus in the brain, there is a ^{mirror}, as we saw in
^>Consciousness>.


\Section The Theory of Subjectivity

As we are finding out, we can licitly use purposes, ends, intentions,
and meanings in the problematic explanation. Just the opposite happens
with the automatic explanations, which do not admit final, or
teleological_{teleology} reasonings and which, consequently, cannot
accommodate meanings at all. The incoherence of an explanation with no
meaning is patent, and this justifies the uneasy cohabitation of
automatic explanations with material explanations that use natural
meanings, that is, things and their meanings.

Let us take the ^{electron} as an example. Although orthodox physics
provides us with the wave equation that describes its behavior as the
sole explanation, because it avoids paradoxes_{paradox} in this way, the
mere fact that the electron is described with its own equation, reveals
that it is still being treated like a thing, or more exactly, following
^[Bohm]\footnote{_(Bohm1951), page~118.}, like an object that is neither
a classic particle nor a wave, but that presents the properties of one
or the other depending on the circumstances.

Straining the situation, only the ^{universe} wave equation completely
avoids paradoxes. But if ^{physics} had to limit itself to this
equation, it would be completely useless, due to the impossibility of
presenting it and even more of resolving it, although the solution is,
quite literally, before our eyes: we only have to open them to see it in
all its splendor.

The ^{theory of subjectivity} unravels the situation by introducing the
^{subject}. Men are subjects, but not any kind of subjects; we humans
are the subjects that evolution has found among the primates. And, as
subjects, meanings_{meaning} serve to ease the relationship between the
exterior conditions and the interior appetites. Besides, because of our
genetic heritage, our meanings fit things_{thing} easily and, with a bit
more difficulty, they fit concepts_{concept}, too. That is why we stick
to the electron as a thing. We need our theories to refer to some object
in order to hang meanings onto them, because it is thus, and in no other
way, that we are able to understand.

What the theory of subjectivity reveals to us in the end is that the
equations depend on how we the subjects are constructed, because even
the existence of the hypothetical objects that we call electrons depends
on the subject that gives them meaning. The ^|theory of subjectivity| is
a subjectivist theory of the subject.


\Section A Priori Synthetic Knowledge

Although, according to the ^{theory of subjectivity}, the subject
constructs the ^{world}, it does not do so upon a blank sheet of white
paper. Let us see why.

The ^{problem of survival}, which defines ^{life}, is an ^{apparent
problem}, or, to put it another way, in the beginning life had no
^{information} about what was good for it and what was not (see ^>The
Problem of Survival>). But now life is a set of living beings, and each
living being is a resolver of the problem of survival that has received
^{information} from other resolvers that have gone before.

Thus, the ^{learner}, capable of making models of the ^{environment},
has \latin{a priori} ^{knowledge} of the exterior. Because its
^{genetic} heritage, what makes it a learner instead of something else,
is also information about the exterior. It informs the learner, in the
purest etymological sense, that modeling the exterior environment helps
it to survive. This shows that the aprioristic ^{categories} that
^[Kant]\footnote{_(Kant1787), page 168 of the 1787 edition} thought
necessary, preceding as he did ^[Darwin], are contingent, as any fruit
of Darwinian ^{evolution} is.

In our own case, the ^{syntax engine} that makes us subjects_{subject}
(see ^>Reason>), that ^[Chomsky] calls ^{universal grammar}, is
aprioristic, that is, we human subjects inherit it genetically.

In order to save ^{empirism}, all that is needed is for
^[Locke]'s^(Locke1690) blank sheet of white paper to refer to the
knowledge that life had when it appeared, not to the knowledge a man has
when he is born. In our jargon: the problem of survival is an apparent
problem, but the ^{problem of the subject} is not.


\Section Logicism

It seems to us, as it did to ^[Locke], that we are capable of thinking
anything, no matter how distant it may be from what we habitually
experience. But be careful, because this is deceptive for two reasons.

\beginpoints
\point What is unimaginable cannot be imagined_{imagine}; therefore,
whatever our imaginative limitation is, we think that everything is
imaginable because, I repeat, it is not possible to imagine the
unimaginable. In logical terms, we cannot represent to ourselves what is
not representable in our ^{internal logic}, as
^[Wittgenstein]^(Wittgenstein1922) revealed to us.

\point Our symbolism does not work directly upon the data captured by
our senses, but it works taking things_{thing} with ^{meaning} as data.
That is, with respect to the data upon which the symbolic apparatus
works, things with meaning, our knowledge is genetically coded.

\noindent We will develop the second point in the following section,
_>Objectivism>. We will develop the first point here.
\endpoints

When we presented the ^{apparent problem} in ^{automatic algebra} (see
^>Formalizing the Apparent Problem>), we wrote that the {\universe}
could be any universe, $\forall\aut U$. We expressed, with this
proclamation, that in the apparent problem no ^{information} is
available \latin{a priori} about what is external. But even though we
may proclaim this, our own limitation makes us incapable of considering
those universes that we are incapable of imagining. How could it be
otherwise? This also happens in the formal presentation of the apparent
problem, since the {\universe} cannot be just anything, but, because we
employ automatic algebra as logic, it must be a finite automaton,
$\forall\aut U \in \{ \aut A \}$.

Because of our lack of imagination, we provide information even when we
are trying not to do so. This is what I call ^|logicism| (see ^>The
Problem of the Learner>). Any formalization of the apparent problem
necessarily is guilty of logicism (see \EPA{4.5}, where I called it
^{essentialism}).

By calling the logic in which an apparent problem is represented
^{external logic}, we can extract some of the consequences of logicism.
External logic must be such that it allows the representation of the
apparent problem itself as well as its possible resolutions and
solutions (see ^>Automatic Algebra>). These are exactly the same
conditions required of the subject's internal logic (see ^>The Symbolic
Subject>, and ^>Reason>). This implies that the external logic of a
particular apparent problem is always good for the internal logic of a
subject of this same problem, and \latin{vice versa}, so that both
logics can be the same. Thus, when a subject presents its own
problematic situation, the best thing it can do is postulate, as
external logic, its own internal logic, as it cannot imagine another
more expressive logic. By the way, the demonstration that \Mental
recursive syntax$L_{\syn U}$ is the most expressive, seen in
^>Expressiveness>, reveals to us that this is precisely our syntax, a
conclusion that is known as the ^[Church]-^[Turing] thesis (see
^[Hofstadter]^(Hofstadter1979)).


\Section Objectivism

Logicism_{logicism} is not, in principle, either good or bad; it is just
what happens and besides, it is inevitable. What is pernicious is not to
realize its influence. Objectivism_|objectivism| is the form this error
takes on among those who do not understand that the objective and
semantic ^{reality} is the specific way we see_{see} what is external,
but who think that that is how things are; they believe that things are
as we see them.

For example, lack of imagination induces us to think that things_{thing}
exist_{existence} on their own, that the reality of things is exterior
to and independent of the subject. And, although the practical
repercussions of objectivism may be negligible, the philosophic
consequence of this fallacy is to put ^{ontology} before
^{epistemology}. Here we prefer to make the theory of knowledge the base
of philosophy and unlink ontology from ^{philosophy} in order to
attribute it to ^{psychology}. Another example is the so-called quantum
paradoxes_{paradox}, such as the wave-particle duality of the
^{electron}, which also is due to the need that we persons have, in
order to understand what is external, of thing-ifying it, that is, of
referring it to objects with meaning. ^[Bohr]^(Bohr1929) had already
thought that it would be necessary to renounce the intuitive
representation of atomic phenomena.

The need to ^{thing-ify} in order to understand is a human need, but not
a need of subjects in general. It is rooted in ^{perception},
^{learning}, and ^{emotion}, which construct things as objects with
^{meaning}. Going back to the analogy of perception with spectacles that
add labels_{label}, that we used in ^>Objective Reality Is Subjective>,
we can argue that, in order to recompose what is beyond the spectacles,
it is necessary to counteract the distorting effect of the lenses. That
is, once we know that perception, learning, and emotion are the causes
of the distortion, we are ready to reinterpret the quantum paradoxes.
The theory of subjectivity offers us this possibility, which means
subordinating ^{physics} to ^{psychology}.


\Section Self and Reality

Every syntactic expression needs a ^{subject} in order to have
^{meaning}; going further, only subjects employ syntactic expressions.
So there is no sense in talking about absolute truths, which would have
to be syntactic expressions whose meaning did not depend on any subject;
nor is there any sense in talking about objective explanations, which
would be explanations based on absolute truths. Truth_{truth} and
^{explanation} are necessarily subjective.

What is immediate is what is subjective. What is objective is a
construction. Thus it is not possible to explain what is subjective by
what is objective, rather what is subjective explains what is objective.
Yet the subject does not see itself as a subject, but as a ^{self}. It
sees itself as free to resolve problems and so it sees itself as
inhabiting ^{syntax}. I am in syntax. That is why semantic reality looks
to the subject as if it were given and independent from the subject
itself, when it actually depends on the subject to the point that it is
constructed by the subject itself.

Finally, here are three summaries that will undo ^{objectivism}:
\beginpoints
\point It is true that self is outside of the ^{reality} of things, but
it is also true that self and the real things are in the subject (see
^>Things and Concepts>).
\point In the subject's ^{world} reality and the self both fit (see
^>The World>).
\point Self and reality are in different levels of the same subject
(see ^>Levels>).
\endpoints


\Section There Are Erroneous Meanings

In the case of the physics theory used to predict_{prediction} the
behavior of the ^{electron}, neither the theory in itself, nor the
electron, which is a ^{concept} and not a ^{thing}, have ^{meaning}. The
^{subject} is who gives it the meaning, the subject who elaborates or
interprets this theory, instead of another one, because it is the one
that allows him to construct the electronic devices that, in the end,
help him to live. All meanings have their origin in the ^{apparent
problem} of survival, and this is why there is no meaning beyond the
subjective loop (see ^>The Subjective Loop>).

But we people also do things for the simple pleasure of doing them,
even, sometimes, at the risk of our lives. Let us forget risky
activities because, in spite of their high profile, they cannot be so
dangerous if we look at their low mortality statistics. Even so, it is
true that we can contemplate a sunset, or ^{sing}, for the mere pleasure
of doing it, and that does not cause us to consider these to be
activities with no sense or meaning. I am certain that, in spite of
their appearance, these activities also, even the dangerous ones,
contribute to mere survival. Our evolutionary history has constructed us
in a certain way, and we are prepared to stand a certain range of
^{relaxation} and of ^{tension} that has been useful for our survival as
a species. When our ^{environment} does not provide the adequate dosage
of relaxation or tension, we must obtain it in other ways.

The point is that, even if we cannot rationally justify why we like
^{music}, if we find it pleasant, then it has meaning. The same goes for
^{pain} that is disproportionate to the danger it signals. If it is
pain, it has, by definition, a meaning (see ^>Sentiment>). It could
happen that, due to some accident of our evolutionary history, a certain
pain may now be counterproductive, because its intensity may force us to
attend to it and, consequently, we would not attend to other less
bothersome but more dangerous pains.

This is no more than an example of the idea that genetic ^{information}
is also provisional, as we saw in ^>Knowledge Is Provisional>. But what
is interesting in this case is that only a subject, by relating all of
these data to the ^{problem of survival}, can conclude that even the
most primitive natural meanings can be erroneous. For a knower there are
no erroneous meanings; there cannot be, because, even though it uses
meanings, it cannot ^{see} them as the subject does, or, consequently,
wonder whether they are adequate or not.

It is certainly interesting to know that natural meanings can be
erroneous, but it complicates the situation enormously. Because the
simple rule that establishes that every natural meaning is necessarily
correct then fails. We people are subjects that are designed by
^{chance} and, because of this, things are not as they appear to us, as
we already mentioned in ^>Down with Objectivism!>.


\Section The Problem Is the Source of Meaning

Objectivists say that a ^{stone} is a thing that exists_{existence} on
its own and that it does not, therefore, require any explanation; the
stone simply is. On the other hand, they require an explanation for
^{God}, ^{nation}, or any other concept, in order to accept them as
existing (see ^>The Material Explanation>). For objectivism,
things_{thing} have ^{meaning}, but it is necessary to look for the
meaning of concepts_{concept}.

For us, the difference between the stone and the nation is that they are
objects that come from different phases of ^{evolution}, so that they
are images or representations that are found in different layers of the
^{world} (see ^>The World>, ^>Layers>, and ^>Self and Reality>). But, as
both are objects, and there is no difference between them in this, both
must be explained, although each turns out to have a different
explanation. It is only because of this that we distinguish things, like
the stone, from concepts, like the nation. The fact that the stone is an
object constructed by ^{perception}, ^{learning}, and ^{emotion}, but
independent of our ^{will}, and that the nation is a theoretic and
voluntary object, is circumstantial and does not fundamentally alter the
situation.

We have to look for the meanings of things as well as of concepts, and
this meaning continues to be the way of reconciling the external
conditions with the internal conditions for the resolver of the problem
of survival (see ^>Meaning> and ^>Intelligence>). And since the natural
meanings of things can be erroneous, as we showed in the previous
section (_>There Are Erroneous Meanings>), definitive meanings must be
obtained in the end from the problem of survival, which permits us to
accurately measure the importance of the different conditions, because
the final objective of all its resolvers is, specifically, its solution.
$$\hbox{Problem}
   \llave{Freedom\cr
    Condition$\,
     \left\{ \vcenter{\nointerlineskip\halign{#\hfil\crcr
      Internal\cr
      $\quad\;\bigm\updownarrow\hbox{Meaning}$\cr
      External\cr
     }}\right.$
    \cr}
$$


\Section The Limits of Knowledge

Meaning_{meaning} harmonizes desires_{desire} with ^{perception} in
mental resolvers. The mental resolvers, knowers as well as subjects,
keep in mind the spatial_{space} ^{geometry} of the conditions of the
problem they are facing, that is, they distinguish whether the
conditions are internal or external. So we can say that the meaning is
the geometry that the system formed by the problem and its resolver
adopts (see ^>Time and Space>). There is no meaning without a problem.
In our case, the apparent problem of survival is the primitive source of
meanings. Life_{life} and ^{death} are the limits of meaning, and we
cannot go beyond them.

This is a consequence of our ^{semantic} theory of subjectivity,
according to which (as we saw in ^>There Are Erroneous Meanings>) the
wave equation that describes the ^{electron} according to ^{physics} has
meaning because, in the end, it allows us to construct electronic
devices that make our lives easier.

It is not possible to explain the meaning of life, rather it is life
that gives us meaning. Explanations can go on until they reach survival,
but no further. No meta-thanatic theory that transcends death, or life,
makes sense.


\Section The Limits of Communication

If ^|communication| is the transferral of meanings_{meaning}, then the
^{theory of subjectivity} also establishes limits to communication:
given that the ^{problem} provides the meanings, there can only be
communication between two resolvers of the same problem. This result has
various consequences because, when we make limits, we are always
defining an inside and an outside, a possible and an impossible.

As the resolvers of the same problem can, in principle, share meanings,
and as all living beings_{life} are resolvers of the problem of
survival, it turns out that living beings can communicate. This is why
we understand plants_{plant} that, even though they are mere mechanisms,
search for the light of the ^{sun}. We could also interpret that the
^{moon} is searching for the most comfortable path around the ^{earth},
but this way of speaking, that gives a spirit to something that is not
alive, is always figurative. It does not work well, because, while the
plant would die if it does not find the light it yearns for, the moon is
exempt from these contingencies.

And on the contrary, since only the resolvers of the same problem can
share meanings, it turns out that we cannot communicate with anything
that is not alive. The search for extraterrestrial ^{intelligence} can
run into an insurmountable difficulty because of this. This is because
any regularity must be considered a symbolic regularity, not merely a
physical one, in order for us to give it meaning. And for a regularity
to be considered symbolic, we must suppose that, underneath the manifest
syntax, there is an underlying semantic ^{intention}. I can explain this
better with an example. The regularity of ^{quartz} crystals can be
interpreted to be the result of resolving a complicated
three-dimensional problem of energy minimization. But as the resolved
problem is not the problem of survival, we do not attribute intention to
it, and, consequently, we cannot communicate with rocks.

A ^{robot} constructed by an engineer would be another matter. It is not
alive according to the traditional definition of life, linked to the
organic chemistry of carbon. However, if it were designed with the
purpose of surviving, then it would form part of life, as defined
according to this theory of subjectivity, that is, defined as an
apparent problem (see ^>The Problem of Survival>). This robot could be
an ^{adaptor}, like a ^{thermostat} designed to maintain the
^{temperature}, but then it could not be intelligent, and communication
with it would be poor. The case of a ^{subject} robot would be very
different; this robot can be more intelligent and rational than a
person. We will return to this subject in _>Man's Successor>.


\Section The Semantic Bubble

All living_{life} beings constitute a semantic universe, or put more
humbly, a ^{semantic bubble}, because we share the same ^{problem}. This
means that, on one hand, we cannot communicate with things that aren't
alive, and, on the other hand, ^{communication} is possible, in
principle, among all living beings. But while knowers can change the
meanings of objects and we subjects can, besides, ^{see} the meanings,
learners and adaptors do not distinguish objects from meanings, and
mechanisms do not even use objects. Thus, the richest communication, and
the only communication capable of transmitting problems and resolutions
unrestrictedly, is symbolic communication between subjects.


\Section The Mental Colony

Man, \latin{homo sapiens}, is very similar genetically to the
^{chimpanzee}, \latin{pan troglodytes}, and nevertheless, in 1980,
according to ^[Ayala]^(Ayala1980), there were some hundred thousand
chimpanzees and some four and a half billion people, that is, forty-five
thousand people for every chimpanzee. So a small ^{genetic} difference
has caused an enormous evolutionary advantage, as we already mentioned
in ^>A Small Difference>.

We hold that ^{symbolization} marks this advantage, and is the most
recently acquired characteristic. If we agree that the ability to
symbolize is the most recently acquired ability, then we share all the
other cognitive characteristics with the other animal species. Thus, for
example, ^{perception}, which makes objects_{object}, and the
^{emotional system}, which makes meanings_{meaning}, are also employed
by other animal species.

Symbolization makes language with ^{recursive syntax} possible, and this
is unique to humans. Symbolization allows human associations having more
than a million members, and turns the human being into the only mammal
that, like the {ant}, forms colonies_{colony}. Because language with
recursive syntax, which makes us able to communicate and share problems,
resolutions, and solutions, allows us to reach a mental
^{specialization} equivalent to the specialization that the ants achieve
in the corporal layer (see ^>Layers>). It is interesting to note that,
according to ^[Hölldobler] and ^[Wilson]^(H\"olldobler1994), the total
weight of all the ants approximately coincides with the weight of all
the human beings.

Symbolization distinguishes us as a species and gives us an advantage
with respect to the other species. It is unique because it corresponds
to the last evolutionary step. Its advantage for survival is that it
permits us to form colonies.


\Section Culture

Symbolism_{symbolism} allows us to establish any syntactic
^{convention}, and the best convention is the one that best serves the
case of the moment. If the syntax is recursive_{recursivity}, the
syntactic machinery is completely flexible and, no matter what syntactic
modeling is needed, it is possible to define it, as we saw in ^>The
Universal Turing Machine>. In particular, we can start out from an
established syntax and extend_{extensibility} it to cover other
purposes, as in the case of mathematical, scientific, or any other
^{jargon}. Thus, syntax is conventional, that is, the syntactic objects
can have any meaning, or none, and they can refer to any object, which
can also be syntactic. And so a syntactic expression can even refer to
itself.

All of this was already known, but the essential difference between our
species and others did not seem to be that we are conventional. It is,
however. The reason, as we already know, is that with ^{recursive
syntax}, symbolism allows the representation and expression of problems,
resolutions, and solutions. This is crucial because living beings are
resolvers of the ^{apparent problem} of survival, so that the only
living beings able to represent the situation as it is, including
themselves in it, are those whose logic is symbolic, that is, we, the
subjects. Only we subjects are conscious of the problematic situation in
which we live.

To put it another way, we humans can express and communicate part of our
cognitive processes, a part which includes our ^{reality} along with our
problems and resolutions. You could say that our conscious ^{thought} is
transparent, or, more exactly, revealing_{revealing}. The consequence is
that some people can take advantage of the knowledge and wisdom of
others, alive or dead, so that the resolutions found by one person, if
they are beneficial, can be employed by any other person. This process
of shared, or distributed ^{cognition}, known by the name of ^|culture|,
is what allows us to form human colonies_{colony} with millions of
members and what, in short, has given \latin{homo sapiens} an
unprecedented evolutionary_{evolution} development during the last
thirty thousand years (see ^[Harris]^(Harris1989)). This may seem like a
long time, but it is not really, compared with the over three billion
years that have gone by since ^{life} appeared. This explosion deserves
more explanation.


\Section Darwinian Evolution

The first strategy employed by ^{life} to resolve the ^{apparent
problem} consisted of applying the process of imperfect reproduction and
natural ^{selection} discovered by ^[Darwin]^(Darwin1859) (see ^>The
Apparent Problem>). This is a two-level process: the higher level
produces the resolvers and the lower level determines the solution to be
applied, in the form of behavior (see ^>The Double Resolution>). Now we
know that this is the method the knowers use to resolve problems (see
^>The Formal Knower>). Besides, as Darwinian evolution chooses by trial
and error, it works as a knower that tests_{test}. This is why
^{Darwinian intelligence} is tentative, not semantic.

While ^{evolution} produced mechanisms, adaptors, and learners, it
maintained the two levels, but when Darwinian evolution, working itself
as a knower that tests, began to generate knowers with semantic
^{intelligence}, the process acquired an additional level. Because the
genetic endowment, when it constructed a ^{knower}, no longer completely
determined the resolution to be employed; this can be any of the
resolutions that the ^{mind} of the knower constructed is capable of
achieving. That is, two intelligences intervene to determine which
resolution will actually be applied: Darwinian intelligence, which
selects a knower, and the semantic intelligence of the knower selected,
which chooses a resolution.

The knower's semantic intelligence must consider the external situation
as well as the internal one, and within this last, the bodily situation
as well as the mental one. With this consideration of its own mental
situation, and we must remember that the mind is the knower's multiple
resolver, the knower meddles for the first time with the resolution of
the ^{problem of survival}. This problem, up to the moment the knowers
appeared, was the exclusive dominion of Darwinian evolution. And, in
order to carry out its part in the resolution of the apparent problem,
the knower must deal with desires_{desire} and sentiments, that is, with
meanings_{meaning}.


\Section Cognitive Evolution

With the knowers, the task of determining which resolution to employ was
divided between Darwinian evolution's_{Darwinian evolution}
^{intelligence} which selects the knower and the intelligence of the
selected ^{knower} itself. But neither of these two intelligences is
capable of foreseeing_{foresight} the effect of the resolutions, that
is, neither is rational_{reason}. When Darwinian evolution began to
generate subjects, which can represent problems, resolutions, and
solutions in their symbolic logic with ^{recursive syntax} (see
^>Algorithms>), evolution itself underwent a qualitative change. Because
the ^{subject}, even though it continued to be the result of the process
of Darwinian evolution, surpasses the process.

The subject completely interiorizes the resolution process of the
^{problem of survival}, as we saw in ^>Layers>. When it does this, the
subject can ponder in mere instants different resolutions that Darwinian
evolution would take generations to try out. With the subjects, the
process of evolution accelerates or, as we said, explodes, because the
physical construction of the resolver becomes unnecessary. This is why
we distinguish ^{physical evolution}, or Darwinian evolution, from
^{cognitive evolution}, which only happens in the subjects' reason.

And just as the different resolvers of the problem of survival that
Darwinian evolution found are physically different, the resolutions of
cognitive evolution are representations in the recursive syntax of the
subject's symbolic logic, and cannot be physically or perceptibly
distinguished. They cannot be distinguished because the subject's
resolutions are syntactic expressions, and, as such, are conventions
which, in themselves, have no ^{meaning}.

So we have, on one hand, ^{Darwinian evolution}, or physical evolution,
that works as a knower that tests_{test} because it is capable of
diverse resolutions that it selects by a trial-and-error procedure. And
on the other hand, we have cognitive evolution, which works through
subjects that reason, because they can represent various possible
resolutions and their consequences to themselves, and, in this way, they
can foresee the results of the different ways of resolving the problem.
Darwinian evolution is intelligent, but tentative, and cognitive
evolution is rational.

\breakif4


\Section Technology

Man could instantly become acclimatized to the cold. It wasn't necessary
for the hairiest individuals to survive better and leave a greater
number of descendents, so that after many generations the population was
mostly hairy. Using the skin of other already acclimatized animals for
warmth and ^{clothing} was enough. Wearing clothes to keep warm seems
simple, but no animal species other than our own does it. The reason is
that we are the only live subjects. But let us stop a moment to
appreciate the details of this better.

We can say that only we subjects extend ^{physical evolution} with
^{cognitive evolution} which, in this case, instead of giving us hair,
clothes us. This explanation, already very general, can be generalized
further if we consider that all of the tools_{tool}, utensils, and
artifacts that we manufacture are, like clothing, ^{prostheses} produced
by cognitive evolution that complete our physical body.

The previous explanation does not illuminate why cognitive evolution
allows subjects to manufacture tools. This is because tools are
resolutions turned into things, so that, in order to be able to
manufacture them, one has to imagine them; that is, it is necessary to
represent them to oneself internally. And only we subjects have a logic
that allows the representation of resolutions, as we saw in ^>Tools>. We
make utensils because our logic is symbolic.

Symbolic logic_{symbolic logic}, with which we interiorize the
resolution of problems, is responsible for ^|technology|, which we
define as the physical and especially the mental disposition that allows
us to manufacture tools. Tools, being resolutions made things, are the
perceivable, or physical, features of cognitive evolution. These same
tools are also the most visible aspects of culture, because ^{culture}
is the transmission of resolutions from subject to subject using
symbolic languages. This is the reason that it is correct to refer to
cognitive evolution as ^{technologic evolution} or ^{cultural
evolution}.


\Section Controlling the Environment

Constructing artifacts can be seen either as the manufacture of
^{prostheses} that extend the body, or as the modification of the
^{environment} in order to accommodate it to our body. If making
^{clothing} fits the first perspective better, building a ^{house} seems
to be better described in the second way. In the end, they are two ways
of facing the same fact: ^{cognitive evolution} acts outside of the
^{body}.

\breakif1

In fact, until now cognitive evolution only acted outside of the body.
Or at least it had only acted indirectly on the bodies through the
^{artificial selection} of domestic animals and plants for agriculture,
that ^[Darwin]^(Darwin1859) used as the first reasoning for this theory.
This is no longer so.

We subjects can ^{see} the whole situation because we can represent the
problem of survival and its resolution, the life in which we ourselves
are included, to ourselves. This is why we subjects have a
^{consciousness} of our position within the whole. And also because of
this, cognitive evolution surpasses and includes ^{physical evolution}.
In plain words: man can intervene in the processes of Darwinian
evolution, in the selection as well as in the genetic reproduction, and
modify them.


\Section The Only Living Subject

A ^{recursive syntax} is an extensible system of conventions that serves
to resolve problems because it allows the expression of problems,
resolutions, and solutions. Because they are conventional, syntactic
symbols are empty of ^{meaning} and the problems are what provide
meanings. Symbolism_{symbolism}, with semantics and recursive syntax,
was designed by Darwinian ^{evolution} because ^{life} is an ^{apparent
problem}. The ^{subject} is the resolver of the apparent problem of
survival whose logic is symbolic. And the only live subject is ^{man}.
Up to now.


\Section Man's Successor

The theoretical resolution of the ^{apparent problem} reveals to us that
^{physical evolution}, or Darwinian evolution, and ^{cognitive
evolution}, or cultural evolution, are two stages of the same process
(see ^[Elias]^(Elias1989)). And while Darwinian evolution's workings are
opportunistic and tentative, that is, they use the test method, also
called trial-and-error, of the knowers that ^{test}, cognitive
evolution, on the contrary, is symbolic and reasoned, as corresponds to
subjects, and is not opportunistic, but finalist and
teleological_{teleology}.

One example of Darwinian evolution's ^{opportunism} is the appearance of
the ^{nervous system} which permitted the step from the ^{mechanism},
capable of a single  ^{behavior}, to the ^{adaptor}, capable of various.
Because the cause that drove its appearance was possibly not this one,
but the fact that the nervous system allowed the transmission of data at
a greater distance, and so organisms with larger bodies but still with a
unitary and coordinated behavior could be constructed.

This suggests that man can improve the opportunistic design of the
Darwinian process and that the successor of \latin{homo sapiens} will be
a product of genetic ^{engineering} designed by man himself, although it
will not necessarily be based on ^{organic chemistry}. There is always
the possibility, of course, that \latin{homo sapiens} may become
extinct_{extinction}, with no descendents. Or, even worse, that the very
success of our species, or of its successor, will turn out to be a
^{plague} and finish off all ^{life}.


\Section Ethics

The only resolver of the ^{apparent problem} that can be
self-conscious_{self-consciousness} is a subject, because only a subject
can, with its symbolic reasoning, represent resolvers to itself. Thus,
only a subject can represent itself to itself. And, also because is has
^{symbolic logic}, the subject is the only resolver capable of
representing the problem it is facing to itself. Only a subject can be
conscious_{consciousness} because only a subject can completely ^{see}
the situation in which it finds itself, including the problem of
survival and itself (see ^>Consciousness and Self-Consciousness>).

By representing the problem it is facing to itself, the conscious
^{subject} reaches the origin of meanings_{meaning} and, by representing
itself to itself, the self-conscious subject reaches its own meaning as
a resolver. Consequently, every conscious and self-conscious subject is
responsible for using its freedom according to its own meaning. This is
the ethical_|ethics| responsibility of the subject that knows it is free
and mortal, that is, alive. We call the subject with ethical
responsibility a ^|person|.

Several consequences follow from these definitions. The ethical problem,
what should be done?, coincides with the ^{problem of the subject}, what
to do in order not to die?, as both point to the ^{problem of survival},
so that ^{epistemology} and ethics are one and the same. It is not by
chance, then, that the development of resolutive evolution manages to
make the subject responsible for the future. In other words,
\latin{homo sapiens}, because he is the only live subject, is the
conscience of life, and his future and that of all life is in his hands.
What an enormous responsibility man has!


\Section Ethics and Epistemology Are the Same

Ethics_{ethics} and ^{epistemology} are one and the same. It is
surprising that, while ^[Socrates] coincides with this result in his
dialogue ^<Protagoras>^(Plato-IV), in which he considers virtue
  (arete, $\grave\alpha\rho\varepsilon\tau\acute\eta$)
 to be equal to knowledge
  (episteme, $\grave\varepsilon\pi\iota\sigma\tau\acute\eta\mu\eta$),
^[Kant], on the other hand, had to write his ^<Critique of Practical
Reason>^(Kant1788) because he was incapable of including ethics in his
^<Critique of Pure Reason>^(Kant1787). The cause of this failure was
that ^[Kant] took as his paradigm of knowledge ^[Newton]'s ^{physics},
which we have classified here as a ^{material explanation}, and which
excludes ^{freedom} (see ^>The Entry Path Explanation>). This is
surprising because ^[Kant]'s ^{Copernican revolution} straightened out
the path that epistemology had lost because it had followed ^{ontology}
according to the dictates of ^[Socrates]. Ironically, for ^[Socrates]
the first thing, even before ontology, was ethics, so that his purpose
was to evade the sophists' subjectivist epistemology, which he judged to
be ethically noxious.


\Section Indoctrination

Controlling the ^{environment} also includes controlling other living
beings, although the degree of manipulation depends on the kind of
resolver that one pretends to dominate. If one is controlling a
mechanism, capable of one single behavior, then the only option is to
take advantage of this behavior or not take advantage of it; we can, if
we wish, cultivate wheat. In the case of adaptors, with various
behaviors, if one of them interests us, we can provoke it by interfering
with its perception; just by moving our hand near a fly that is sitting
we provoke it to fly away. But, given the rigidity of their behaviors,
it is impossible to train mechanisms or adaptors.

In order to ^{train} animals, they need to be learners, and it only
works if it is possible to model their ^{reality} according to our
interests. Thus, for example, ^[Lorenz]^(Lorenz1949) could {\em really}
be the mother to some geese_{goose}.

Domesticating_{domesticate} animals is possible if they are knowers,
because one can influence their assignation of meanings. A dog can be
trained to give the ^{meaning} \meaning{food} to the sound of a bell, or
to fetch ^{slippers}.

But the greatest control possible is the control that can be exercised
over a ^{subject}. The greatest domination is obtained by altering the
problem of the subject, which is its very ^{self}, because a subject
indoctrinated_{indoctrinate} in this way will use all its force of
resolution, and all of its ^{freedom}, and all of its being, to achieve
its purpose. This can even be literally so, because a subject can
consciously commit suicide if it has decided that ^{suicide} is the
solution.


\Section Suicide

A ^{knower} can kill itself if it fatally assigns an inadequate meaning
to a mortal sign, as we saw in ^>Arbitrary Signs>. But it would not be
correct to say that the knower committed suicide, because it was not its
^{intention} to die.

So only we subjects can commit suicide. Suicide_{suicide} is
contradictory to the live nature of the ^{subject}, but precisely
because it is such an extreme occurrence, it serves to show the
subject's great flexibility, and its enormous danger.


\Section Life

The ^{problem of survival}, in short, life, gives us sense and
meanings_{meaning} because we people are living subjects. Living because
we are resolvers of the apparent problem of survival in particular, not
another problem. And subjects because, faced with an apparent problem,
we are resolvers capable of presenting the problem that we are facing to
ourselves in our ^{symbolic logic} with recursive ^{syntax}. To
summarize: because we are living subjects, we are part of life and are
responsible for its future, even though we are not indispensable to it.

And ^|life| is an ^{apparent problem}. To this definition of life, which
is more a postulate than a definition (see _>The Subjective Loop>), only
two kinds of information can be aggregated. One is redundant
information, such as that we know nothing of life except that it is a
problem; the other is circumstantial information referring to its
resolution, for example, historical information about the Darwinian
^{evolution} of the species, which takes advantage of certain processes
that ^{organic chemistry} studies.

And the apparent problem, as we have seen, is only ^{freedom} and
condition. The condition distinguishes life from death. The freedom is
\dots, well, that is what I wanted to get to.


\Section Freedom

Freedom_|freedom| is one of the two parts that constitute every
^{problem}. There is no problem without freedom, nor freedom without a
problem. Because without freedom there is necessity and perhaps
^{chance}; there is ^{fatality} but there is no problem. However, there
is no freedom without limits, without conditions, and freedom with a
condition is a problem. If freedom were complete, there would be no
^{desire}, but only satisfaction. And, with total satisfaction, there
could be no problem, a problem being the opposite of satisfaction.
Freedom and problem are inseparable.

The problem is freedom and condition. And when the resolver of the
problem is complex, it must use meanings that integrate the external
conditions with its needs or internal conditions. That is why there is
no ^{meaning} without a problem, and as there is no problem without
freedom, either, it turns out that in order for there to be meaning,
there has to be freedom. Meaning is impossible without freedom.

But the meaning is in the conditions, and freedom is, precisely, the
other part of the problem. Freedom is a concept that is necessarily,
even tautologically, free of meaning. This is why ^{semantics} is
insufficient and a ^{symbolism} with semantics and recursive ^{syntax}
is necessary, in order to represent freedom. Freedom is a syntactic
concept. Freedom and symbolic logic are inseparable.

Given the problematic essence of ^{life}, freedom is an inseparable part
of life that gives us meaning and that is the source of all meaning. The
apparent problem of survival is the problem, and all other problems
derive from it; they are its subproblems. Just as freedom is limited and
only exists in problems, it turns out that all freedom derives from the
problem of survival. Freedom and life are inseparable.

The resolution of the problem of survival is an evolutionary process
that culminates in the ^{subject}, a resolver capable of representing
the problem it faces to itself, as well as representing itself to
itself. The subject has a ^{symbolic logic} that allows it to be
conscious of the problem it faces and conscious that it is, itself, a
resolver. The subject knows it is alive, that is, the subject knows that
it is free and mortal. The subject defines itself in relation to the
problem that gives it meaning: I am freedom to not die. Freedom and the
subject are inseparable.

Man, \latin{homo sapiens}, is the only living subject. Knowing that it
is alive and part of life lays upon the subject responsibilities towards
life, that is, it makes the subject a ^{person}. Most of all because,
given his possibility of surpassing Darwinian evolution, the person is
free to drastically modify the conditions of life, and with them the
whole problem of survival. Freedom and ^{ethics} are inseparable.


\Section Subjective Science

Occidental natural philosophy, or ^{science}, finds no place for
^{freedom} and, because of this, cannot study man properly. For the same
reason, it cannot comprehend ^{ethics}, or the ^{subject}, or ^{life},
or ^{symbolism}, or ^{meaning}, or the ^{problem}. A subjective science
is necessary. What we propose, with this ^{theory of subjectivity}, is
to surpass the ^{material explanation} and the ^{automatic explanation}
with the ^{problematic explanation}.

Science's difficulties arise because it limits itself to the study of
physical ^{reality}, in which there is no room for freedom. What is
physical and real is what the old logic of the subject, ^{semantics}, is
capable of representing. Because of this, the things that we see thanks
to perception are physical and real. Reality has the accumulated
experience of millions of years in its favor. But just as having two
feet and five fingers does not mean that the exterior is, in some way,
bipedal and pentidactilar, the fact that we see real things doesn't mean
that the exterior is, in some way, real, either. In all three cases, the
present situation depends on chance decisions, reinforced by their
initial success and then established irreversibly, made by evolution
millions of years ago in circumstances that probably no longer prevail.

What is theoretical has, in comparison with what is real, very little
experience, hardly even a few thousand years. Even so, ^{syntax}, which
is the subject's new logic for expressing theories and their concepts,
is the way that evolution has found to revise and broaden reality beyond
semantics. Besides, we should remember that what is peculiar about man
is, precisely, the recursive syntax that completes his symbolic logic so
that he can represent problems_{problem}, resolutions, and solutions.

The restriction that limits science to the study of physical reality
loses its foundation if real things as well as theoretical concepts are
representations and the difference between them is merely historical. We
see real things and we do not see theoretical concepts because of our
cognitive constitution, the result of our evolutionary history. Only a
subjective science that studies the whole world, things as well as
concepts, has room for ^{freedom}. Because, I repeat, we are constructed
in such a way that we cannot see freedom, and that is why we say that
freedom is a concept, not a thing.

One consequence of freedom not being real is that, even if we construct
a free ^{robot}, which must be a symbolic resolver, that is, a subject
faced with an ^{apparent problem}, we will not be able to see where we
have put freedom, simply because freedom is a concept and concepts are
beyond the reach of ^{perception}. And for the same reason,
neurologists_{neurology} will never find freedom in the human ^{brain};
it's not that it isn't there, it's that we cannot see it. Nevertheless,
in these cases, the importance of invisible concepts is greater than
that of visible things, because it is not possible to understand a
subject if we do not understand that it is free.


\Section The Emancipation of the Subject

As ^[Thiebaut]^(Thiebaut1990) points out beautifully, the greatest
revolution in ^{history} is the ^{emancipation} of the self.
^[Descartes]^(Descartes1641) marks, in ^{philosophy}, the moment that
the self begins to become independent, a process which, despite the time
that has transcurred, is still unfinished. The conception of the world
as a ^{mechanism} ruled by ^{universal laws} remits to the authority of
^{God}; so the ^{subject} will only cease to be subjected when it
acknowledges that it is free because the ^{world} is a problem, as
subjective science proposes, not an imposed order.


\Section Up with Subjectivism!

The ^{history} of events up to the present can be summarized in four
stages, that begin in the seventeenth century.

\beginpoints
\siglo{xvii} ^[Descartes], in the seventeenth century, established
the foundations of modern ^{philosophy}: the only certain thing is the
^{self}. He also pointed out the mutually irreducible nature of
^{reality} as opposed to the ^{freedom} of the self, which he resolved
by way of an ontological ^{dualism}.

\siglo{xviii} ^[Kant] pointed out that an apparatus for understanding,
which we call ^{logic} here, must be previous to understanding. But
Kantian logic, even though it was capable of representing reality, could
not represent freedom.

\siglo{xix} ^[Darwin] postulated that ^{man}, \latin{homo sapiens},
is a product of the ^{evolution} of the species. Consequently, his logic
and his self must also be products of evolution. And even his freedom.
From this point on, man cannot be understood without understanding life.

\siglo{xx} ^[Turing] invented, following ^[Gödel], the ^{syntax
engine}, dilucidating, in the process, symbolisms. A ^{symbolism} is a
recursive ^{grammatical logic}, that is, a logic of maximum
^{expressiveness}, divided in two layers: ^{semantics} and ^{syntax}.

\siglo{xxi} And now our task is to integrate the discoveries of the
four previous centuries.

\noindent
In order to do this, we only have to substitute ^[Kant]'s logic with a
symbolism, so that ^[Descartes]' ontologic dualism is transformed into a
logical dualism. There are not two different substances, but rather two
types of logical representations: semantic objects_{object}, which are
the real things_{thing} that are seen without having to think, and
syntactic objects, which are the theoretical concepts_{concept} that we
have to think but do not see.
\endpoints

Logical dualism's explanation is historical and contingent; that is, it
is Darwinian. Thus, to fit ^[Darwin], we must show that symbolism
improves the possibilities for survival. And this is so if we postulate
that ^{life} is an ^{apparent problem}, that is, exclusively freedom and
the condition of not dying. Because, in order for a resolver of an
apparent problem to be able to comprehend the complete situation, which
includes the problem with its freedom and the resolver itself as the
resolution, its logic has to be symbolic.

A symbolic subject thus defined will see himself faced with the
^{problem of survival}, that is, free, but under the condition of not
dying. And his freedom will be as genuine as the freedom of the problem
of survival. In other words, it is completely genuine, if we accept life
as problematic and absolute.


\Section Freedom Is Never Complete

Admitting that ^{freedom} is a basic scientific concept requires us to
admit that the ^{problem} is, also. And once it finds itself under
science's_{science} discipline, freedom, whose infinity inspired the
dreams of the romantics, is necessarily limited. This limitation also
infects the ^{meaning} and, from the meaning, passes on to knowledge.

\breakif2

Knowledge_{knowledge} is not absolute; it depends on the ^{subject} who,
in turn, takes the meanings from the problem that it is resolving. The
limit of freedom, the limit of meaning, and the limit of knowledge is
the same; it is the ^{apparent problem} of survival, it is life.
Life_{life} is a bubble of knowledge and freedom. Death_{death} has no
meaning.

In the end, not even freedom is a ^{transcendental concept}. No concept
transcends death. Freedom is, however, one of the fundamental concepts,
because it defines me. I am freedom to not die.


\Section Why Do We Search for Freedom?

Why do we search for ^{freedom}, if it only means problems for us?
Because we like to advance in resolving problems and, even more, we like
to solve them. It makes us happy because we are designed to solve
problems. But, in order to solve a problem, there needs to be a problem.
That is why we are curious, inquisitive, and for this same reason we
search for the freedom that implies having problems and many different
ways of resolving them. We flee from the tedium involved in mechanical
and repetitive action, which may be effective but never problematic.


\Section Knowledge Is Not Absolute

The problematic and paradoxical nature of ^{life} and of the ^{subject}
limit ^{knowledge}. This may seem inconvenient, but it would be mistaken
to believe the contrary. And, on the other hand, the fact that knowledge
is not absolute but depends on the subject who, in turn, is nothing more
than a resolver of the apparent problem, generalizes two other
scientific principles: that ^{space} is not absolute, proposed by
^[Galileo], and that ^{time} is not absolute, established by
^[Einstein].

Besides, understanding that life is an ^{apparent problem}, allows us to
integrate Darwinian ^{evolution} and ^{cultural evolution} in a single
process that hinges on the subject. This subject is thus defined, and
enjoys such an advantageous position, because his logic is
symbolic_{symbolic logic}, so that he is conscious of the problem that
he faces and conscious of himself_{self-consciousness} as a resolver of
the problem. That is, life's problematic nature explains why symbolic
language, ^{culture}, ^{technology}, ^{consciousness}, ^{ethics}, and
^{freedom} coincide in man, the only living subject.

\vskip1\baselineskip
\centerline{\fonttwo The End}

\endinput
